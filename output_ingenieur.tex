




\documentclass[%
	fontsize=10pt,%
	twoside,%
	headings=optiontoheadandtoc,%
	showtrims]{scrbook}
\usepackage[T1]{fontenc}
\usepackage[utf8]{inputenc}
\usepackage{scrlayer-scrpage}
\usepackage[french,english,ngerman, italian, main=ngerman]{babel}
\babelprovide[import]{arabic}
\usepackage{multicol}

	\usepackage{microtype}% verbesserter Randausgleich
	
\usepackage[
paperwidth=125mm,%
paperheight=185mm,%
layoutwidth=115mm,%
layoutheight=175mm,%
layouthoffset=5mm,%
layoutvoffset=5mm,%
inner=15mm,%
outer=7.5mm,%
width=92.5mm,%
height=153.7mm,%
top=10.3mm,%
bottom=11mm,%
footskip=0mm,%
headheight=8mm,%
headsep=1.5mm,%
showcrop
]{geometry}
\usepackage{xurl}

\usepackage[autostyle=false,style=german,german=guillemets]{csquotes}

\usepackage[hidelinks]{hyperref}

	\usepackage{graphicx}

\usepackage{fontspec}
\usepackage{makeidx}
\usepackage[unbalanced=true]{idxlayout}
\usepackage{enumitem}
\usepackage{ragged2e}
\RequirePackage{etoolbox}

\usepackage{picture}
\usepackage{tikz}
\usepackage[absolute]{textpos}
\usepackage{atbegshi}
\usepackage{textcase}
\usepackage{environ}
\usepackage{ifthen}


\usepackage{ifluatex}
\ifluatex
  \edef\pdfpageattr{\pdfvariable pageattr}
  \edef\pdfcompresslevel{\pdfvariable compresslevel}
  \edef\pdfobjcompresslevel{\pdfvariable objcompresslevel}
\fi
\pdfcompresslevel=0
\pdfobjcompresslevel=0
\pdfpageattr{
    /MediaBox [0 0 354.330708625 524.409448765]
    /BleedBox [0 0 354.330708625 524.409448765]
    /CropBox [14.173228345 14.173228345 340.15748027999996 510.23622042]
    /TrimBox [14.173228345 14.173228345 340.15748027999996 510.23622042]
}




%---------- uncomment for grid --------------

	\usepackage{eso-pic}



\geometry {
paperwidth=125mm,%
paperheight=185mm,%
layoutwidth=115mm,%
layoutheight=175mm,%
layouthoffset=5mm,%
layoutvoffset=5mm,%
inner=15mm,%
outer=7.5mm,%
width=92.5mm,%
height=153.7mm,%
top=10.3mm,%
bottom=11mm,%
footskip=0mm,%
headheight=8mm,%
headsep=1.5mm,%
showcrop
}

\newcommand{\smallpage}{%
	\newgeometry{
paperwidth=125mm,%
paperheight=185mm,%
layoutwidth=115mm,%
layoutheight=175mm,%
layouthoffset=5mm,%
layoutvoffset=5mm,%
inner=15mm,%
outer=7.5mm,%
width=92.5mm,%
height=153.7mm,%
top=10.3mm,%
bottom=11mm,%
footskip=0mm,%
headheight=8mm,%
headsep=1.5mm,%
showcrop
}%
}
\newcommand{\normalpage}{%
	\newgeometry{
paperwidth=125mm,%
paperheight=185mm,%
layoutwidth=115mm,%
layoutheight=175mm,%
layouthoffset=5mm,%
layoutvoffset=5mm,%
inner=15mm,%
outer=7.5mm,%
width=92.5mm,%
height=153.7mm,%
top=10.3mm,%
bottom=11mm,%
footskip=0mm,%
headheight=8mm,%
headsep=1.5mm,%
showcrop
}%
}
\newcommand{\widepage}{%
	\newgeometry{
paperwidth=125mm,%
paperheight=185mm,%
layoutwidth=115mm,%
layoutheight=175mm,%
layouthoffset=5mm,%
layoutvoffset=5mm,%
inner=15mm,%
outer=7.5mm,%
width=92.5mm,%
height=153.7mm,%
top=10.3mm,%
bottom=11mm,%
footskip=0mm,%
headheight=8mm,%
headsep=1.5mm,%
showcrop
}%
}
\newcommand{\tallpage}{%
	\newgeometry{
paperwidth=125mm,%
paperheight=185mm,%
layoutwidth=115mm,%
layoutheight=175mm,%
layouthoffset=5mm,%
layoutvoffset=5mm,%
inner=15mm,%
outer=7.5mm,%
width=92.5mm,%
height=160.6mm,%
top=8.9mm,%
bottom=5.5mm,%
footskip=0mm,%
headheight=8mm,%
headsep=1.5mm,%
showcrop
}%
}
\newcommand{\bastardpage}{%
	\newgeometry{
paperwidth=125mm,%
paperheight=185mm,%
layoutwidth=115mm,%
layoutheight=175mm,%
layouthoffset=5mm,%
layoutvoffset=5mm,%
inner=15mm,%
outer=7.5mm,%
width=92.5mm,%
height=164.56mm,%
top=4mm,%
bottom=6.44mm,%
footskip=0mm,%
headheight=8mm,%
headsep=1.5mm,%
showcrop
}%
}
\newcommand{\tocpage}{%
	\newgeometry{
paperwidth=125mm,%
paperheight=185mm,%
layoutwidth=115mm,%
layoutheight=175mm,%
layouthoffset=5mm,%
layoutvoffset=5mm,%
inner=12.5mm,%
outer=12.5mm,%
width=90mm,%
height=153.7mm,%
top=10.3mm,%
bottom=11mm,%
footskip=0mm,%
headheight=8mm,%
headsep=1.5mm,%
showcrop
}%
}

% -----------------
% Spaces
% -----------------

\ifundef{\genericindent}{\newlength{\baselinegrid}}{}%
\setlength{\baselinegrid}{3.92mm}%


\ifundef{\genericindent}{\newlength{\genericindent}}{}%
\setlength{\genericindent}{7.5mm}%

\ifundef{\spacebetweenlabelandtext}{\newlength{\spacebetweenlabelandtext}}{}%
\setlength{\spacebetweenlabelandtext}{4.5mm}


% ------------------------------------------------------------------------
% font Settings
% ------------------------------------------------------------------------



\directlua{
    fonts.handlers.otf.addfeature {
        name = "onecb",
        type = "substitution",
        data = {
                ["1"] = "one.tf",
		}
	}
}

\directlua{
    fonts.handlers.otf.addfeature {
        name = "sixcb",
        type = "substitution",
        data = {
                ["6"] = "six.ss11",
		}
	}
}

\directlua{
    fonts.handlers.otf.addfeature {
        name = "ninecb",
        type = "substitution",
        data = {
                ["9"] = "nine.ss11",
		}
	}
}

\directlua {
	luaotfload.add_fallback
		("myfallback", {
			"DejaVuSans:mode=harf;script=grek;",
			"cmuserif:mode=node;script=cyrl;",
			"NotoSansBengali:mode=harf;script=bng2",
			"NotoColorEmoji:mode=harf;"
		})
}

% --------------------------------
% - Font Classes                 -
% --------------------------------

\newfontfamily\Lyon{LyonText}[
	Extension   = .otf,
	UprightFont = *-Regular,
	ItalicFont  = *-RegularItalic,
	Numbers		= Lining,
	RawFeature  = {fallback=myfallback}
 ]

\newfontfamily\LyonBold{LyonText}[
	Extension   = .otf,
	UprightFont = *-Regular,
	ItalicFont  = *-RegularItalic,
	Numbers		= Lining,
	RawFeature  = {fallback=myfallback}
 ] 

\newfontfamily\LyonFig{LyonText}[
	Extension   = .otf,
	UprightFont = *-Regular,
	ItalicFont  = *-RegularItalic,
	Numbers		= Monospaced,
	RawFeature  = {fallback=myfallback}
 ] 

\newfontfamily\Diatype{ABCDiatype}[
	Extension   = .otf,
	UprightFont = *-Regular,
  	BoldFont 	= *-Bold,
	ItalicFont  = *-RegularItalic,
	BoldItalicFont  = *-BoldItalic,		  
	RawFeature=+onecb;+sixcb;+ninecb;,
	RawFeature={fallback=myfallback}
 ]

\newfontfamily\DiatypeFig{ABCDiatype}[
	Extension   = .otf,
	UprightFont = *-Regular,
  	BoldFont 	= *-Bold,
	ItalicFont  = *-RegularItalic,
	BoldItalicFont  = *-BoldItalic,		  
	RawFeature=+onecb;+sixcb;+ninecb;,
	RawFeature={fallback=myfallback}
 ] 

\newfontfamily\DiatypeBold{ABCDiatype}[
	Extension   = .otf,
	UprightFont = *-Bold,
	ItalicFont  = *-BoldItalic,
	RawFeature=+onecb;+sixcb;+ninecb;,
	RawFeature={fallback=myfallback}
 ] 

\newfontfamily\DiatypeBoldFig{ABCDiatype}[
	Extension   = .otf,
	UprightFont = *-Bold,
	ItalicFont  = *-BoldItalic,
	RawFeature=+onecb;+sixcb;+ninecb;,
	RawFeature={fallback=myfallback}
 ] 

\setmainfont{ABCDiatype}[
	Extension   = .otf,
	UprightFont = *-Regular,
  	BoldFont 	= *-Bold,
	ItalicFont  = *-RegularItalic,
	BoldItalicFont  = *-BoldItalic,		  
	RawFeature=+onecb;+sixcb;+ninecb;,
	RawFeature={fallback=myfallback}
 ]

% --------------------------------
% - Font Commands                -
% --------------------------------

\newcommand{\grotesk}[3]{%
	\fontsize{#1}{#2}%
	\Diatype%
	\addfontfeature{LetterSpace=#3}%
	\selectfont%
}

\newcommand{\groteskbold}[3]{%
	\fontsize{#1}{#2}%
	\DiatypeBold%
	\addfontfeature{LetterSpace=#3}%
	\selectfont%
}

\newcommand{\groteskfig}[3]{%
	\fontsize{#1}{#2}%
	\DiatypeFig%
	\addfontfeature{LetterSpace=#3}%
	\selectfont%
}

\newcommand{\groteskboldfig}[3]{%
	\fontsize{#1}{#2}%
	\DiatypeBoldFig%
	\addfontfeature{LetterSpace=#3}%
	\selectfont%
}

\newcommand{\antiqua}[3]{%
	\fontsize{#1}{#2}%
	\Lyon%
	\addfontfeature{LetterSpace=#3}%
	\selectfont%
}

\newcommand{\antiquabold}[3]{%
	\fontsize{#1}{#2}%
	\LyonBold%
	\addfontfeature{LetterSpace=#3}%
	\selectfont%
}

\newcommand{\antiquafig}[3]{%
	\fontsize{#1}{#2}%
	\LyonFig%
	\addfontfeature{LetterSpace=#3}%
	\selectfont%
}

% --------------------------------
% - Grotesk / Writer Settings    -
% --------------------------------

\newcommand{\regularfontgrotesk}{%
	\normalfont%
	\grotesk{9.8bp}{12bp}{.05}%
}
\newcommand{\smallfontgrotesk}{%
	\normalfont%
	\grotesk{7bp}{8.5bp}{.15}%
}
\newcommand{\largefontgrotesk}{%
	\normalfont%
	\grotesk{16bp}{18.5bp}{.20}%
}

% --------------------------------
% - Antiqua / Writer Settings    -
% --------------------------------

\newcommand{\regularfontantiqua}{%
	\normalfont%
	\antiqua{9.3bp}{12bp}{.05}%
}
\newcommand{\smallfontantiqua}{%
	\normalfont%
	\antiqua{7bp}{8.5bp}{.15}%
}
\newcommand{\largefontantiqua}{%
	\normalfont%
	\antiqua{16bp}{18.5bp}{.20}%
}
\newcommand{\regularfontantiquabold}{%
	\normalfont%
	\antiquabold{9.5bp}{12bp}{.05}%
}
\newcommand{\smallfontantiquabold}{%
	\normalfont%
	\antiquabold{7bp}{8.5bp}{.15}%
}

% --------------------------------
% - Fonts for Preface/Toc/Lists  -
% --------------------------------

\newcommand{\regularfont}{%
	\regularfontgrotesk%
}
\newcommand{\smallfont}{%
	\smallfontgrotesk%
}

\newcommand{\regularfontfig}{%
	\normalfont%
	\groteskfig{9.8bp}{12bp}{.05}%
}

% BOLD VERSIONS

\newcommand{\smallfontbold}{%
	\normalfont%
	\groteskbold{7bp}{8.5bp}{.25}%
}
\newcommand{\regularfontbold}{%
	\normalfont%
	\groteskbold{9.8bp}{12bp}{.05}%
}
\newcommand{\regularfontboldcaps}{% a little bit wider spaced for all caps situations
	\normalfont%
	\groteskbold{9.8bp}{12bp}{.20}%
}
\newcommand{\largefont}{%
	\normalfont%
	\groteskbold{14.5bp}{16bp}{.20}%
}
\newcommand{\hugefont}{%
	\normalfont%
	\groteskbold{36.5bp}{36.4bp}{.30}%
}
\newcommand{\smallfontboldfig}{%
	\normalfont%
	\groteskboldfig{7bp}{8.5bp}{.15}%
}

% --------------------------------
% - GTA/Intercom Template Switch -
% --------------------------------


\newcommand{\footnotemarkfontfig}{%
	\normalfont%
	\antiquafig{5bp}{8.5bp}{.15}%
}

\newcommand{\smallfontdefaultfig}{%
	\normalfont%
	\antiquafig{7bp}{8.5bp}{.15}%
}

\newcommand{\smallfontdefault}{%
	\smallfontantiqua%
}

\newcommand{\smallfontdefaultbold}{%
	\smallfontantiquabold%
}

\newcommand{\regularfontdefault}{%
	\regularfontantiqua%
}

\newcommand{\regularfontdefaultbold}{%
	\regularfontantiquabold%
}

\newcommand{\largefontdefault}{%
	\largefontantiqua%
}



\def\UrlFont{\smallfontdefault}
\uccode`ß="1E9E

% ------------------------------------------------------------------------
% HR: Newpage
% ------------------------------------------------------------------------
\makeatletter
\renewcommand{\hrulefill}{\newpage\par\@afterindentfalse\@afterheading}
\makeatother

% ------------------------------------------------------------------------
% Image Download Helper Function
% ------------------------------------------------------------------------


	\newcommand*{\convert}[2]{%
		\IfFileExists{#2}{}{%
			\immediate\write18{wget -o #2.log --tries=2 -O #2 "\detokenize{#1}"; convert -flatten -colorspace gray -density 300 #2 #2.jpg}%
		}%
	}%


	\makeatother%
	\newcommand{\geometrylen}[1]{\csname Gm@#1\endcsname}%
	\makeatletter%
	\newbool{printlof}%
	\usepackage{afterpage}%
	\usepackage{settobox}%
	\usepackage[export]{adjustbox}[2011/08/13]
	\newsavebox{\capbox}%
	\newsavebox{\imgbox}%
	\newcommand*{\placeoriginal}[7]{%
		% placeoriginal{originalfile}{localfile}{captions}{variant}{landscape|portrait|auto}{byline}{label}
		% variant is 0, 1, 2, 3 - and can be set in rokfor writer
		% Version 3: Double Side Scaling
		\ifthenelse{\equal{#4}{3}}{%
			\ifoddpage
				\afterpage{%
		\ifundef{\doubleimagewidth}{\newlength{\doubleimagewidth}}{}%
		\setlength{\doubleimagewidth}{215mm}%
		\begin{figure}[tp]%
			\noindent\begin{minipage}[t][1\textheight]{1\doubleimagewidth}%
			\includegraphics[width=1\doubleimagewidth, keepaspectratio]{#2}%
			\end{minipage}%
		\end{figure}%
		\begin{figure}[tp]%
			\ifundef{\doubleimageshift}{\newlength{\doubleimageshift}}{}%
			\setlength{\doubleimageshift}{.5\doubleimagewidth+15mm}%
			\hspace*{-1\doubleimageshift}%
			\noindent\begin{minipage}[t][1\textheight]{1\doubleimagewidth}%
			\begin{flushright}%	
			\includegraphics[width=1\doubleimagewidth, keepaspectratio]{#2}%
			\end{flushright}%	
			\end{minipage}%
			\ifblank{#6}{}{\captionlistentry{#6}\global\booltrue{printlof}}%
			\label{#7}%
		\end{figure}%
}%
			\else
				\afterpage{\afterpage{%
		\ifundef{\doubleimagewidth}{\newlength{\doubleimagewidth}}{}%
		\setlength{\doubleimagewidth}{215mm}%
		\begin{figure}[tp]%
			\noindent\begin{minipage}[t][1\textheight]{1\doubleimagewidth}%
			\includegraphics[width=1\doubleimagewidth, keepaspectratio]{#2}%
			\end{minipage}%
		\end{figure}%
		\begin{figure}[tp]%
			\ifundef{\doubleimageshift}{\newlength{\doubleimageshift}}{}%
			\setlength{\doubleimageshift}{.5\doubleimagewidth+15mm}%
			\hspace*{-1\doubleimageshift}%
			\noindent\begin{minipage}[t][1\textheight]{1\doubleimagewidth}%
			\begin{flushright}%	
			\includegraphics[width=1\doubleimagewidth, keepaspectratio]{#2}%
			\end{flushright}%	
			\end{minipage}%
			\ifblank{#6}{}{\captionlistentry{#6}\global\booltrue{printlof}}%
			\label{#7}%
		\end{figure}%
}}%
			\fi
		}%
		{%
		% Version 2: Full Placement on Page, center, 100% size
		\ifthenelse{\equal{#4}{2}}{%
			\afterpage{%
				\newpage%
				\noindent%
				\begin{textblock*}{\paperwidth}(0mm,0mm)%
				\noindent\begin{minipage}[c][1\paperheight]{1\paperwidth}%
				\noindent\begin{figure}%
					\noindent\vspace*{\fill}%
					\includegraphics[center]{#2}%
					\vspace*{\fill}%
					\ifblank{#6}{}{\captionlistentry{#6}\global\booltrue{printlof}}%					
					\label{#7}%
				\end{figure}%
				\end{minipage}%
				\end{textblock*}%
			\null%
			\newpage}
		}%
		{%
		% Version 1: Full Placement on Layout, center, 100% size OR no captions!
		\ifthenelse{\equal{#4}{1} \OR \equal{#3}{}}{%
			\begin{figure}%
				\noindent\begin{minipage}[c][1\textheight]{1\textwidth}%
				\includegraphics[center]{#2}%
				\ifblank{#6}{}{\captionlistentry{#6}\global\booltrue{printlof}}%
				\label{#7}%
				\end{minipage}%
			\end{figure}%		
		}%
		{%
		% Default: Rotation (Landspace to Portrait) and optimized Scaling according to caption size
			{%
				\begin{figure}[tp]%
				\noindent%
				\ifundef{\myheight}{\newlength{\myheight}}{}%
				\ifundef{\imgwidthfinal}{\newlength{\imgwidthfinal}}{}%
				\ifundef{\mywidth}{\newlength{\mywidth}}{}%
				\ifundef{\capheight}{\newlength{\capheight}}{}%
				\ifundef{\imgwidth}{\newlength{\imgwidth}}{}%
				\ifundef{\layoutheight}{\newlength{\layoutheight}}{}%		
				\ifthenelse{\equal{#5}{landscape}}{%
					\setlength{\layoutheight}{\textwidth}%
					\noindent%
					\smallfontdefault%
					\setlength{\myheight}{\layoutheight}%
	% CYCLE 1		
					\savebox{\imgbox}{\noindent\includegraphics[width=\textheight,height=\myheight, keepaspectratio]{#2}}%
					\settoboxwidth{\imgwidth}{\imgbox}%
					\savebox{\capbox}{\parbox[b]{\imgwidth}{\caption*[]{#3}}}%
					\settoboxtotalheight{\capheight}{\capbox}%
					\setlength{\myheight}{\layoutheight-\capheight}%
	% CYCLE 2
					\savebox{\imgbox}{\noindent\includegraphics[width=\textheight,height=\myheight, keepaspectratio]{#2}}%
					\settoboxwidth{\imgwidth}{\imgbox}%
					\savebox{\capbox}{\parbox[b]{\imgwidth}{\caption*[]{#3}}}%
					\settoboxtotalheight{\capheight}{\capbox}%
					\setlength{\myheight}{\layoutheight-\capheight+\baselineskip}%
	% CYCLE 3 -> placement							
					\savebox{\imgbox}{\noindent\includegraphics[width=\textheight,height=\myheight, keepaspectratio]{#2}}%		
					\settoboxwidth{\imgwidthfinal}{\imgbox}%
					\rotatebox[origin=c]{90}{%
					\noindent\begin{minipage}[t][\textwidth]{\textheight}%
						\begin{flushright}%							
							\begin{minipage}{\imgwidthfinal}%
								\usebox{\imgbox}%
								\\
								\parbox[b]{\imgwidthfinal}{\caption[]{#3}\label{#7}\ifblank{#6}{}{\captionlistentry*{#6}\global\booltrue{printlof}}}%
							\end{minipage}%
						\end{flushright}%
					\end{minipage}%
					}%
				}%
				{%
					\setlength{\layoutheight}{\textheight}%
					\noindent\begin{minipage}[t][1\textheight]{1\textwidth}%
					\setlength{\myheight}{\layoutheight}%	
		% CYCLE 1		
						\sbox0{\includegraphics[width=1\textwidth,height=\myheight,keepaspectratio]{#2}}%
						\setlength{\mywidth}{\wd0}%
						\sbox1{\begin{minipage}{\mywidth}\caption*[]{#3}\end{minipage}}%
						\setlength{\myheight}{\layoutheight-\ht1-2\abovecaptionskip}%	
		% CYCLE 2
						\sbox2{\includegraphics[width=1\textwidth,height=\myheight,keepaspectratio]{#2}}%
						\setlength{\mywidth}{\wd2}%		
						\sbox3{\begin{minipage}{\mywidth}\caption*[]{#3}\end{minipage}}%
						\setlength{\myheight}{\layoutheight-\ht3-2\abovecaptionskip}%
						\sbox4{\includegraphics[width=1\textwidth,height=\myheight, keepaspectratio]{#2}}%
		% CYCLE 3 -> placement	
						\begin{minipage}{\wd4}%
							\usebox4%
							\caption[]{#3}\label{#7}\ifblank{#6}{}{\captionlistentry*{#6}\global\booltrue{printlof}}\end{minipage}%			
					\end{minipage}%
				}%
				\end{figure}%
			}%
			}%
		}%			
		}%
	}%


	\newcommand*{\placeimage}[4]{%
		% placeimage(remote, local, captions, alttext) 
		\convert{#1}{#2}%
		\begin{figure}%
		\noindent\begin{minipage}[c][0.75\textheight]{1\textwidth}%
		\centering%
		\includegraphics[width=1\textwidth,height=0.5\textheight,keepaspectratio]{#2.jpg}%
		\setcapmargin[0mm]{0mm}%
		\caption*{\centering\smallfont #3 \\ #4}%
		\end{minipage}%
		\end{figure}%
	}%


	\newenvironment{paragraphalternate}
	    {\begin{center}
    	\begin{tabular}{|p{0.9\textwidth}|}
    	\hline\\
    	}
    	{ 
    	\\\\\hline
    	\end{tabular} 
    	\end{center}
    	}


% ------------------------------------------------------------------------
% Headings, Footer and Sections
% ------------------------------------------------------------------------

\pagestyle{scrheadings}
\setkomafont{pagehead}{\smallfontbold}
\setkomafont{pagefoot}{\smallfontbold}
\setkomafont{pagenumber}{\smallfontboldfig}
\clearpairofpagestyles

\makeatletter
\newcommand{\rightorleftmark}{%
  \begingroup\protected@edef\x{\rightmark}%
  \ifx\x\@empty
    \endgroup\MakeTextUppercase\leftmark
  \else
    \endgroup\MakeTextUppercase\rightmark
  \fi}
\makeatother

\newcommand{\bothhead}{
	\lohead{\rightorleftmark}%
	\ohead[\pagemark]{\pagemark}
}

\newpairofpagestyles{onlypage}{%
  \ohead[\pagemark]{\pagemark}%	
  }

\renewcommand*\raggedsection{\centering}


\RedeclareSectionCommand[%
 pagestyle=empty,%
 beforeskip=0pt,%
 font=\hugefont\vspace{-4.5mm},%
 afterindent=false,%
]{chapter}


\makeatletter 
\renewcommand\chapterlinesformat[3]{%
  \@hangfrom{#2}{\phantomsection\MakeUppercase{#3}}%
} 
\makeatother

\RedeclareSectionCommand[%
 beforeskip=2\baselineskip,%
 afterskip=\baselineskip,%
 font=\largefont\MakeTextUppercase,%
 afterindent=false,%
]{section}

\RedeclareSectionCommand[%
 beforeskip=2\baselineskip,%
 afterskip=\baselineskip,%
 font=\regularfontboldcaps\MakeTextUppercase,%
 afterindent=false,%
]{subsection}

\RedeclareSectionCommand[%
 beforeskip=2\baselineskip,%
 afterskip=\baselineskip,%
 font=\regularfontbold,%
 afterindent=false,%
]{subsubsection}

\RedeclareSectionCommand[%
 beforeskip=0.5\baselineskip,%
 afterskip=0.5\baselineskip,%
 font=\regularfontdefault\emph,%
 afterindent=false,%
]{paragraph}

% ------------------------------------------------------------------------
% Section Command used in config.js, used for h1 ... hX from Markdown
% ------------------------------------------------------------------------
%\newcommand{\printsectiontitle}[2][#2]{\section[#1]{#2}}
\newcommand\printsectiontitle[2][\DefaultOpt]{%
  \def\DefaultOpt{#2}%
  \hrulefill%
  \TPoptions{absolute=false}%
  \begin{textblock*}{1\textwidth}(0mm,-0.9\baselineskip)%
	  \section[#1]{#2}%
  \end{textblock*}%
  \TPoptions{absolute=true}%
  \addsecmark{#1}%
  \ifthispageodd{\thispagestyle{onlypage}}{}% Only suppress header on odd page
  \vspace{11\baselinegrid}%
}

% ------------------------------------------------------------------------
% Paragraphen & Seitenumbrüche
% ------------------------------------------------------------------------

% Ohne Blindzeile 

\setlength{\parindent}{\genericindent}
\setlength{\JustifyingParindent}{\genericindent}
\KOMAoptions{parskip=false} 

\setlength{\topsep}{0.75em plus 0.1em minus 0.1em}	

\clubpenalty = 10000
\widowpenalty = 10000
\displaywidowpenalty = 10000
\brokenpenalty=100				% 100
\hbadness=10000					% 1000 ab wann werden schlechte linien gemeldet
\hfuzz=0.1pt					% 0.1pt wie stark darf eine linie überlappen
\looseness=1					% anzahl zeilen, mit denen latex einen paragraphen verlängern kann um schusterjungen zu vermeiden
%\flushbottom
\raggedbottom

% Mehr Trennungen
\pretolerance=150
\tolerance=500					% 200 je weniger desto weniger tolerant, d.h. weniger trennungen
\hyphenpenalty=50				% 50
\exhyphenpenalty=0				% 50 Explizite Trennzeichen
\clubpenalty=0					% 150 Trennung an erster Zeile eines Absatzes
\emergencystretch=2em
\doublehyphendemerits=1000000

% Weniger Trennungen
% \pretolerance=100
% \tolerance=200					% 200 je weniger desto weniger tolerant, d.h. weniger trennungen
% \hyphenpenalty=100				% 50
% \emergencystretch=3em



% ------------------------------------------------------------------------
% Language
% ------------------------------------------------------------------------

\selectlanguage{ngerman}

% ------------------------------------------------------------------------
% ToC (ETOC)
% ------------------------------------------------------------------------

\usepackage{etoc}
\usepackage{tabularx}
\newcommand\numberprint[1]{\ifnum #1 < 10 0\fi #1}

\setcounter{secnumdepth}{-2}
\etocsetstyle{chapter}
{\nopagebreak[4]}
{\noindent\par\bigskip\leavevmode\leftskip 0cm\relax\noindent}
{\begin{tabularx}{\textwidth}{X r}\raggedright\leftskip=7.5mm\parindent=-\leftskip\regularfontboldcaps\etocname & \regularfontbold\numberprint{\etocthepage}\end{tabularx}\vspace{-0.25mm}\par}
{\pagebreak[1]}

\etocsetstyle{section}
{\nopagebreak[4]}
{\leavevmode\leftskip 0cm\relax}
{\begin{tabularx}{\textwidth}{X r}\raggedright\leftskip=15mm\parindent=-7.5mm\regularfontbold\etocname & \regularfontbold\numberprint{\etocthepage}\end{tabularx}\vspace{-0.25mm}\par}
{\pagebreak[1]}

\etocsetstyle{subsection}
{\nopagebreak[4]}
{\leavevmode\leftskip 0cm\relax}
{\begin{tabularx}{\textwidth}{X r}\raggedright\leftskip=15mm\parindent=-7.5mm\regularfontbold\etocname & \regularfontbold\numberprint{\etocthepage}\end{tabularx}\vspace{-0.25mm}\par}
{\pagebreak[1]}
\etocsettocstyle {}{}

% ------------------------------------------------------------------------
% Title Page
% ------------------------------------------------------------------------


\makeatletter
\renewcommand{\maketitle}{%
	\hugefont%
	\begin{center}%
	\uppercase{%
	Der Ingenieur
	\vfill Grammatik eines Hoffnungs\-trägers
	\vfill Robert Leucht
	}%
	\end{center}
}
\makeatother


% ------------------------------------------------------------------------
% Footnote Styling
% ------------------------------------------------------------------------

\setfootnoterule{0pt}
\deffootnote[\genericindent]{0em}{0em}{% [labelwidth]{labelindent}{paragraph indent}
  \enskip%label definition
}
\setkomafont{footnote}{\smallfontdefaultfig\thefootnotemark\enskip\smallfontdefault}
\deffootnotemark{\textsuperscript{\footnotemarkfontfig{\thefootnotemark}}}
\setlength{\footnotesep}{0.5\baselineskip}
\makeatletter
\@removefromreset{footnote}{chapter}
\makeatother

% ------------------------------------------------------------------------
% Index
% ------------------------------------------------------------------------

\idxlayout{
	columns=1,
	justific=raggedright,
	indentunit=0mm,
	hangindent=\spacebetweenlabelandtext
}
\setlength{\indexcolsep}{7.5mm}
\makeatletter
\renewcommand{\idx@@heading}{}%
\renewcommand{\ila@prologue}{}%
\renewcommand{\indexfont}{\smallfontdefault}
\renewcommand{\indexname}{} 
\makeindex

% ------------------------------------------------------------------------
% Blockquote
% ------------------------------------------------------------------------

\renewenvironment{quote}{%
  \addmargin[\genericindent]{0pt}%
  \KOMAoptions{parskip=true}% falls das wirklich gewünscht wird
%  \KOMAoptions{parskip=false}% sonst ggf. so
  \ifdim\parskip>0pt\else\addvspace{\intextsep}\fi
}{%
  \par
  \endaddmargin\vspace{\intextsep}
}
% No indent after Quote
\usepackage{noindentafterquote}
\NoIndentAfterEnv{quote}

% ------------------------------------------------------------------------
% Verbatim
% ------------------------------------------------------------------------


% ------------------------------------------------------------------------
% Textttt
% ------------------------------------------------------------------------

% \renewcommand{\texttt}{\mbox} -> keep on one line
\usepackage{hyphenat}
\renewcommand{\texttt}{\nohyphens} % disallow hypenation, except manual breaks and spaces

% ------------------------------------------------------------------------
% two col description
% ------------------------------------------------------------------------

\newlist{twocoldescription}{description}{1}
\setlist[twocoldescription,1]{
	leftmargin=0mm, 
	listparindent=\genericindent,
	labelsep=0.5em,
	rightmargin=0mm,
	noitemsep,
	style=unboxed,
	before=\csname par\endcsname\raggedright,
	font=\hspace{\genericindent}\normalfont\smallfontdefaultbold
}

\makeatletter
\newenvironment{twocolumnlist}
                {%
					\smallfontdefault%
					\raggedcolumns\begin{multicols}{2}%
					\RaggedRight%
					\setlength{\parindent}{\genericindent}%
					\renewenvironment{description}{\begin{twocoldescription}}{\end{twocoldescription}}%
				}
                {\end{multicols}}
\makeatother

% ------------------------------------------------------------------------
% Itemized Lists
% ------------------------------------------------------------------------


\setlist[itemize,1]{
	leftmargin=12mm, 
	listparindent=0em,
	itemindent=0mm, 
	labelsep=0mm, 
	labelwidth=4.5mm,
	labelindent=0mm,
	label=\regularfont\symbol{"25CB},
	rightmargin=0mm,
	parsep=0mm,
	itemsep=0mm,
	topsep=0mm,
	align=left
}

\setlist[enumerate,1]{
	leftmargin=15mm, 
	listparindent=0em,
	itemindent=0mm, 
	labelsep=0mm, 
	labelwidth=\genericindent,
	labelindent=0mm,
	label=\arabic*.,
	rightmargin=0mm,
	parsep=0mm,
	itemsep=0mm,
	topsep=0mm,
	align=left
}

\setlist[description,1]{
	leftmargin=0mm, 
	listparindent=0mm,
	labelsep=2.5mm,
	rightmargin=0mm,
	noitemsep,
	style=unboxed,
	parsep=0mm,
	itemsep=0mm,
	topsep=0mm,	
	font=\RaggedRight\regularfontdefaultbold	
}

% ------------------------------------------------------------------------
% Captions and Figures, Rotation, Image Counter
% ------------------------------------------------------------------------

\usepackage[figure]{totalcount}
\usepackage{rotating}
\usepackage{chngcntr}
\counterwithout{figure}{chapter}
\usepackage{caption}
\DeclareCaptionFont{smallfont}{\smallfontdefault}
\DeclareCaptionLabelFormat{plain}{\makebox[\spacebetweenlabelandtext][l]{\smallfontdefaultfig #2}}
\captionsetup{
	labelformat=plain,%
	format=hang,%
	labelsep=none,%
	margin=0pt,%
	textfont=smallfont,%
	singlelinecheck=off,%
	skip=1.5mm,%
	position=bottom,%
	justification=raggedright%
}
\usepackage{calc}% 

% ------------------------------------------------------------------------
% List of Figures
% ------------------------------------------------------------------------

\makeatletter
\renewcommand\listoffigures{%
        \@starttoc{lof}%
}
\makeatother
\DeclareTOCStyleEntry[indent=0mm, raggedpagenumber=true,linefill={}, pagenumberbox=\phantom, pagenumberformat=\nullfont, numwidth=\spacebetweenlabelandtext]{tocline}{figure}

% ------------------------------------------------------------------------
% Bibliography
% ------------------------------------------------------------------------



\usepackage[
	backend=biber,
	style=ext-verbose-trad1,
	maxnames=99,
	ibidtracker=false,
	idemtracker=false,
	loccittracker=false,
	opcittracker=false,		
	sorting=nty,
	dashed=false,
	innamebeforetitle=true,
	innameidem=true,
	dateabbrev=false,
	citepages=permit
]{biblatex}

% Nach Autor (in: Autor): Doppelpunkt, Space
\DeclareDelimFormat[bib,biblist]{nametitledelim}{\addcolon\space}
\DeclareDelimFormat[bib,biblist]{innametitledelim}{\addcolon\space}

% Vor Datum: Nur Space
\renewcommand*{\pubdatedelim}{\addspace}
\renewcommand*{\locdatedelim}{\addspace}

% Datum (1988 [1950]) - bei misc types nie daten drucken
\newbibmacro*{rawdate}{%
	\ifentrytype{misc}{}%
	{%
		\printtext[parens]{%
			\printdate
			\iffieldundef{origyear}{%
			}{%
				\setunit*{\addspace}%			
				\printtext[brackets]{\printorigdate}%
			}%
		}%
	}%
}

\renewbibmacro*{date}{%
	\iffieldundef{url}%
	{%
		\usebibmacro{rawdate}
	}%
	{}
}

%%%%%
% Issue-Datum (Periodias und Journals) ohne Klammer (schon oben gesetzt)
%%%%%

\DeclareFieldFormat{issuedate}{#1} % Entfernt klammer, schon im Makro date gesetzt
% Nur Space davor
\renewcommand*{\volnumdatedelim}{\addspace}
% Slash zwischen Volumen/Nummer
\renewcommand*{\volnumdelim}{/}

%%%%%
% Herausgeber: (hg.)
%%%%%

\DefineBibliographyStrings{german}{%
	editor = {\mkbibparens{Hg\adddot}}, %Hg. statt Hrsg.
	editors = {\mkbibparens{Hg\adddot}}, %Hg. für plural
	byeditor = {hg\adddotspace von}
}
% Herausgeber: (Hg.) - Klammer ohne Komma vorher
\DeclareDelimFormat[bib,biblist]{editortypedelim}{\addspace}

%%%%%
% Generell: Komma statt Punkt für newpunct
%%%%%
\renewcommand*{\newunitpunct}{\addcomma\space}
\renewcommand*{\titleaddonpunct}{\addspace}

%%%%%
% Komma Space vor in: (, in:)
%%%%%

\renewbibmacro*{in:}{%
\setunit{\addcomma\space}%
\printtext{%
	\bibstring{in}\intitlepunct}}

%%%%%
% Vorname Name Patch
%%%%%

% \usepackage{xpatch}
% \xpatchbibmacro{author}{\printnames}{\printnames[first-last]}{}{}  
% \xpatchbibmacro{editor}{\printnames}{\printnames[first-last]}{}{}
% \xpatchbibmacro{bookauthor}{\printnames}{\printnames[first-last]}{}{}

%%%%%
% URL-Feld ohne "url:", dafür einem (Date) am Ende
%%%%%

\newcommand*{\ifnodate}[1]{%
  \iffieldundef{#1year}
    {\iffieldundef{#1endyear}}
    {\@secondoftwo}}

\DeclareFieldFormat{url}{\url{#1}}
\renewbibmacro*{url+urldate}{%
	\iffieldundef{url}{}%
	{%
		\iffieldundef{urlyear}
			{}%
			{%
				\setunit*{\addspace}%
				\usebibmacro{urldate}%
			}
		\usebibmacro{url}%
		\ifnodate{}
			{}%
			{%
    		\setunit*{\addspace}%
		    \usebibmacro{rawdate}% Adding Date because we did not print it before in url-cases
			}%   
	}%
}

%%%%%%%
% movie / misc: unterdrückung von pubinstorg (movie, misc) und location (misc)
%%%%%%%


% Clone Actual Bibmacro

\renewbibmacro*{pubinstorg+location+date}[1]{%
  \printlist{location}%
  \iflistundef{#1}
    {\setunit*{\locdatedelim}}
    {\setunit*{\locpubdelim}}%
  \printlist{#1}%
  \setunit*{\pubdatedelim}%
  \usebibmacro{date}%
  \iffieldundef{series}
	{}%
	{%
		\printtext[parens]{%
		  =
		  \printfield{series}%
		  \setunit*{\sernumdelim}%
  		  \printfield{number}%
		}%
	}
  \newunit}


%%%%%%%%
% Kurzbelege
%%%%%%%%

\renewbibmacro*{cite:name}{%
 \printnames{labelname}%
 \setunit{\addcolon\space}}

\newbibmacro*{cite}{%
  \ifentrytype{misc}{\usebibmacro{cite:full}}%
  {%
	\ifentrytype{movie}{\usebibmacro{cite:full}}%
	{%
		\ifentrytype{artwork}{\usebibmacro{cite:full}}%
		{%
			\ifciteseen
				{%
					\iffieldundef{shorthand}
						{%
							\ifnameundef{author}{%
								\usebibmacro{cite:title}%
							}%
							{%
								\usebibmacro{cite:name}%
								\usebibmacro{cite:title}%
							}%
							\setunit*{\addspace}%
							\usebibmacro{rawdate}%
						}%
						{%
						\usebibmacro{cite:shorthand}%
						}
				}
				{\usebibmacro{cite:full}}			
		}
	}
  }
}

\newbibmacro*{cite:year}{%
  \printfield{year}%
  \setunit*{\addcomma\space}}

%%%%%%%%%%%%%%%%%%%%
% Series + number
%%%%%%%%%%%%%%%%%%%%
% clean macro, added before to pubinstorg-location-date
\renewbibmacro*{series+number}{}

%%%%%%%%%%%%%%%%%%%%%%
% Artwork
%%%%%%%%%%%%%%%%%%%%%%

\newbibmacro*{artwork:title-date}{%
  \ifboolexpr{
    test {\iffieldundef{title}}
    and
    test {\iffieldundef{subtitle}}
  }
    {}
    {\printtext[title]{%
      \printfield[titlecase:title]{title}%
	    \iffieldundef{subtitle}{%
      }{%
		    \setunit{\subtitlepunct}%
     	 \printfield[titlecase:title]{subtitle}%
		  }%
		  \setunit*{\addspace}%	
		 }%
     \usebibmacro{date}%
    }%
	\setunit{\titleaddonpunct}%
	\printfield{titleaddon}%
  }

\newbibmacro*{artwork:publisher+location}{%
  \printlist{location}%
  \iflistundef{publisher}
    {\setunit*{\locdatedelim}}
    {\setunit*{\locpubdelim}}%
  \printlist{publisher}%
  \newunit}

\DeclareBibliographyDriver{artwork}{%
  \usebibmacro{introcite:plain}%
  \usebibmacro{bibindex}%
  \usebibmacro{begentry}%
  \usebibmacro{author/editor+others/translator+others}%
  \setunit{\addcomma\space}\newblock
  \usebibmacro{artwork:title-date}%
  \newunit
  \usebibmacro{language}%
  \newunit\newblock
  \usebibmacro{byauthor}%
  \newunit\newblock
  \usebibmacro{byeditor+others}%
  \newunit\newblock
  \printfield{howpublished}%
  \newunit\newblock
  \printfield{type}%
  \newunit
  \usebibmacro{version}%
  \newunit
  \usebibmacro{note}%
  \newunit\newblock
  \usebibmacro{artwork:publisher+location}%
  \newunit\newblock
  \usebibmacro{doi+eprint+url}%
  \newunit\newblock
  \usebibmacro{addendum+pubstate}%
  \setunit{\bibpagerefpunct}\newblock
  \usebibmacro{pageref}%
  \newunit\newblock
  \iftoggle{bbx:related}
    {\usebibmacro{related:init}%
     \usebibmacro{related}}
    {}%
  \usebibmacro{finentry}}

%%%%%%%%%%%%%%%%%%%%%%%%%%%%%
% movie
%%%%%%%%%%%%%%%%%%%%%%%%%%%%%

\DeclareBibliographyDriver{movie}{%
  \usebibmacro{introcite:plain}%
  \usebibmacro{bibindex}%
  \usebibmacro{begentry}%
  \usebibmacro{author/editor+others/translator+others}%
  \setunit{\addcomma\space}\newblock
  \usebibmacro{title}%
  \newunit
  \usebibmacro{language}%
  \newunit\newblock
  \usebibmacro{byauthor}%
  \newunit\newblock
  \usebibmacro{byeditor+others}%
  \newunit\newblock
  \printfield{howpublished}%
  \newunit\newblock
  \printfield{type}%
  \newunit
  \usebibmacro{version}%
  \newunit
  \usebibmacro{note}%
  \newunit\newblock
  \usebibmacro{organization+location+date}%
  \newunit\newblock
  \usebibmacro{doi+eprint+url}%
  \newunit\newblock
  \usebibmacro{addendum+pubstate}%
  \setunit{\bibpagerefpunct}\newblock
  \usebibmacro{pageref}%
  \newunit\newblock
  \iftoggle{bbx:related}
    {\usebibmacro{related:init}%
     \usebibmacro{related}}
    {}%
  \usebibmacro{finentry}}

%%%%%%%%%%%%%%%%%%%%%%%%%%%%%%%%%%%
% Title (o.v)
%%%%%%%%%%%%%%%%%%%%%%%%%%%%%%%%%%%

\DeclareFieldFormat{titleaddon}{\printtext[parens]{#1}}%

\renewbibmacro*{title}{%
  \ifboolexpr{
    test {\iffieldundef{title}}
    and
    test {\iffieldundef{subtitle}}
  }
    {}
    {\printtext[title]{%
       \printfield[titlecase:title]{title}%
       \setunit{\subtitlepunct}%
       \printfield[titlecase:title]{subtitle}}%
     \setunit{\titleaddonpunct}}%
  \iffieldundef{titleaddon}{%
  }{
	\printfield[titleaddon]{titleaddon}%
  	\newunit}}

\IfFileExists{texput.bib}{
	\addbibresource{texput.bib}	
}{}

\setlength{\bibhang}{\genericindent}%
\setlength{\bibparsep}{0pt}%	
\setlength{\bibitemsep}{0pt}%
\renewcommand*{\bibfont}{\smallfontdefault}%
\defbibenvironment{bibliography}
	{\list
	{}
	{\setlength{\leftmargin}{0mm}%
	\setlength{\itemindent}{\bibhang}%
	\setlength{\itemsep}{\bibitemsep}%
	\setlength{\parsep}{\bibparsep}}}
	{\endlist}
	{\item}


% ------------------------------------------------------------------------
% Robust Uppercase Environment
% ------------------------------------------------------------------------


\NewEnviron{upc}{%
  \MakeUppercase{\BODY}%
}

% ------------------------------------------------------------------------
% Clever Ref
% ------------------------------------------------------------------------

% \usepackage{cleveref}
% \crefformat{figure}{\textsuperscript{\footnotemarkfontfig{□#2#1#3}}}
% \AtBeginDocument{\renewcommand{\ref}[1]{\cref{#1}}}

% ------------------------------------------------------------------------
% END HEADER
% ------------------------------------------------------------------------


% ------------------------------------------------------------------------
% Bastard Title Page
% ------------------------------------------------------------------------


\begin{document}
\bastardpage\smallfontbold\noindent intercom-mono.com/01
\cleardoublepage{}
\justifying

% ------------------------------------------------------------------------
% Title Page
% ------------------------------------------------------------------------

{
	\tallpage
	\regularfont
	\maketitle
	\cleardoublepage{}
}

% ------------------------------------------------------------------------
% Preface Page
% ------------------------------------------------------------------------




% ------------------------------------------------------------------------
% Toc Page
% ------------------------------------------------------------------------

\tocpage
\regularfont
\tableofcontents{}
\cleardoublepage{}

\normalpage
\bothhead{}

%
	%
		%
			%
			%
			%
				\selectlanguage{ngerman}%
			%
			%
				%
					
						\chapter[head={Einleitung}, tocentry={EINLEITUNG}]{Einleitung}%
					
					\-
					\par			
					\newpage%
					\noindent%
				%
			%
			%
				%
					\regularfontdefault%
				%
			%
		%
	%
	%
	
\subsection[Die Gesichtszüge    des Ingenieurs]{Die Gesichtszüge \- \protect\\ des Ingenieurs}
\par\noindent  \placeoriginal{https://writer.rokfor.ch/asset/4255/20486/HausmannEngineers1920.jpg}{/var/folders/fy/gnywf9l90v7gjh1n01ghh9zr0000gn/T/download_files2024126-65351-1xfuaxw.6inpf/b427a3eb-427b-60a8-1654-89145a04de20.jpg}{Raoul Hausmann, \emph{Die Ingenieure} (1920), 36 x 25 cm, Jerusalem: Israel Museum.}{0}{portrait}{Raoul Hausmann, \emph{Die Ingenieure} (1920), 36 x 25 cm, Jerusalem: Israel Museum. ©2022, ProLitteris, Zurich.}{4255-1}{false}{undefined}1920 beendet Raoul Hausmann, ein österreichischer Maler, Dichter und Fotograf, seine Arbeit an dem heute fast vergessenen Gemälde \emph{Die Ingenieure}. Das Bild zeigt drei elegant gekleidete Männer, die auf einem Platz stehen und Messungen vornehmen. Einer der drei hält ein Lineal in Händen, am rechten Bildrand ist ein Skizzenplan zu sehen. Bemerkenswert an Hausmanns Ingenieurfiguren, die heute im Israel Museum in Jerusalem zu sehen sind, ist, dass ihnen klare Gesichtszüge fehlen. Während wir die drei Männer vielleicht aufgrund ihrer Werkzeuge, sicher aber wegen des Titels als Ingenieure identifizieren können, bewahrt Hausmann in ihrer Gestaltung ein gehöriges Maß an Offenheit.\par Indem dieser Mitstreiter der dadaistischen Bewegung den Ingenieur als eine schemenhafte Figur zeichnet, exponiert er \textendash{} vermutlich ungewollt \textendash{} die Grundlage für jenes Phänomen, das hier als \emph{Imaginationsgeschichte des Ingenieurs} bezeichnet ist. Denn es ist die physiognomische Unschärfe des Ingenieurs, seine Unbestimmtheit, die das Fundament dafür bildet, dass sich eine Fülle derart gegensätzlicher Fantasien an ihn knüpfen konnte: In den Vorstellungswelten des 20. Jahrhunderts begegnet uns der Ingenieur als großer Mann wie auch als farbloser Funktionär, als sozialreformerischer Planer und faschistischer Held, als für den Sozialismus engagierter Arbeiter genauso wie als radikaler Islamist und Prometheus des digitalen Zeitalters. Man spricht von »Ingenieuren der Seele«, »Gesellschafts-«, ja »Gesundheits-Ingenieur{[}en{]}«\footnote{\cite[][]{westermann2003a}; \cite[][]{plank1938a}; \cite[][]{wesolowski2010a}, S. 56.}  und vergleicht sowohl die Verbrechen Alois Brunners, des »Ingenieurs der Endlösung«, als auch das Schreiben von Gedichten \textendash{} man denke an Paul Valérys »Dichteringenieur« \textendash{} mit den Tätigkeiten des Ingenieurs.\footnote{\cite[][]{ov2017a}; \cite[][]{bohnenkamp2002a}, hier S. 76.} \par Es ist die Abwesenheit klarer Gesichtszüge, die eine so vielfältige Gestaltbarkeit der Ingenieurfigur erst ermöglicht. Auf die andere Seite hin formuliert: Würde der Ingenieur eine von vornherein klar konturierte Physiognomie aufweisen, wäre er auf der Bühne der politischen Imagination des 20. Jahrhunderts nicht zu einer derart omnipräsenten, ideologisch hart umkämpften und schillernden Figur geworden.\par Und doch: Ungeachtet der verschiedenen Gesichter, mit denen uns der Ingenieur während der letzten mehr als 100 Jahre in Literatur, Philosophie, Wissenschaft, Politik und Unterhaltung entgegentritt, sind die allermeisten Darstellungen dieser Figur durch eine Gemeinsamkeit miteinander verbunden: Der Ingenieur wird über historische, mediale und ideologische Grenzen hinweg zumeist als ein \emph{Hoffnungsträger} imaginiert; als eine Figur, von der das Versprechen ausgeht, eine andere, bessere, oftmals ideale Welt nicht nur zu verheißen, sondern tatsächlich auch herstellen zu können. Und dieses Versprechen, so möchte ich behaupten, beruht auf der bis heute ungebrochenen Annahme, dass Ingenieure dazu in der Lage seien, ihr technisches Know-how, ihre Schaffenskraft und ihren Scharfsinn (lat. \emph{ingenium}) für die Gestaltung einer besseren Welt einzusetzen.
\subsection[Gegenwärtige Apotheosen    des Ingenieurs]{Gegenwärtige Apotheosen \- \protect\\ des Ingenieurs}
\par Vorstellungen vom Ingenieur als einem Weltverwandler sind nicht nur ein historisches Phänomen. Wir begegnen ihnen auch gegenwärtig, an Orten, die ungleichartiger nicht sein könnten. Als »Engineering Enlightenment« bezeichnet der Medien\-his\-to\-ri\-ker Brett T. Robinson das Versprechen des kalifornischen Ap\-ple-Kon\-zerns, nach dem technische Geräte dem Menschen mehr Autonomie ermöglichten.\footnote{\cite[][]{robinson2013a}, S. 84.}  In Umkehrung der (in zahlrei\-chen Dystopien entworfenen) Vorstellung von der Technik als einem Mittel der Unterwerfung ist sie hier als Instrument menschlicher Selbstbestimmung dargestellt. Bemerkenswert an dieser Verheißung ist, dass sie dem Ingenieur eine tragende Rolle zuspricht:\begin{quote}
\par The people who built Silicon Valley were engineers. They learned business, they learned a lot of different things, but they had a real belief that humans, if they worked hard with other creative smart people could solve most of humankind\textquoteright{}s problems. I believe that very much.\footnote{\cite[][]{jobs1996a}.} 
\end{quote}
\par Die Apotheose des Ingenieurs, wie sie aus diesen \texttt{Worten~her\-vorgeht} \textendash{} sie stammen von niemand Geringerem als dem Ap\-ple-Gründer Steve Jobs \textendash{}, wird nirgendwo so greifbar wie~im Kult, der in sozialen Medien um den sogenannten »10X Engineer« getrieben wird, ein Ingenieur, »so talented that he or she does the work of 10 merely competent engineers«:\footnote{\cite[][]{mcbride2013a}.}  »It is a given, that they can delve into a new area and quickly come up to speed. But technology does not define them, it is just the foundational layer for everything else they do.«\footnote{\cite[][]{ryan2016a}.}  Es handelt sich hierbei um eine überhöhende Vorstellung vom Ingenieur, eine \emph{Imago}, die ihm \textendash{} mit Blick auf Hausmanns Gemälde gesprochen \textendash{} eine bestimmte Physiognomie zuweist.\par Einer Verherrlichung des Ingenieurs, wie sie im Silicon Valley zu beobachten ist, begegnen wir heute noch an einem anderen, in mancher Hinsicht konträren Schauplatz: im radikalen \texttt{Islamismus.} Diego Gambetta und Steffen Hertog gelangen in \emph{Engineers of \texttt{Jihad}} zu dem Befund, dass Ingenieure in der Propaganda des extremistischen Islamismus eine Schlüsselrolle einnähmen. Yahya Ayyash beispielsweise, der den Beinamen »the engineer« trug und in den 1990er-Jahren als »Hamas master bomb-maker« galt, ist auf einschlägigen Websites zu einem Helden überhöht.\footnote{\cite[][]{gambetta2016a}, S. 12.}  Dessen Darstellungen verbinden Hinweise auf seine technische Expertise, seinen Mut, sein Märtyrertum und seine Gewöhnlichkeit. Gemäß der beiden Forscher bilde den Hintergrund einer solchen Idolatrie, dass Ingenieure in radikal islamistischen Bewegungen wie der Hamas oder Hisbollah die am stärksten vertretene Berufsgruppe darstellten. Es handle sich um Angehörige einer gut ausgebildeten technischen Elite, die sich aufgrund fehlender beruflicher Perspektiven radikalisierte. Die \emph{Imago} eines Yahya Ayyash, der seine technischen Fähigkeiten nicht nur zur Konstruktion einer neuen, sondern auch Destruktion der bestehenden Welt einsetzt, erhebt den gewöhnlichen Mann aus der Menge, den Ingenieur en masse, zu einem umstürzlerischen Helden. Und so gegensätzlich seine Gesichtszüge verglichen mit denen der kalifornischen Hightech-Ingenieure auch sein mögen, hier wie dort wird der Ingenieur als Wegbereiter in eine bessere Welt imaginiert; sei es, dass er die bestehende erschüttert, um einer neuen Platz zu machen, oder aber dass er eine Befreiung des Menschen mit technischen Mitteln verspricht.\par Zuletzt tauchen heldenhafte Ingenieurfiguren, wenn auch mit erst schattenhaft sichtbaren Gesichtszügen noch in einer an\-deren Sphäre, im Bereich der synthetischen Biologie, einer Subdisziplin der Gentechnologie, auf. Als »Ingenieure des Lebens«, so der Titel einer ihnen gewidmeten Buchpublikation, bezeichnet, suchen sie nach Wegen, um Genome, das heißt, die materiellen Träger der vererbbaren Informationen einer Zelle zu verändern. Diese neue Form des \emph{engineering}, als \emph{genome}, \emph{\texttt{genetic}} oder auch \emph{human engineering} bezeichnet,\footnote{\cite[][]{anders1980a}, S. 37.}  zielt so auf nichts Geringeres ab als auf die Herstellung, ja »Konstruktion von Leben«.\footnote{\cite[][]{satw2011a}, S. 10.}  Aus der Perspektive dieses Essays gesehen, setzt sich in den »Ingenieuren des Lebens« die Vorstellung fort, dass das schöpferisch-entwerferische Handeln von Ingenieuren das Tor in eine bessere Welt öffne. Und es ist nicht auszuschließen, dass einer dieser heute noch abseits des gesellschaftlichen Rampenlichts agierenden »Ingenieure des Lebens« demnächst aus seiner Unsichtbarkeit heraus- und in das Scheinwerferlicht der Gesellschaft treten wird.\footnote{Beispiele von \emph{genome engineering} in der populären Kultur gibt \cite[][]{rallapalli2020a}. Zwei frühe Berühmtheiten aus diesem Bereich sind Craig Venter und Francis Collins, die 2000 das erste Genom sequenzierten und auf dem Titel des \emph{Time Magazine} über dem Schriftzug »Cracking the Code!« abgebildet waren. Vgl.: \cite[][]{ov2000a}.} \par Vorstellungen vom Ingenieur, wie wir sie heute im Silicon Valley, im radikalislamischen Untergrund oder in Debatten über Gentechnologie beobachten können, sind keine Erfindung unserer Gegenwart. Sie haben vielmehr eine lange Vorgeschichte: Der Ingenieur als herausragendes Individuum, das auf Grundlage technischen Wissens in neue Sphären vorzustoßen vermag, als Weltverwandler, sogar Heilsbringer unter den Vorzeichen der westlichen Moderne \textendash{} hierin sind Echos auf die Apotheose des Ingenieurs zu hören, wie wir sie aus dem 20. Jahrhundert kennen. Der Technikhistoriker Thomas P. Hughes hat darauf aufmerksam gemacht, dass in der Rede über die Akteure des Silicon Valley ein Widerhall auf heroische Ingenieur-, Erfinder- und Unternehmerfiguren des frühen 20. Jahrhunderts zu vernehmen ist, und zuletzt behauptet der Literaturwissenschaftler Adrian Daub, dass hier die alte Vorstellung vom genialen Einzelnen aufgerufen werde.\footnote{Hughes schreibt: »During the past century, American inventors became heroic figures symbolizing the country\textquoteright{}s aptitude for innovation. Thomas Edison remains the foremost independent inventor-entrepreneur remembered for his electric light, phonograph, telegraph, and numerous other inventions. He was the proverbial small-town, plainspoken, self-made American whose untutored genius brought him fame and wealth. How unlike the highly complex Leonardo, who was both, artist and engineer. We should not be surprised that a century later Americans eager to recapture and refresh their image as a nation of inventors relish stories about a new wave of youthful inventor-entrepreneurs preparing business plans to finance Silicon Valley start-up companies that are intended to stimulate a computer revolution.« \cite[][]{hughes2004a}, S. 69. Für die entsprechenden Passagen aus Daubs Essay, vgl.: \cite[][]{daub2020a}, S. 70\textendash{}71.}  Zu beobachten ist heute also eine Rückkehr der Vorstellung vom Ingenieur als Ermöglicher einer besseren Welt \textendash{} wenn auch seine Gesichtszüge im Einzelnen andere geworden sind.
\subsection[Imaginationsgeschichte]{Imaginationsgeschichte}
\par Der vorliegende Essay hat es sich zum Ziel gesetzt, die Geschichte jener Vorstellungen zu rekonstruieren, die sich seit dem 20. Jahrhundert um den Ingenieur herum gebildet haben. Geleitet von dem literaturwissenschaftlichen Ansatz der Imaginationsgeschichte, gilt seine Aufmerksamkeit vorwiegend, aber nicht nur literarischen Quellen.\footnote{Von den rund 20 behandelten Quellen sind etwa die Hälfte literarisch.}  Diese werden als Zugriff auf kollektiv ge\-teilte Vorstellungen einer jeweiligen Epoche vom Ingenieur in den Blick genommen.\par Im Zentrum stehen damit also nicht die Analyse von Institutionen der Ausbildung und Standesvertretung von Ingenieuren, die Gegenstand einer Institu\-tionengeschichte des Ingenieurs sind;\footnote{Vgl.: \cite[][]{hortleder1970a}; \cite[][]{kaiser2006a}.}  ebenso wenig die Geschichte der von Ingenieuren erschaffenen Fabrikate (Brücken, Maschinen oder Verkehrswege), wie sie im Bereich der Technikgeschichte entfaltet wird;\footnote{\cite[][]{hughes1998a}.}  schließlich auch nicht die Ausbildung eines spezifischen Ingenieurwissens sowie seiner Umbrüche und Verästelungen, wie sie die Wissens\-geschichte entfaltet.\footnote{\cite[][]{picon2004a}; \cite[][]{porter1995a}.}  Während die drei zuletzt genannten An\-sätze \textendash{} idealtypisch gesprochen \textendash{} untersuchen, was tatsächlich stattfand, richtet die Imaginationsgeschichte des Ingenieurs ihre Aufmerksamkeit programmatisch auf die (oft realitätsfernen) Vorstellungen, die sich um den Ingenieur gebildet haben.\par Dass eine Rekonstruktion dieser Vorstellungen voraussetzt, über die institutionen-, technik- und wissensgeschichtliche Entwicklung des Ingenieurberufs gut Bescheid zu wissen, ist fraglos. Denn die zu untersuchenden Vorstellungen vom Ingenieur antworten mehr als einmal auf die ungleich nüchterne Wirklichkeit, will sagen, sind oftmals aus dieser heraus motiviert. Hinzu kommt, dass diese Vorstellungen in einigen Fällen auf die Wirklichkeit zurückwirken, also realitätsbildenden Charakter haben. Deshalb ist die tatsächliche Entwicklung des Ingenieurwesens der stets zu bedenkende Hintergrund, vor dem eine Imaginationsgeschichte des Ingenieurs zu entfalten ist, mehr noch, vor dem diese überhaupt erst verstanden werden kann.\par Um nun aber zu konkretisieren, wie genau sich der Zugriff der Imaginationsgeschichte auf die zu untersuchenden Vorstellungen gestaltet, sei ein längeres Zitat aus Michael Gampers Monografie \emph{Der große Mann. Geschichte eines politischen Phantasmas} angeführt:\begin{quote}
\par Sie {[}die Imaginationsgeschichte; RL{]} befasst sich weniger mit dem, was wirklich stattgefunden hat, als mit dem, was hätte stattgefunden haben können, also was in einer bestimmten Vergangenheit als denkbare Möglichkeit, als Alternative zur Gegenwart oder als Option für die Zukunft galt und in dieser Funktion wesentlich auf das einwirkte, was gemäß den zugänglichen Quellen tatsächlich stattfand und statthatte. Die sprachlich oder bildlich realisierten Figuren, Gegenstände und Handlungen der Imaginationsgeschichte gewinnen in der Regel als Produkte der Imagination Einzelner mediale Form, diese Einzelnen beziehen sich aber stets mehr oder minder stark auf kollektive Ideen und Befindlichkeiten. Gerade dichterische imaginationsgeschichtliche Elemente gewinnen ihre Bedeutung deshalb zum einen durch ihre prägnante Ausdruckskraft, {[}...{]} zum anderen aber durch allgemeinere Dispositionen, die in diese Gestaltungen eingegangen sind. {[}...{]} Imaginationsgeschichte ist deshalb immer irreduzibel an kollektive Zustände gebunden, wobei gerade die Kollektivierungsprozesse als dynamisch gedacht werden müssen.\footnote{\cite[][]{gamper2013a}, S. 99.} 
\end{quote}
\par Aus diesem Zitat wird zunächst, wenn auch in anderen begrifflichen Registern (»Alternative«, »Option«), das bereits skizzierte Programm einer Imaginationsgeschichte deutlich, die gegenüber dem Stattgefundenen das Vorgestellte privilegiert. Darüber hinaus ermöglicht es, den Zugriff der Imaginationsgeschichte auf die zu untersuchenden Vorstellungen genauer zu fassen: Die einzelne Quelle, heißt es bei Gamper sinngemäß, sei prägnanter Ausdruck eines kollektiven Imaginären, das heißt, eines Arsenals von zu einem bestimmten Zeitpunkt gesellschaftlich geteilten Vorstellungen. Die theoretische Unterscheidung zwischen Quelle und kollektivem Imaginären lässt anschaulich werden, dass die einzelne Quelle also sowohl als punktueller Zugriff auf das kollektive Imaginäre einer jeweiligen Zeit zu verstehen ist als auch als dessen je spezifische Formung. Anders, besser: Die im Rahmen dieses Essays zu untersuchenden Einzelquellen sind für das kollektive Imaginäre (die Vorstellungswelt einer bestimmten Epoche) zum einen repräsentativ, sie stellen zum anderen aber einen je spezifischen Ausdruck seiner Modellierung dar.\par Mit dieser theoretischen Rahmung verbunden ist schließlich eine spezifische Arbeitsweise: Anstatt deduktiv festzulegen, was ein Ingenieur ist, wird der vorliegende Essay induktiv untersuchen, welche \emph{Imago} vom Ingenieur eine jeweilige Quelle evoziert. Er wird, mit Blick auf Hausmann gesprochen, die verschiedenen Gesichter des Ingenieurs aus den Quellen erschließen, um so die oftmals disparaten Vorstellungen zu konturieren, welche die Ingenieurfigur zu verbinden vermag.
\subsection[Historischer Fokus und    das Konzept der Figur]{Historischer Fokus und \- \protect\\ das Konzept der Figur}
\par Seinen historischen Fokus hat der Essay auf der Periode ab dem frühen 20. Jahrhundert. Denn erst in dieser Zeit, als Ingenieurausbildungen etabliert,\footnote{1901 erfolgte in Deutschland die Gleichstellung der Technischen Hochschulen mit den Universitäten. Vgl.: \cite[][]{treue1967a}, hier S. 465.}  technische Museen gegründet (München 1903, Wien 1909) und Interessenvertretungen für Ingenieure geschaffen wurden,\footnote{Der 1856 gegründete Verein Deutscher Ingenieure begann mit 172 Mitgliedern und umfasste schon 1870 2500. Vgl.: \cite[][]{treue1967a}, \texttt{S.~461.}}  traten Ingenieure auf die Bühne der politischen Imagination. Sie haben diese seither nicht mehr verlassen, sehr wohl aber verschieden intensiv bevölkert. Als intensivste Phasen des Ingenieurhelden können neben unserer Gegenwart die Jahre nach dem Ersten Weltkrieg sowie die Anfänge des Kalten Krieges gelten. Es sind diese drei Perioden, aus denen die zentralen Quellen dieses Essays stammen und denen seine besondere Aufmerksamkeit gilt.\par Dabei gehört es zu seinen Zielen, Einsichten, die aus der Analyse der frühen Zwischenkriegszeit und des Kalten Krieges gewonnen werden, dazu zu nutzen, um den Blick auf die Ingenieurhelden unserer Gegenwart zu schärfen; das heißt, ein gegenwärtig wiederkehrendes Phänomen durch die Rekonstruktion seiner Vorgeschichte genauer beobachten zu lernen: Werden Vorstellungen vom Ingenieur als Ermöglicher einer besseren Welt besonders in Zeiten sozialer Desorientierung hörbar? Man denke an die Jahre nach dem Großen Krieg, als bei der Suche nach einer neuen Ordnung die Forderung nach Gesellschaftstechnikern immer lauter wurde. Ähnliches gilt für die Jahrzehnte nach 1945, als Ingenieure im Zuge des Wiederaufbaus sowie in einem Klima der nuklearen Aufrüstung zu gefragten Experten aufstiegen. Welche Rückschlüsse lassen sich aus diesen für die Imaginationsgeschichte des Ingenieurs heißen Phasen mit Blick auf seine Wiederkehr im digitalen Zeitalter gewinnen? Worauf antwortet sein Auftauchen als handlungsorientierende Größe heute? Und welche Rolle spielen in allen drei Perioden die jeweils neuen Medien: der Film in den 1920er-, das Fernsehen in den 1950er-Jahren, das Internet heute?\par Der historische Anspruch, den dieser Essay stellt, meint also nicht eine teleologische Entwicklung in der Idolatrie des Ingenieurs zu suggerieren. Vielmehr sollen die Ungleichzeitigkeiten in seiner Geschichte sichtbar werden. Um das zu erreichen, ist er um das Prinzip der Figur aufgebaut, das es ermöglicht, das Verschwinden und Wiederauftauchen, das Sich-Wiederholen, Variieren und Abhandenkommen bestimmter Vorstellungen vom Ingenieur in den Blick zu rücken.\par Im Unterschied zum Modell der literarischen Figur, \texttt{das~den} relativ genau konturierten Protagonisten zumeist \emph{\texttt{eines}}\texttt{~lite\-ra\-rischen} Textes in den Blick nimmt, zielt das hier entfaltete Fi\-gurenkonzept darauf ab, den Ingenieur als ein gesamtgesell\-schaftliches Phänomen zu fassen. Diese Erweiterung des Fi\-gurenkonzepts, die begrifflich in der Rede von einer Figur des kollektiven Imaginären zum Ausdruck kommt, trägt dem Umstand Rechnung, dass sich Vorstellungen vom Ingenieur nicht auf die Dichtung im engeren Sinne beschränken,\footnote{Vgl. zur literarischen Ingenieurfigur: \cite[][]{schwiglewski1995a}; \cite[][]{brandt2007a}.}  sondern in unterschiedliche Medien (Text, bildende Kunst, Film), Diskurse (Wissenschaft, Werbung, Politik) sowie über die sozial motivierten Grenzen von Hoch- und Populärkultur hinweg verästeln.\footnote{Die hier vorgenommene Erweiterung der literaturwissenschaftlichen Kategorie der Figur erfolgt in der Überzeugung, dass an der Literatur ausgebildete Instrumentarien der Analyse sinnvoll dazu verwendet werden können, um Phänomene auch außerhalb der Literatur genauer zu beschreiben. Diese Überlegungen schließen an \texttt{Albrecht~Koschorke} an, vgl.: \cite[][]{koschorke2007a}.}  Den Ingenieur als eine Figur des kollektiven Imaginären zu fassen ermöglicht also, um noch einmal auf Hausmanns Gemälde zurückzugreifen, die Vielfalt seiner Gesichtszüge über die Grenzen einzelner Subsysteme der Gesellschaft hinweg zu rekonstruieren. Und dennoch wird es dabei nicht alleine um das Auffächern dieser Vielfalt gehen, sondern \textendash{} das signalisiert der im Untertitel aufscheinende Begriff der Grammatik \textendash{} gerade auch um jene Regularitäten, die sich über die verschiedenen Gesichter dieser \emph{Imago} hinweg beobachten lassen.%
	%
%
	%
		%
			%
			%
			%
				\selectlanguage{ngerman}%
			%
			%
				%
					
						\chapter[head={Große und kleine Ingenieure}, tocentry={GROẞE UND KLEINE INGENIEURE}]{Große und kleine Ingenieure}%
					
					\-
					\par			
					\newpage%
					\noindent%
				%
			%
			%
				%
					\regularfontdefault%
				%
			%
		%
	%
	%
	
\subsection[Der Ingenieur als    großer Mann und gesellschaftlicher    Hoffnungs\-träger]{Der Ingenieur als \- \protect\\ großer Mann und gesellschaftlicher\- \protect\\  Hoffnungs\-träger}
\par\noindent 1935 \textendash{} am Ende der Zwischenkriegszeit \textendash{} erscheint in den amerikanischen Kinos der Film \emph{Trans-Atlantic Tunnel}, der auf dem deutschsprachigen Roman \emph{Der Tunnel} basiert. Sein Verfasser, Bernhard Kellermann, als Autor des S. Fischer Verlags in einer Reihe stehend mit Thomas Mann und Alfred Döblin, landet mit diesem 1913 publizierten Text den Erfolg seines Lebens. \emph{Der Tunnel} wurde viermal verfilmt, in über 20 Sprachen übersetzt und erzielte bis 1943 373 Auflagen.\footnote{Vgl.: \cite[][]{segeberg1987a}, S. 173\textendash{}208.} \par Sicher war die Wahl der Hauptfigur für diesen außergewöhnlichen Erfolg mit ausschlaggebend. Denn als Kellermann die Figur des Richard Mac Allan zeichnet, jenes Ingenieurhelden, der im Mittelpunkt des Geschehens steht, sind Ingenieure in aller Munde, und ihr gesellschaftliches Ansehen hat dies- und jenseits des Atlantiks einen neuen Höhepunkt erreicht: Nach einem lange währenden Kampf war nämlich die ständische Konsolidierung des Ingenieurs an der Wende zum 20. Jahrhundert auch im deutschsprachigen Raum abgeschlossen;\footnote{\cite[][]{treue1967a}, S. 466. Diese Entwicklung des Berufsstandes verlief national durchaus verschieden: In Paris wurde 1747 die École des Ponts et Chaussées und 1794 die École Polytechnique ins Leben gerufen, in Berlin 1799 die Preußische Bauakademie. Während in Wien schon 1846 ein österreichischer Architekten- und Ingenieurverein gegründet worden war, datiert der VDI (Verein Deutscher Ingenieure) auf das Jahr 1856. Vgl. hierzu: \cite[][]{koenig2007a}; \cite[][]{treue1967a}, S. 459; sowie \cite[][]{popplow2007a}, hier S. 970.}  immer häufiger traten Ingenieure in der Öffentlichkeit in Erscheinung, zumal Zeitung und Rundfunk bei der populärwissenschaftlichen Vermittlung technischer Inhalte seit den ersten Jahrzehnten des 20. Jahrhunderts auf deren Expertise setzten,\footnote{\cite[][]{popplow2007a}, S. 962; \cite[][]{schirrmacher2013a}.}  und mit der Technokratiebewegung, einer von 1932 bis 1936/37 währenden politischen Initiative, setzte sich zuerst in den USA, dann in Deutschland die Idee durch, dass gesellschaftliche Probleme auf der Grundlage des rationalisierten Denkens der Technikwissenschaften zu lösen seien.\footnote{\cite[][]{willeke1995a}. »Germany has her own technocratic movement«, heißt es in einem Artikel vom 23. Januar 1933 in der \emph{New York Times}. Vgl.: \cite[][]{ov1933a}.}  Die Entscheidung also, eine heldenhafte Ingenieurfigur in den Mittelpunkt des fiktionalen Geschehens zu stellen, bedeutete, kollektiv geteilte Vorstellungen der Epoche aufzunehmen, mehr noch, kursierende Bilder vom Ingenieur im Rahmen künstlerischer Fiktionen, mit literarischen und filmischen Mitteln weiter zuzuspitzen.\par In Elveys filmischer Adaption von Kellermanns Ingenieurroman, die so wie der Text auch um den Bau eines Tunnels zwischen Europa und den USA kreist, ist es der Amerikaner Richard Dix, auf dem \emph{Hollywood Walk of Fame} mit einem Stern gewürdigt, der Kellermanns Mac Allan verkörpert. Dessen Erfindung eines Diamantstrahlbohrers ist es zu verdanken, dass der Bau einer unterirdischen Verbindung zwischen den beiden Kontinenten (die Roman\-idee reagiert auf den Untergang der Titanic im Jahr 1912) überhaupt erst möglich wurde.\par Als Mac Allan seinen Geldgebern zu Beginn des Films das Vorhaben eines transatlantischen Tunnels erläutert, betont er, dass er dazu in der Lage sei, das vermeintlich Unwirkliche Wirklichkeit werden zu lassen: »I am only an engineer, putting up an engineering proposition. You may find it a bit fantastic, but I know it can be done.«\footnote{\cite[][]{elevy1935a}, 06:50.}  \emph{I-t c-a-n b-e d-o-n-e} \textendash{} Mit diesen vier Worten ist das Versprechen artikuliert, eine irreal scheinende Idee zu realisieren. Es handelt sich um eine Vorstellung, die, wie wir noch sehen werden, einen wiederkehrenden Baustein in der Gestaltung von Ingenieurfiguren bildet. Und im Rahmen der literarischen wie filmischen Fiktion behält Mac Allan auch recht: Es gelingt ihm, nachdem sich im Stollen zwei Katastrophen ereignet, die Geld\-geber ihren Rückzug angedroht und zahlreiche Menschen ihr Leben verloren haben, schließlich der Durchstich von England nach New York. Die zahlreichen »engineering failures«, die Mac Allans Weg zu diesem Ziel pflastern, haben dramaturgisch gesehen keine andere Funktion als die Heldenhaftigkeit dieses Ingenieurs zu verstärken.\footnote{Der Begriff »engineering failure« ist einer Studie Loren R. Grahams entnommen, die sich mitunter dem Scheitern von Ingenieur-Großprojekten in der Sowjetunion und deren verheerenden Auswirkungen widmet, allen voran der Atomkatastrophe von Tschernobyl im Jahr 1986. \cite[][]{graham1993a}.}  Allen Widerständen und Rückschlägen zum Trotz, ließe sich sagen, setzt sich die Ingeniosität dieses Mannes schließlich durch.\par Um nun \textendash{} im Rückgriff auf Hausmanns Gemälde \textendash{} die dieser Ingenieurfigur eigene Physiognomie deutlich werden zu lassen, ist es wichtig, sich klarzumachen, dass Mac Allans Erfindungsgabe nicht allein der Herstellung eines technischen Artefakts, eines Bohrers oder transatlantischen Tunnels dient, sondern in eine gesellschaftliche Perspektive gerückt ist. Seine Schaffenskraft, so verheißt Mac Allan am Ende des Films selbst, werde nicht weniger als eine bessere Welt hervorbringen: »I believe that my work will bring peace to the world.«\footnote{\cite[][]{elevy1935a}, 01:23:39.}  Und in ebendiesem Sinne spricht ein englischer Politiker von Mac Allans Tunnel als »not merely a remarkable piece of engineering, but something far greater {[}...{]} a common sharing of each other\textquoteright{}s progress, a closer understanding of each other\textquoteright{}s problems«.\footnote{\cite[][]{elevy1935a}, 42:40\textendash{}43:49.}  Wir sehen hier jene Vorstellung am Werk, die den Ingenieur erst zu einem Hoffnungsträger der Gesellschaft werden lässt. Sie lautet ausbuchstabiert: Seine Ingeniosität bringt nicht nur technischen, sondern auch gesellschaftlichen Fortschritt.\par Die daraus sich ergebende exponierte Position des Ingenieurs wird in den visuellen Arrangements von \emph{Trans-Atlantic Tunnel}, genauer: an einer Reihe von Einstellungen deutlich, in denen sich Mac Allan einer Menschenmenge gegenübersieht. An Georg Wilhelm Friedrich Hegels entwickeltes Konzept des großen Mannes erinnernd, der sich nach Hegel durch individuelle Größe von den Vielen abhebe und ihnen voranzugehen vermag, zeigt der Film den Ingenieur als einen Anführer in eine bessere Welt.\footnote{\cite[][]{elevy1935a}, 56:50\textendash{}57:20. Dieselbe Konstellation findet sich ebd. 01:20:36. In einer weiteren, ebenso wiederkehrenden szenischen Anordnung spricht Mac Allan in eine Kamera. Hier wiederholt sich das Motiv der auf ihn gerichteten Blicke der Vielen, vgl.: ebd. 13:05; 23:40.}  Auffällig ist hier die Disproportionalität der Blickrichtungen: Während der Ingenieur in die Menge blickt, blickt die Menge auf ihn, das Zentrum der Aufmerksamkeit. Möchte man dieses Standbild als eine symbolische Momentaufnahme der damaligen Gesellschaft lesen, zeigt sie den Ingenieur als einen Träger kollektiv geteilter Hoffnungen.
\subsection[Kurzer Blick zurück:    Ein Vorfahre des Ingenieurhelden    aus dem 19.~Jahrhundert]{Kurzer Blick zurück: \- \protect\\ Ein Vorfahre des Ingenieurhelden \- \protect\\ aus dem 19.~Jahrhundert}
\par Um das Gesicht dieses Ingenieurs, der, um es zu wiederholen, also nicht nur Bastler oder Techniker, sondern ein gesellschaftlicher Hoffnungsträger ist, nun weiter zu profilieren, sei an eine Vorform des Ingenieurhelden erinnert, in welcher diese spezifische Dimension der Figur noch nicht ausgebildet ist. Die Rede ist von Impey Barbicane aus Jules Vernes Roman \emph{Von der Erde zum Mond} (1868):\begin{quote}
\par {[}...{]} ein Mann von vierzig Jahren, ruhig, kühl, streng, ein ausgesprochen ernsthafter und konzentrierter Geist, der exakt wie ein Chronometer funktionierte, von abgebrühtem Naturell und unerschütterlichem Charakter war; wenig umgänglich, aber abenteuerlustig, steuerte er Ideen praktischer Art zu den verwegensten Unternehmungen bei.\footnote{\cite[][]{verne2013a}, S. 17.} 
\end{quote}
\hrulefill
\par Etwa 50 Jahre vor Kellermann stoßen wir hier auf einen Männertypus, an dem vieles an den Ingenieur Mac Allan erinnert. Wie er ist Barbicane exakt, abgebrüht, unerschütterlich, wenig umgänglich und abenteuerlustig. Seine Energie, Kühnheit und Kaltblütigkeit, so der Erzähler, wären sogar in seinem Gesicht erkennbar:\begin{quote}
\par Seine markanten Gesichtszüge schienen mit Lineal und Winkelmaß gezogen und wenn es stimmt, dass man, um den Charakter eines Menschen zu erkennen, dessen Profil anschauen soll, dann zeigte Barbicane so betrachtet die eindeutigsten Zeichen von Energie, Kühnheit und Kaltblütigkeit.\footnote{\cite[][]{verne2013a}, S. 18.} 
\end{quote}
\par Barbicane, den Verne als einen »Erfinder«\footnote{\cite[][]{verne2013a}, S. 18.}  bezeichnet, hat all diese Eigenschaften nötig. Denn er tritt dazu an, gemeinsam mit dem Amerikaner Captain Nicholl und dem Franzosen Michel Ardan in einem Projektil zum Mond zu fliegen. Ähnlich wie in Kellermanns \emph{Tunnel} verfolgen wir Leser*innen einen atemberaubenden Wettkampf zwischen Menschen und Natur, den schließlich dank seiner Erfindungsgabe der Mensch zu gewinnen ansetzt. Worauf es ankommt: Technische Innovationen haben in diesem Roman aus der zweiten Hälfte des 19. Jahrhunderts, salopp gesagt, nur sportlichen Charakter. Sie dienen anders als bei Kellermann noch nicht dem Fortschritt oder Weltfrieden, sondern werden um ihrer selbst willen hervorgebracht. Und wenn auch Verne Szenen evoziert, in denen die Menschenmenge diesen erfindungsreichen Mann bewundert,\footnote{\cite[][]{verne2013a}, S. 229\textendash{}238.}  gleicht diese Bewunderung eher der gegenüber einem Athleten als einem gesellschaftlichen Hoffnungsträger. Anders gesagt, steht bei den Erfindungen Mac Allans viel mehr auf dem Spiel als bei jenen seines Vorfahren, Impey Barbican.\hrulefill

\subsection[Gesellschaftsingenieure:    Faschistische und sozial\-reformerische    Anführer]{Gesellschaftsingenieure: \- \protect\\ Faschistische und sozial\-reformerische \- \protect\\ Anführer}
\par Zurück zum kollektiven Imaginären der Zwischenkriegszeit, in dem Ingenieure nicht allein für technischen, sondern auch gesellschaftlichen Fortschritt stehen, muss der Blick bei der Suche nach heroischen Ingenieurfiguren auf eine der profiliertesten avantgardistischen Bewegungen dieser Jahre fallen: den Futurismus.\par Lesbar als eine Steigerung von Kellermanns Ingenieur, steht die Hauptfigur aus Filippo Tommaso Marinettis futuristischem Roman \emph{Mafarka der Futurist} (1909) auf der anderen Seite des Figurenspektrums als Vernes Erfinder. Es handelt sich hierbei um einen Konstrukteur, der noch viel deutlicher als Kellermanns Mac Allan als ein gesellschaftlicher Erneuerer in Szene gesetzt ist und damit dem kollektiven Imaginären dieser Jahre in prägnanter Weise Ausdruck verleiht.\begin{quote}
\par Im Namen des menschlichen Stolzes, den wir anbeten, verkündige ich {[}F. T. Marinetti; RL{]} euch, daß die Stunde nahe ist, in der Männer mit hohen Schläfen und stählernem Kinn auf wunderbare Weise, mit einer einzigen Anstrengung ihres aus den Augenhöhlen getretenen Willens, unfehlbare Riesen schaffen werden ...\footnote{\cite[][]{marinetti2004a}, S. 10.} 
\end{quote}
\par Das schreibt Marinetti im Vorwort seines Romans, das seinem Publikum die Ankunft einer Schöpferfigur verspricht. Im Roman selbst, der zur Vorgeschichte der futuristischen Bewegung gehört\footnote{\cite[][]{schmidt-bergmann2004a}, hier S. 262.}  und im selben Jahr erschienen ist wie das sie proklamierende Manifest \emph{Le Futurisme}, trägt diese Schöpferfigur den Namen Mafarka, der »Konstrukteur mechanischer Vögel«.\footnote{\cite[][]{marinetti2004a}, S. 123.}  In seinen eigenen Wor\-ten: »Ich konstruiere {[}...{]} meinen Sohn {[}Gazourmah; RL{]}, einen unbesiegbaren riesigen Vogel, der große biegsame Flügel hat, geschaffen, um die Sterne zu umarmen!«\footnote{\cite[][]{marinetti2004a}, S. 123.}  Und dann: »Unser Wille muß aus uns heraustreten, um sich der Materie zu bemächtigen und sie nach unserem Belieben zu verändern. Auf diese Weise können wir alles, was uns umgibt, formen und endlos das Antlitz der Welt erneuern.«\footnote{\cite[][]{marinetti2004a}, S. 125.}  Auffällig ist hier zunächst Marinettis Ausdrucksweise: Mafarkas Anspruch, \emph{alles, was uns umgibt,} zu verändern, wird mit einem Pathos des Neuen artikuliert, in einer expressiven, von den Konventionen der Alltagskommunikation entlasteten Sprache, die sich von der Unterhaltungsprosa eines Bernhard Kellermann unterscheidet. Hieran wird deutlich, dass sich die \emph{Imago} eines die Gesellschaft erneuernden Ingenieurs zu Beginn des 20. Jahrhunderts \textendash{} mit dem Soziologen Pierre Bourdieu gesprochen \textendash{} über die Grenzen von experimenteller und kommerzieller Literatur, von Avantgarde (Futurismus) auf der einen und populärer Belletristik (Kellermann) auf der anderen Seite, hinweg verästelt.\par Der nun von ihm angekündigte Gazourmah, der von Mafarka zu konstruierende Sohn, halb Mensch, halb Maschine, steht \emph{pars pro} \emph{toto} für den neuen Menschen:\begin{quote}
\par Das »Reich der Maschine«, das aus dem gewaltsamen Umsturz alles Bestehenden erwachsen soll, wird das großstädtische Leben und die umfassende Technifizierung aller Lebensbereiche zu einem Funktionsraum umgestalten, in dem der neue »Mensch«, ein moderner »Zentaur«, genährt von »Feuer, Haß, Schnelligkeit«, halb Mensch und halb Maschine, eine neue »dynamische Realität« finden sollte.\footnote{\cite[][]{marinetti2004a}, S. 272. Hörbar ist der Verweis auf die antike Gestalt des Kentauren, dessen Wesen, halb Mensch, halb Tier, hier zu einer Mischung aus Mensch und Maschine (ebd. S. 125 und S. 127) gewandelt ist.} 
\end{quote}
\par Entscheidend ist hier der aus dem Zitat hervorgehende Anspruch einer Umgestaltung der Gesellschaft, wie sie im kollektiven Imaginären dieser Jahre mit Ingenieuren als großen Individuen verbunden ist: Mafarkas Fabrikation eines neuen Menschen, Gazourmah, will die Hervorbringung einer »neue{[}n{]} dynamische{[}n{]} Realität«. Der Aufbruch, den Mafarka für die Technik, Sprache, Kunst und Gesellschaft reklamiert, meint auf der einen Seite Erneuerung, auf der anderen jedoch auch die Zerschlagung des kulturellen wie sozialen Status quo. In diesem destruktiven Moment lassen sich entfernte Echos auf die Ingenieure des Heiligen Krieges vernehmen, bei denen sich fast 100 Jahre später das Versprechen einer neuen Welt ähnlich mit der Zertrümmerung der bestehenden verbindet.\par Dass Marinettis literarische Fantasien Affinitäten zum italienischen Faschismus unterhalten, ist \emph{opinio communis}.\footnote{\cite[][]{schmidt-bergmann2004a}, S. 263.}  Diese liegen sowohl auf der Ebene der persönlichen Netzwerke (man denke an die gegenseitigen Würdigungen von Marinetti und Benito Mussolini) wie auch auf jener der ideologischen Standpunkte:\footnote{\cite[][]{schmidt-bergmann2004a}, S. 283. Über den Bruch zwischen Marinetti und Mussolini informiert: \cite[][]{hans2015a}, S. 287.}  eine rigorose Entwertung der Tradition; ein gegen das Gleichheitsprinzip gerichtetes Rassedenken; ein rabiates Männlichkeitsbild, in dem sich dieselbe urbane Jugend wiedersah, die zu den Protagonisten der sich zeitgleich ausbildenden faschistischen Bewegung wurde; sowie ein Kult der Maschine, wie er exemplarisch in Mussolinis Rhetorik \textendash{} »Lo Stato fascista è, più che uno \emph{stato}, una \emph{dinamo}«\footnote{Zitiert nach: \cite[][]{pellizzi1924a}, S. 165.} \textendash{} deutlich wird.\par Mit Blick auf die Bedeutung, die Vorstellungen von einem \texttt{Gesellschaftskonstrukteur~in} der faschistischen Bewegung zukommt, ist es zunächst erstaunlich und für das Phänomen des Ingenieurs als einer Figur des kollektiven Imaginären aufschlussreich, dass eine \emph{Imago} des Gesellschaftstechnikers, die in Teilen an jene des Futurismus erinnert, etwa zur selben Zeit auch am anderen Extrem des politisch-ideologischen Spektrums auftaucht: Im Nachbarland Österreich entwirft der Sozialreformer und Philosoph Otto Neurath 1919, wenige Jahre vor der Machtübernahme Mussolinis in Italien (1922), die Vorstellung eines »Gesellschaftstechnikers« oder Gesellschaftsingenieurs. In \emph{Die Utopie als gesellschaftstechnische Konstruktion}, seinem Plädoyer für eine bewusste Neugestaltung der Gesellschaft nach dem Ersten Weltkrieg, schreibt er:\begin{quote}
\par Der Gesellschaftstechniker, welcher sich auf seine Arbeit versteht und eine Konstruktion liefern will, die für praktische Zwecke als erste Anleitung verwendbar sein soll, muß die seelischen Eigenschaften des Menschen, seine Lust am Neuen, seinen Ehrgeiz, sein Hängen an der Überlieferung, seinen Eigensinn, seine Dummheit, kurz, alles, was ihm eignet und sein gesellschaftliches Handeln im Rahmen der Wirtschaft bestimmt, genau so berücksichtigen, wie etwa der Ingenieur die Elastizität des Eisens, die Bruchfestigkeit des Kupfers, die Farbe des Glases und ähnliches mehr.\footnote{\cite[][]{neurath1919a}, hier S. 227.} 
\end{quote}
\par Neurath vergleicht hier den Prozess der Neugestaltung der Gesellschaft mit der Tätigkeit eines Ingenieurs: So wie der Ingenieur über die Gegenstände seiner Arbeit, Eisen, Kupfer und Glas, Bescheid wissen müsse, so müsse auch jener Ingenieur, der die Gesellschaft nun wiederaufbaue, über die seinen, die seelischen Eigenschaften des Menschen, seine Lust am Neuen etc., Bescheid wissen. Es ist, kurz gesagt, die Arbeit des Ingenieurs, die, gemäß Neuraths Argumentation, für jene an der Gesellschaft als Leitlinie dient.\par Wenn Marinettis aggressiver, frauenverachtender He\-ro\-is\-mus und Neuraths sozialreformerische \emph{Imago} des Gesellschaftstechnikers ideologisch auch weit auseinanderliegen, fällt auf, dass beide in ihrem Plädoyer für eine Umgestaltung der Gesellschaft etwa zeitgleich auf das Bildfeld der Technik rekurrieren und eine bestimmte Vorstellung vom Ingenieur evozieren. Der amerikanische Historiker Charles Steven Maier hat diesen auf den ersten Blick so verblüffenden Sachverhalt in einer Untersuchung über die europäische Rezeption von Technokratie und Taylorismus erläutert und gezeigt, dass die in den USA entwickelte Vorstellung von Ingenieuren als \emph{scientific managers} im Europa der Zwischenkriegszeit besonders an den Rändern des politischen Spektrums Resonanz gefunden habe \textendash{} exemplarisch gesprochen, Mussolini (anfangs ein Marxist!) ebenso inspiriert habe wie Lenin.\footnote{\cite[][]{maier1970a}, hier S. 43 und S. 51.}  Beide politischen Lager, so Maier, hätten nämlich in der Lösung sozialer Spannungen auf Grundlage einer Neuorganisation von oben, genauer: durch einen die Gesellschaft leitenden Ingenieur, noch genauer: eine Totalisierung der Rationalität einen Ausweg gesehen; die sozialreformerische Linke aus dem Dilemma des Kapitalismus, in dem eine soziale Klasse die andere ausbeute; die faschistische Rechte hingegen aus der Bedrohung durch den Marxismus, nach dessen Prophezeiung eine gewaltsame Revolution von unten unausweichlich sei: »Postulating a new social category of producers«, schreibt Maier, »or more narrowly, an elite of scientific managers who arbitrated conflict, meant that the hostile confrontation between the traditional classes was superseded.«\footnote{\cite[][]{maier1970a}, S. 43.}  In diesem Sinne wurde die Vorstellung vom Ingenieur als Erlöser aus sozialen Spannungen für die Gegner des Kapitalismus wie auch jene des Marxismus attraktiv.\footnote{In dieser Hinsicht mit dem Taylorismus vergleichbar ist die weit verästelte Wirkungsgeschichte von Henry Ford in Europa. Vgl.: \cite[][]{link2020a}.}  In der Sprache dieses Essays gesagt, ist diese Vorstellung während der Zwischenkriegszeit ein elementarer Bestandteil ihres kollektiven Imaginären, das sich in den einzelnen Quellen in durchaus verschiedenen Physiognomien der Ingenieurfigur manifestiert, sei es als tollkühner Erfinder (bei Kellermann), faschistischer Macho (bei Marinetti) oder sozialreformerischer Erneuerer (bei Neurath), gleichzeitig aber über diese disparaten Physiognomien hinweg als ein Wasserzeichen der Epoche sichtbar wird \textendash{} als ein und derselbe Mythos in seinen je verschiedenen Schattierungen.
\subsection[Von kollektiver Größe:    Der Inge\-nieur als farbloser    Funktionär]{Von kollektiver Größe: \- \protect\\ Der Inge\-nieur als farbloser\- \protect\\  Funktionär}
\par Ungeachtet der verschiedenen diskursiven Verfahren, \texttt{mit~denen} Kellermanns Mac Allan, Marinettis Mafarka und noch \texttt{Neuraths} »Gesellschaftstechniker« evoziert werden, alle diese In\-gen\-ieur\-figu\-ren lassen sich dem Paradigma des großen Mannes zuordnen. Das den Ingenieur als eine Figur des kollektiven Imaginären kennzeichnende Versprechen, gesellschaftlichen Fortschritt mit den Mitteln der Technik zu erzielen, kommt hier wie dort aus dem Munde von hervorragenden Einzelnen, Anführerfiguren, charismatischen Helden.\par Um die Bandbreite der \emph{Imago} vom Ingenieur und damit die verschiedenen, oftmals gegensätzlichen Physiognomien dieser \emph{Imago} weiter herauszuarbeiten, sei die Aufmerksamkeit nun auf den Gegenpol zum Ingenieur als großem Mann gelenkt, nämlich auf den kleinen, gesichtslosen Ingenieur, den farblosen \texttt{Funktionär}.\par Nirgendwo ist dieses Gesicht prägnanter gezeichnet worden als in Jewgenij Samjatins Roman \emph{Wir}, eine Dystopie \emph{avant la lettre}, in russischer Sprache verfasst, aufgrund der stalinistischen Zensur jedoch 1925 erstmals auf Englisch erschienen. Worum dieser Roman kreist: D-503, Konstrukteur und Mathematiker, lebt im sogenannten Einzigen Staat, der unschwer als politische Realität nach der Russischen Revolution zu decodieren ist, und arbeitet dort an der Fertigstellung des Raumschiffs Integral. Doch hören wir selbst:\begin{quote}
\par Ich, Nr. D-503, der Konstrukteur des \emph{Integral}, ich bin nur einer~der vielen Mathematiker des Einzigen Staates. Meine an Zahlen gewöhnte Feder vermag keine Musik aus Assonanzen und Rhythmen zu schaffen. Ich kann nur das wiedergeben, was ich sehe, was ich denke, genauer gesagt, was WIR denken. WIR \textendash{} das ist das richtige Wort, und deshalb sollen meine Aufzeichnungen den Titel WIR tragen.\footnote{\cite[][]{samjatin2003a}, S. 6.} 
\end{quote}
\par Greifbar wird hier der Gegensatz zum Ingenieur als großem Individuum. D-503 ist nämlich Teil der Vielen, ein »Molekül des Einzigen Staates«, wie es heißt, der unauffällige Mann aus der Menge.\footnote{\cite[][]{samjatin2003a}, S. 198.}  Anstatt durch individuelle Größe zeichnet er sich dadurch aus, eins zu sein mit dem titelgebenden WIR, dem Kollektiv dieser Gesellschaft. Mit diesem spezifischen Profil korrespondiert auch seine Rolle innerhalb des Einzigen Staates: Anstatt diesen \textendash{} wie beispielsweise Mafarka \textendash{} zu erneuern, besteht die Aufgabe D-503s darin, die, um im Bildfeld zu bleiben, Staatsmaschine am Laufen zu halten. Der Philosoph und Begründer des Kritischen Rationalismus, Karl Popper, hat 1947 in \emph{Utopie und Gewalt}, einem im Schatten des NS-Terrors verfassten Aufsatz, geschrieben: »Ziele müssen ihm {[}dem Ingenieur; RL{]} gesetzt werden; und was er tut, ist lediglich, Werkzeuge zu schaffen, mit denen diese Ziele verwirklicht werden können.«\footnote{\cite[][]{popper1968a}, hier S. 320.}  Mit D-503 hat Samjatin ebendiesem Typus der Höheren zudienenden Ingenieurfigur eine prägnante Gestalt verliehen.\par Deutlich wird der Kontrast zum Ingenieur als großem Mann auch an der folgenden Stelle, in der wir wie in jener zuvor an die Sicht der Tagebuch schreibenden Ich-Figur, D-503, gebunden sind:\begin{quote}
\par Feuersbrunst. In den Jamben erzittern die Häuser, sprühen als flüssiges Gold empor, stürzen donnernd zusammen. Die grünen Bäume krümmen sich, sinken, das Harz fließt \textendash{} und schon sind sie schwarze, verkohlte Skelette. Und da erschien Prometheus (damit sind natürlich wir gemeint): / \emph{Und zwang das Feuer in Maschin}\textquoteright{} \emph{und Stahl, / das Chaos in die Fesseln des Gesetzes}.\footnote{\cite[][]{samjatin2003a}, S. 47.} 
\end{quote}
\par Entscheidend hier die Wendung: \emph{Prometheus (damit sind natürlich wir gemeint)}. Mit dem Bezug auf den Prometheus-Mythos rekurriert Samjatin nämlich auf einen wiederkehrenden Vergleichspunkt für den Ingenieur. Kaum eine andere mythologische Figur wird im frühen 20. Jahrhundert, mit dem Ziel, dem vergleichsweise jungen Berufsbild des Ingenieurs Kontur zu verleihen, so häufig aufgegriffen wie diese. Man denke an einen Werbespruch der Firma Du Pont aus dem Jahr 1922, in dem es heißt: »THIS is today\textquoteright{}s Prometheus . . . . . Bringer of Comforts . . . . . The Chemical Engineer«,\footnote{\cite[][]{ov1922a}.}  an den Titel einer der führenden populärwissenschaftlichen Technikzeitschriften in Deutschland, \emph{Prometheus. Illustrierte Wochenschrift über die Fortschritte in Gewerbe, Industrie und Wissenschaft} (Berlin, 1889\textendash{}1921), oder noch an eine Technikgeschichte der Nachkriegszeit, David Saul Landes\textquoteright{} \emph{The Unbound Prometheus. Technological Change and Industrial Development in Western Europe from 1750 to the Present} (1969). In dem all diesen Formungen zugrunde liegenden Mythos wird Prometheus als ein Menschenfreund erinnert, der den Göttern das Feuer raubte, um es den Menschen auf die Erde zu bringen. Es ist die ungewöhnliche Art und Weise, in der Samjatin ebendiesen Mythos aufnimmt, \emph{Prometheus (damit sind natürlich wir gemeint),} in der das Gesicht \emph{seines} Ingenieurhelden noch einmal deutlich wird. Während nämlich der Prometheus-Mythos, wie das folgende Beispiel, Ludwig Brinkmanns 1908 publizierte Schrift \emph{Der Ingenieur}, exemplarisch verdeutlicht, gewöhnlich dazu dient, die Größe und Autonomiewerdung des Ingenieurs zu markieren,\begin{quote}
\par Dann aber {[}...{]} wird die Technik es verschmähen, lediglich Dienerin anderer zu sein, wird Selbstzweck und damit Kunst werden, und eine neue Kulturepoche wird aufgehen {[}...{]}. Dann wird der Ingenieur wieder aus einem Diener zu einem Herrn werden. {[}...{]} Dieser angeschmiedete Prometheus wird dann wieder frei sein, wird mit dem Feuer seines Geistes schaffen, um alles andere unbekümmert, nicht mehr leidvoller Handlanger anderer Zwecke, sondern sich selbst zur Genüge, da sein Schaffen Kultur ist.\footnote{\cite[][]{brinkmann1908a}, S. 85.} 
\end{quote}
\par ist er bei Samjatin \emph{expressis verbis} ein WIR. Prometheus ist hier kein übermenschliches, gegen die Autoritäten \textendash{} seien es die Götter der Antike oder die Höfe der Adeligen im 18. Jahrhundert\footnote{Zu denken ist hier an Goethes Gedicht \emph{Prometheus} aus dem Jahr 1773/4, dessen aufbegehrendes lyrisches Ich sich nicht nur, aber auch gegen die Aristokratie wendet.}  \textendash{} aufbegehrendes Einzelwesen, sondern Inbegriff eines Kollektivs von Konstrukteuren. Die Größe des Ingenieurs liegt gemäß Samjatins literarischer Fiktion in der kollektiven Anstrengung der kleinen Einzelteile \textendash{} man erinnere sich noch einmal an das Bild »Molekül des Einzigen Staates«\footnote{\cite[][]{samjatin2003a}, S. 198.}  \textendash{}, jedoch nicht im Glanz eines Einzelnen begründet. Die in dieser Dystopie in ein kritisches Licht getauchte \emph{Imago}, so möchte ich behaupten, ist Äußerung des kollektiven Imaginären im postrevolutionären Russland und wird nach dem Zweiten Weltkrieg in den Vorstellungswelten der DDR eine Renaissance erleben. Was der Seitenblick auf Samjatin und damit den Ingenieur als kleinen Mann jedoch ebenfalls verdeutlicht: Das der modernen Ingenieurfigur eigene Versprechen, technisches Know-how nicht als Selbstzweck, sondern für die Gesellschaft einzusetzen, bleibt auch hier hörbar \textendash{} ungeachtet der im übertragenen Sinne Größe oder Kleinheit der jeweiligen Ingenieurfigur.
\subsection[Großer Mann, ganz klein \textendash{}    Ingenieursatiren]{Großer Mann, ganz klein \textendash{} \- \protect\\ Ingenieursatiren}
\par Wo die Größe einer Figur, wie das beim Ingenieur während der Zwischenkriegszeit der Fall ist, besonders vehement betont wird, ist mit satirischen Reaktionen zu rechnen; mit Reaktionen nämlich, die auf diese Apotheose antworten, indem sie in der behaupteten Größe das tatsächlich Kleine entblößen. Satiren des Ingenieurs, wie sie auch in kanonischen Texten der literarischen Moderne, bei Robert Walser (\emph{Der Gehülfe}, 1908) oder Robert Musil (\emph{Der Mann ohne Eigenschaften}, 1930/2), auftauchen, beweisen \emph{ex negativo} nichts anderes als die imaginäre Aufladung des Ingenieurs zu Beginn des 20. Jahrhunderts, auf die sie eine kritische Antwort geben.\par Exemplarisch für dieses Gesicht der Ingenieurfigur sei ein 1925 in der Zeitung \emph{Arbeiterwille} publizierter Text genannt, Ernst Fischers \emph{Das Irrenhaus}. Ein Icherzähler berichtet hier von einem Traumerlebnis, in dem er, begleitet von einem Mann namens Johannes Tulemond, ein \emph{Institut} besichtigt.\footnote{\cite[][]{fischer1925a}, S. 5 (7.10.).}  Ebenda macht er die Bekanntschaft verschiedener Figuren, das heißt Berufsgruppen, u.a. des Arztes Nervus Schmerz, des Rassetheoretikers Thorismund Wicking, des Staatstheoretikers Justus Lex sowie des Ingenieurs Willibald Technikel.\footnote{\cite[][]{fischer1925a}, S. 5 (7.10.); S. 5 (4.10.); S. 5 (9.10.); S. 5 (4.10.).}  Bereits die Namen dieser Figuren zeigen an, dass es das Prinzip dieses Textes ist, diese Berufsgruppen in ein satirisches Licht zu tauchen. Doch worauf genau zielt die Satire Willibald Technikels, des Ingenieurs, der behauptet, das \emph{Perpetuum mobile},\footnote{\cite[][]{fischer1925a}, S. 6 (10.10.).}  einen Kreislauf erfunden zu haben, in dem sich Maschinen selbst reproduzierten, eigentlich ab? Hierzu zwei Passagen:\begin{quote}
\par Willibald Technikel stand in Hemdärmeln an einem riesigen Reißbrett, zeichnend und spuckend. Ich verstehe zu wenig von der darstellenden Geometrie, um zu beurteilen, ob er gut zeichnete, aber daß er vorzüglich spuckte, kann ich beschwören. Der Boden war außerdem von Zigarettenstummeln bedeckt. Als wir eintraten, pfiff er gerade eine Operettenmelodie und ich war eigentlich von dem großen Manne enttäuscht.\footnote{\cite[][]{fischer1925a}, S. 5 (10.10.).} \par Ich war natürlich begeistert, doch irgend etwas stimmte nicht. Da kniff Technikel plötzlich die Augen zusammen, sah mir schief ins Gesicht und lachte nervös: »Ich errate Ihre Gedanken \textendash{} aber Sie haben unrecht, ich habe auch das in Erwägung gezogen und habe die Lösung gefunden \textendash{} lächerlich einfach. Sie denken, das Eisen wird einmal ausgehen und was dann? Also, was dann, mein Lieber? Dann schmelzen wir selbstverständlich die ersten Maschinen ein und geben sie ihren Kindern zu fressen, und aus ihren Kindern werden sie wieder auferstehen. So wird die Technik auf Erden nie stillstehen, nie!«\footnote{\cite[][]{fischer1925a}, S. 6 (10.10.).} 
\end{quote}
\par Die erste Szene evoziert die Enttäuschung des Icherzählers, der offenbar, wie es im Text explizit heißt, einen großen Mann erwartet hatte und nunmehr einen Ingenieur sieht, der hinter ebendiese Vorstellung zurückfällt. Die zweite Sequenz erläutert die Hintergründe dieser Enttäuschung: Die Technik ist für ebendiesen Ingenieur ein Selbstzweck. Seine Erfindung zielt einzig und allein darauf ab, dass die Technik \emph{auf Erden nie stillstehen} werde.\par Vor der Folie der bis hier diskutierten Quellen, besonders Kellermann, Marinetti und Neurath, nach denen die imaginierte Größe des Ingenieurs darin besteht, technisches Know-how in den Dienst der Gesellschaft zu stellen, wird deutlich, dass Fischers Ingenieur ebendas gerade nicht anstrebt. Im Gegenteil zeigt diese kurze Erzählung den Ingenieur als einen Fachidioten. Sie zieht der kollektiv geteilten \emph{Imago} des Ingenieurs gewissermaßen den Zahn, indem sie ein Bild des Ingenieurs evoziert, dem seine Sache zu einem Selbstzweck geworden ist, der, anders gesagt, Technik um der Technik willen betreibt. In den theoretischen Registern dieses Essays gesprochen, macht Fischers Erzählung eine im kollektiven Imaginären seiner Zeit tief verankerte Vorstellung vom Ingenieur, wie sie in Film, Literatur und Politik aufgenommen, ja oftmals weiter verstärkt wird, als eine Übertreibung sichtbar. Sie erinnert an eine Wirklichkeit, die durch den gesellschaftlichen Aufstieg der Ingenieure, ihre neue Rolle als Vermittler populären Wissens in der Öffentlichkeit sowie die Ideen der Technokratiebewegung nicht aus der Welt geschaffen wurde, sondern gleichzeitig hierzu fortbestand; daran nämlich, dass Ingenieure in der Zwischenkriegszeit oftmals kleinbürgerliche Existenzen führten und ihnen im Gegensatz zu den ihrem Berufsfeld benachbarten Architekten auch der Ruf als höhere Klempner anhaftete.\footnote{\cite[][]{treue1967a}, S. 462.}  Das Bild eines Betriebsingenieurs aus dem Jahr 1933, abgebildet in August Sanders Fotoband \emph{Menschen des 20. Jahrhunderts. Porträtphotographien 1892}\textendash{}\emph{1952} gibt von dieser Realität des Ingenieurs, die, um es zu wiederholen, mit seinem gesellschaftlichen Aufstieg und seiner imaginären Aufladung als gesellschaftlicher Hoffnungsträger koexistiert, Zeugnis.\footnote{\cite[][]{sander1980a}, S. 138.}  Die in der Zwischenkriegszeit bestehende Kluft zwischen dem Ingenieur als einer Figur des kollektiven Imaginären und der von dieser \emph{Imago} unbeeindruckten Wirklichkeit, das heißt, zwischen dem Ingenieur als großem Mann und seiner Realität als kleinem Beamten, zwischen dem gesellschaftlichen Hoffnungsträger auf der einen und dem Fachidioten auf der anderen Seite ist nirgendwo so eloquent zum Ausdruck gebracht worden wie im zehnten Kapitel des \emph{Mann ohne Eigenschaften} von Robert Musil, einem Autor, der selbst eine Ingenieurausbildung absolviert hatte und deshalb über dieses Gefälle besser Bescheid wusste als viele seiner Zeitgenossen:\begin{quote}
\par Das war zweifellos eine kraftvolle Vorstellung vom Ingenieurwesen. Sie bildete den Rahmen eines reizvollen zukünftigen Selbstbildnisses, das einen Mann mit entschlossenen Zügen zeigte, der eine Shagpfeife zwischen den Zähnen hält, eine Sportmütze aufhat und in herrlichen Reitstiefeln zwischen Kapstadt und Kanada unterwegs ist, um gewaltige \texttt{Entwürfe} für sein Geschäftshaus zu verwirklichen. Zwischendurch hat man immer noch Zeit, gelegentlich aus dem technischen Denken einen Ratschlag für die Einrichtung und Lenkung der Welt zu nehmen oder Sprüche zu formen {[}...{]}.\par Es ist schwer zu sagen, warum Ingenieure nicht ganz so sind, wie es dem entsprechen würde. {[}...{]} Beiweitem gilt das natürlich nicht von allen, aber es gilt von vielen, und die, welche Ulrich kennen lernte, als er zum erstenmal den Dienst in einem Fabrikbüro antrat, waren so, und die, die er beim zweitenmal kennen lernte, waren auch so. Sie zeigten sich als Männer, die mit ihren Reißbrettern fest verbunden waren, ihren Beruf liebten und in ihm eine bewundernswerte Tüchtigkeit besaßen; aber den Vorschlag, die Kühnheit ihrer Gedanken statt auf ihre Maschinen auf sich selbst anzuwenden, würden sie ähnlich empfunden haben wie die Zumutung, von einem Hammer den widernatürlichen Gebrauch eines Mörders zu machen.\footnote{\cite[][]{musil2007a}, S. 37\textendash{}38.} 
\end{quote}
%
	%
%
	%
		%
			%
			%
			%
				\selectlanguage{ngerman}%
			%
			%
				%
					
						\chapter[head={Orte des Ingenieurs}, tocentry={ORTE DES INGENIEURS}]{Orte des Ingenieurs}%
					
					\-
					\par			
					\newpage%
					\noindent%
				%
			%
			%
				%
					\regularfontdefault%
				%
			%
		%
	%
	%
	\par\noindent Der Historiker Charles Steven Maier hat die Behauptung aufgestellt, dass sich eine Apotheose des Ingenieurs, wie sie die Zeit zwischen den beiden Weltkriegen erlebt habe \textendash{} er spricht von \emph{engineering as social redemption} \textendash{}, erst wieder nach dem Zweiten Weltkrieg beobachten lasse.\footnote{\cite[][]{maier1970a}, S. 61.}  Dieser Befund ist nicht ganz zutreffend, führt man sich die Flut an völkischen Zukunftsromanen vor Augen, die während der Herrschaft des Nationalsozialismus industriell produziert und massenhaft gelesen wurden und in denen Ingenieure zum Kernbestand des Figurenpersonals zählen.\footnote{Das ist in einigen Fällen schon an den Romantiteln erkennbar, wie die Bibliografie aus Dina Brandts Geschichte dieser Gattung zeigt: \cite[][]{brandt2007a}.}  Maiers Behauptung ist aber insofern richtig, als eine Fülle an Vorstellungen vom Ingenieur, wie wir sie in der Zwischenkriegszeit beobachten, erst wieder im Zuge des Kalten Krieges sichtbar wird, als sich die Gegenüberstellung zweier unvereinbar scheinender Weltanschauungen formiert. Erst da bildet der Ingenieur als eine Figur des kollektiven Imaginären wieder eine mit der Zwischenkriegszeit vergleichbare Vielfalt an Gesichtern aus, deren Kern aber auch hier die Vorstellung bleibt, dass Ingenieure dazu in der Lage seien, ihr technisches Know-how zur Gestaltung einer besseren Welt einzusetzen. Um einen Zugriff auf diese Figurenspannung zu gewinnen, wurde im vorangegangenen Abschnitt anhand der Frage nach den Proportionen des Ingenieurs besonders die Achse groß/klein in den Fokus gerückt. Dieser Spannung soll nun anhand der Kategorie der Orte des Ingenieurs nachgegangen werden, die ihn als eine Figur der Konstruktion bzw. Destruktion beleuchtet. Inwiefern, so die Leitfrage dieses Abschnitts, exponieren die verschiedenen Arbeitsorte des Ingenieurs ihn als eine Figur der Gestaltung oder Zerstörung?\par Für die Bedeutung des Ingenieurs im kollektiven Imaginären des Kalten Krieges spricht, dass C. P. Snow, ein englischer Gelehrter und Schriftsteller, in einer der wirkmächtigsten Reden dieser Epoche, \emph{The Two Cultures} (1959), auf ebendiese Figur zu sprechen kommt. Snow gelangt dort zu dem Befund, dass die von ihm beobachtete Kluft zwischen den beiden Kulturen, literarischer und naturwissenschaftlicher Intelligenz, in Russland kleiner sei als in England und den USA und dies daran abzulesen wäre, dass Ingenieure in der russischen Literatur viel selbstverständlicher vorkämen als in der US-amerikanischen oder englischen: »Ein Ingenieur wird, wie es scheint, in einem sowjetischen Roman ebenso selbstverständlich hingenommen wie ein Psychiater in einem amerikanischen.«\footnote{\cite[][]{snow1987a}, hier S. 47. Snow zeigt sich über die Tatsache besorgt, dass nirgendwo so viele Ingenieure ausgebildet würden wie in der Sowjetunion. Vgl. S. 46.}  Möchte man anstatt nur von der russischen Literatur von der des ehemaligen Ostblocks sprechen, so erscheint Snows Behauptung eines Übergewichts an Ingenieuren aus heutiger Perspektive überzeichnet. Denn gerade in der angloamerikanischen Unterhaltungsliteratur des Kalten Krieges wimmelt es nur so von heldenhaften Ingenieuren. Man denke an die seit den 1960er-Jahren ein Massenpublikum erreichende Science-Fiction-Literatur, wie sie zunächst in \emph{pulp magazines} distribuiert wurde; Geschichten aus der Feder eines Isaac Asimov, Arthur C. Clarke oder Ray Bradbury. Aber Snow, dessen Einschätzung aus der Zeit vor der Renaissance der Science-Fiction stammt und der keine Gelegenheit zu einer rückblickenden Zusammenschau hatte, ist insofern im Recht, als Ingenieure in den Vorstellungswelten des Ostens eine eminent wichtige Rolle spielen, und dort, genauer in der Literatur der DDR, vor allem auf Baustellen gezeigt werden.
\subsection[Auf der Baustelle:    Der Ingenieur als eine Figur    der Konstruktion]{Auf der Baustelle: \- \protect\\ Der Ingenieur als eine Figur \- \protect\\ der Konstruktion}
\par Hierfür (und damit für den Typus des konstruierenden Ingenieurs) exemplarisch ist zweifellos Heiner Müllers Theaterstück \emph{Der Bau}, geschrieben 1963/4, ein Jahr später publiziert, jedoch aufgrund \texttt{seiner~regimekritischen~Ausrichtung} erst 1980 uraufgeführt.\footnote{\cite[][]{emmerich1996a}, S. 218.}  Müllers Stück, das Motive des 1964 erschienenen Romans \emph{Spur der Steine} von Erik Neutsch aufgreift, ist in der Differenzierung seiner Orte überaus genau: Leser*innen und Zuschauer*innen des Stücks werden in (Erfinder-)Büros, an Kraft- oder Wasserwerke,\footnote{\cite[][]{mueller1981a}, hier S. 47 und S. 80; S. 45; S. 46 und S. 58.}  vor allem aber an eine zum Stillstand gekommene Baustelle geführt, dem vielleicht wichtigsten Ort des Stücks, an dem handlungsleitende Konflikte ausagiert werden:\begin{quote}
\par \emph{Baustelle}\par \emph{Donat. Bezirkssekretär. Stellvertreter}.\par \emph{Bezirkssekretär} Es gefällt mir nicht, dass ich die Baustelle wiedererkenne. Mit der Ausnahme, die mir noch weniger gefällt: auf jedem Abschnitt rosten zwei Exemplare neuer Technik mehr. Ein Tag ist ein Jahr, was hast Du getan in dreißig Jahren?\footnote{\cite[][]{mueller1981a}, S. 60.} 
\end{quote}
\par Müllers Figurenpersonal, bestehend aus Bezirks- und Parteisekretären, zwei Ingenieuren namens Hasselbein und Schlee, Brigadieren, einem Oberbauleiter und einer Reihe von Arbeitern, wird erst vor der Folie einer spezifischen Form der Arbeitsorganisation der DDR dieser Jahre verständlich. Dieses Figurenensemble erinnert stark an eine Arbeitsbrigade, eine Form der Arbeitsteilung, wie sie sich in der Sowjetunion der 1920er-Jahre ausgebildet hat und in der neu gegründeten DDR weiterentwickelt wurde. Arbeitsbrigaden der DDR umfassten \textendash{} Müllers Stück vergleichbar \textendash{} etwa sieben bis zwölf Personen und wurden mit dem Anspruch organisiert, den neuen Menschen in die Praxis des sozialistischen Arbeitens einzuführen, das heißt, die für ihn vorgesehenen Beziehungen kameradschaftlicher Zusammenarbeit durchzusetzen.\footnote{Vgl.: \cite[][]{roesler1994a}, hier S. 144\textendash{}158.} \par Die Baustelle, an der sich die Geschichte von Müllers Brigade nun entfaltet \textendash{} einmal wird sie als »sozialistische Großbaustelle{[}{]}«\footnote{\cite[][]{mueller1981a}, S. 72.}  bezeichnet \textendash{}, ist unschwer als ein Mikrokosmos zu erkennen, der stellvertretend für den Makrokosmos des sozialistischen DDR-Staates steht. Besser: Die Konflikte auf Müllers Baustelle stehen so zum einen für sich, meinen zum anderen aber auch jene im Staate der DDR.\footnote{Das Stück, so Riewoldt, könne als eine gesamtgesellschaftliche Metapher gelesen werden. Vgl.: \cite[][]{riewoldt1983a}, hier S. 163.} \par Mit Blick auf dieses metaphorische Potenzial des Ortes erwähnenswert ist, dass sich eine der gefährlichsten Eskalationen der DDR-Geschichte, ein Moment, an dem Teile der Bevölkerung gegen die SED-Spitze den Aufstand probten, 1953 auf einer Baustelle ereignete.\footnote{\cite[][]{stoever2008a}, S. 45.}  Und auch in der politischen Rhetorik der DDR-Funktionäre sind Metaphern des Bauens wiederkehrend: Auf der zweiten Parteikonferenz der SED am 12. Juli 1952 wird zur grundlegenden Aufgabe der DDR-Politik der »Aufbau des Sozialismus« erklärt.\footnote{\cite[][]{stoever2008a}, S. 44.} \par Der Schauplatz Baustelle bot Autor*innen der DDR-Literatur so eine diskursive Zone der Aushandlung politischer Konflikte. Im Falle von Müllers \emph{Bau} besteht dieser Konflikt darin, dass das Baugeschehen vor Ort mit dem verordneten Plan in Konflikt gerät, dass Planung und tatsächliche Produktion also auseinanderklaffen.\par Dass nun die Baustelle eine solche bleibt, zeigt auch den Ingenieur als eine Figur in der Krise. Mit Blick auf das metaphorische Potenzial von Müllers zentralem Schauplatz als einer Baustelle des sozialistischen Staates lassen sich für seine Ingenieure zwei Befunde anführen: Zum einen greift er mit seinem \emph{setting} die \texttt{bereits~aus} der Zwischenkriegszeit bekannte \emph{Imago} vom Ingenieur als \texttt{einem~Ingenieur} der Gesellschaft auf (die technische Arbeit auf der Baustelle ist als Arbeit an der sozialistischen Organisation codiert). Zum anderen aber verzeichnen wir hier, betrachten wir die Handlungsführung des Stücks, auch eine Abschwächung dieser Vorstellung, zumal die Ingenieure am Umbau der Gesellschaft ja scheitern.\par Aufmerksamkeit verdient nun, wie genau Müller, ein Kenner der Literaturgeschichte und antiken Mythologie \textendash{} er übersetzte 1967/8 den \emph{Prometheus} (!) von Aischylos \textendash{}, dieses Scheitern des Gesellschaftsingenieurs inszeniert.\footnote{Zu Müllers Aischylos-Übersetzung vgl.: \cite[][]{emmerich1996a}, S. 219. Überhaupt spielt die Antike hier eine besondere Rolle. In den einander kreuzenden Figurenreden des Dramas wird die Arbeit des Ingenieurs im Lichte antiker Stoffe betrachtet. Hierfür einschlägig ist ein Dialog zwischen Hasselbein und einem Maler, in dem Letzterer sagt: »Das {[}gemeint ist die Arbeit an der Baustelle; RL{]} erinnert mich an einen Kriminalfall aus der Bibel. Der Turmbau. Fragment geblieben durch Mangel an Kooperation. Der Bauherr, Nebukadnezar oder hab ich seinen Namen vergessen, wurde mit einer eigenen Sprache bestraft. Er konnte nur noch mit sich selber reden, blubbblubbb und blabla. Das ist der Formalismus.« \cite[][]{mueller1981a}, S. 97. Es handelt sich um eine versteckte, aber als kritisch erkennbare Referenz auf die Entkoppelung zwischen der Sprache der Mächtigen (Formalismus) und jener der Arbeiter, zwischen der der Befehlsgeber und Ausführenden, die den Bau hier wie im alttestamentarischen Babel zum Stocken bringt.}  Für uns entscheidend ist eine Passage, in denen die Ingenieurfigur Hasselbein von der des Hamlet überblendet wird. Aus einem Gespräch der beiden Ingenieure:\begin{quote}
\par HASSELBEIN: {[}...{]} wer mit dem Kopf durch die Wand will, muss die Wand im Kopf haben, der Volkswirtschaftsplan ein homerisches Epos, Menschen die Götter im Wettlauf mit ihren Terminen unterm Fuß der Zeit. »Kommunismus, Endbild, immer erfrischtes, mit kleiner / Münze zahlt ihn der Alltag aus, unglänzend, von Schweiß blind.« Praxis, Esserin der Utopien. SCHLEE: Wasch dir die Nacht aus dem Gesicht, mein Prinz. HASSELBEIN: Und mach dem guten Dänemark schöne Augen.\footnote{\cite[][]{mueller1981a}, S. 56. Bei der in Anführungszeichen stehenden Sequenz handelt es sich um ein Selbstzitat aus Müllers Gedicht \emph{Bilder} (1949). Eine weitere Hamlet-Referenz befindet sich: ebd. S. 68.} 
\end{quote}
\par Keine andere Gestalt der Weltliteratur repräsentiert den handlungsschwachen Zweifler derart einprägsam wie William Shakes\-peares Hamlet. Mit dieser Referenz akzentuiert Müller Eigenschaften seiner Ingenieurfigur, wie sie auch aus anderen Passagen hervorgehen, nämlich Orientierungslosigkeit und Entscheidungsschwäche: \- \protect\\ \\ Erstes Beispiel: Figurenrede BARKAS, des Brigadiers:\begin{quote}
\par \emph{auf Hasselbein} fragen Sie den, er weint alle Tage darüber und ändert nichts, auch ein Ingenieur.\footnote{\cite[][]{mueller1981a}, S. 53.} 
\end{quote}
\par Zweites Beispiel: Figurenrede DONATS, des Parteisekretärs:\begin{quote}
\par Die Zukunft, Ingenieur, ist aus der Mode \textendash{} schimpfen Sie \texttt{weiter}, Sie sind parteilos. Baun ist ein \texttt{Fehler}. Warten ist ein Fehler. Was tun wir?\footnote{\cite[][]{mueller1981a}, S. 63.} 
\end{quote}
\par Die Überblendung von Hamlet- und Ingenieurfigur \textendash{} die mit Blick auf Müllers Ingenieur zentrale Operation \textendash{} führt dazu, dass der Ingenieur Hasselbein als ein Schwächling in Szene gesetzt wird. Das ist bemerkenswert, bedenkt man seine Idolatrie als großer Mann während der Zwischenkriegszeit, aber auch in synchroner Perspektive, das heißt vor dem Hintergrund jener Vorstellungen, von denen Ingenieure in Müllers unmittelbarer Schreibgegenwart umgeben sind.\par Während der Ingenieur im kollektiven Imaginären der DDR \textendash{} das zeigen uns politische Plakate dieses Staates \textendash{} zwar nicht als großer Mann, sehr wohl aber als für den DDR-Bürger geltendes Vorbild intakt ist, ›baut‹ Müller ihn in seinem Stück zu einem Zweifler um.\footnote{Vgl.: \cite[][]{unknown-ddr}.}  Mit Blick darauf wird man zu dem Schluss kommen, dass Müller, indem er seine Ingenieure auf einer nicht fertig werden wollenden Baustelle (ver)zweifeln lässt, eine im kollektiven Imaginären der DDR verankerte Figur aufnimmt und mit dem Ziel, den Sozialismus zu kritisieren, umformt.\footnote{Beispielhaft für diese Überforderung des Ingenieurs: »Ein Auto und ein Fahrrad in ein Flugzeug umbauen während der Fahrt, das ist ungefähr unsere Aufgabe.« \cite[][]{mueller1981a}, S. 49.}  Er kommt der Vorgabe staatlicher Institutionen, die sozialistische Produktionssphäre zum Thema der Fiktion zu machen, nach,\footnote{Die erste Bitterfelder Konferenz (1959) beschloss, Literatur und Arbeitswelt zu verbinden. Vgl.: \cite[][]{schmitt1983b}, hier S. 56. Der Auftrag an die Literatur lautete, die sozioökonomischen Umwälzungen voranzubringen. Vgl.: \cite[][]{riewoldt1983a}, S. 144; S. 156.}  aber anders als geplant \textendash{} auch das ein Bruch zwischen Planung und Umsetzung. \emph{Der Bau} wird, weil er durch die Gestaltung eines handlungsschwachen Ingenieurs den sozialistischen Aufbau problematisiert, auf den Index gesetzt.\footnote{Weitere Hinweise zur Rezeption des Stücks gibt: \cite[][]{emmerich1996a}, S. 215.} 
\subsection[Im \emph{War Room}: Der Ingenieur als    eine Figur der~Zerstörung]{Im \emph{War Room}: Der Ingenieur als \- \protect\\ eine Figur der~Zerstörung}
\par Während der Schauplatz Baustelle den Ingenieur \textendash{} Krise hin oder her \textendash{} als eine konstruierende, oder besser: einer Konstruktion nachgehende Figur zeigt, macht ihn sein Auftreten in der militärischen Befehlszentrale der US-amerikanischen Regierung, deren \emph{War Room}, als eine Figur der Destruktion sichtbar. So geschehen in Stanley Kubricks klassisch gewordenem Film \emph{Dr. Strangelove, Or How I Learned to Stop Worrying and Love the Bomb}, der 1963, also fast zeitgleich mit Müllers \emph{Der Bau} erschienen ist. Mit dem Übergang von einer Baustelle zu einer militärischen Befehlszentrale vollzieht sich ein Wechsel von einem Ort der technischen Praxis zu einem der technischen Planung:\footnote{Die Formulierung »Orte der technischen Praxis« findet sich bei Popplow, der diesbezüglich von Baustelle, Werft, Werkstatt und Bergrevier spricht. Vgl.: \cite[][]{popplow2007a}, S. 965.}  Strangelove ist als eine beratende, kalkulierende, abwägende, jedoch an keinem Punkt tatsächlich agierende Figur gezeichnet.\footnote{\cite[][]{kubrick1963a}, 49:05\textendash{}51.46; 01:24:48\textendash{}01:28:51. Der Film basiert auf \cite[][]{bryant1958a}.} \par Kubrick präsentiert uns diese Ingenieurfigur, deren deutscher Akzent sie zu einer Karikatur des ehemaligen NS-Ingenieurs Wernher von Braun macht, als sie gerade über die sogenannte »doomsday machine« spricht, eine Waffe, die von beiden Supermächten, Russland und den USA, bereits hergestellt werden und die Menschheit fast zur Gänze ausrotten könne.\footnote{Die Idee zur \emph{doomsday machine} soll von Hermann Kahn inspiriert worden sein. Vgl.: \cite[][]{erickson2013a}, S. 87.}  Strangelove, der sich selbst als »director of weapons research« bezeichnet, befindet sich während seiner Ausführungen mit den mächtigsten Männern der Welt an einem ovalen, nach den höchsten Standards der damaligen Zeit ausgestatteten Konferenztisch: dem amerikanischen Präsidenten und seinen Generälen, einem russischen Gesandten sowie dem russischen Präsidenten selbst, der in einem Telefonat, das einen irrtümlichen Erstschlag der Amerikaner auf Russland und damit einen Atomkrieg abwenden soll, zugeschaltet wird. Im Hintergrund dieser Szene ist eine exorbitante Weltkarte zu sehen, auf der Raketenflugbahnen abgebildet sind. Zweifellos also befindet sich der seine technische Expertise (mit)teilende Ingenieur an einem Schauplatz der Macht.\par Die hier nicht wiederzugebende Sprechweise des Ingenieurs evoziert eine Schadenfreude, ja einen Sadismus und illustriert exemplarisch, dass die Ingeniosität dieses Mannes den destruktiven Machtspielen der Politik gerne zu Hilfe eilt.\par Blickt man von der Figur des Dr. Strangelove auf uns bereits bekannte Ingenieurfiguren zurück, Mac Allan aus Kellermanns Bestseller bzw. Elveys Verfilmung, oder Mafarka aus \texttt{Marinettis} Roman, wird deutlich, dass das Eliminieren von Leben, in den \textendash{} im übertragenen Sinne \textendash{} Physiognomien dieser älteren Ingenieurfiguren durchaus schon angelegt ist, in der \texttt{Charakterisierung} \texttt{Strangeloves} aber eine neue Dimension erfährt. Ich möchte behaupten, dass in der Zerstörungskraft von Kubricks Ingenieur\-figur ein kollektives Imaginäres des Kalten Krieges seinen Ausdruck findet, das sich etwas vorläufig als Furcht vor den Potenzialen der Technik fassen lässt. Leo Marx hat in einem nach dem Kalten Krieg verfassten Aufsatz mit dem Titel \emph{The Idea of} ›\emph{Technology}‹ \emph{and Postmodern Pessimism} die Behauptung aufgestellt, dass die Fortschrittsverheißungen durch Technik (\emph{technology}) nach 1945 immer stärker unter Druck geraten.\footnote{\cite[][]{marx1994a}, hier S. 21\textendash{}22. Marx prägt ein spezifisches Verständnis von technology: Er rekonstruiert, dass zwischen 1870 und 1920 sogenannte »large complex systems« entstanden seien, Telegrafen-, Telefon- und Elektrizitätssysteme (ebd. S. 16), im Zuge deren Etablierung das Konzept der »mechanic arts« von dem der »technology« ersetzt worden wäre (ebd. S. 16). Die Idee, dass »technology« ein autonomer Agent sozialer Veränderung sei, verliere, so Marx, nach 1945 an Glaubwürdigkeit (ebd. S. 19).}  Er nennt exemplarisch die amerikanische \emph{counterculture} der 1960er-Jahre, die, so Marx, dafür gekämpft habe, die Technik (\emph{technology}) moralischen Zielsetzungen unterzuordnen und nicht umgekehrt.\footnote{\cite[][]{marx1994a}, S. 23.}  Neben die Vorstellung von Technik als einem Mittel des Fortschritts trete nun also immer deutlicher ein pessimistisches Bild von ihr.\par Die spezifische Bedrohung durch Technik im Zeitalter des Kalten Krieges, genauer gesprochen, das In-den-Dienst-Stellen technischer Expertise für Machtpolitik, ungeachtet deren moralischer Implikationen \textendash{} und hierfür steht Kubricks Karikatur eines Ingenieurs \textendash{}, hat dazu geführt, dass Dr. Strangelove auch als »the twentieth-century Faust« bezeichnet wurde.\footnote{\cite[][]{smith2008a}, hier S. 103.}  Mit diesem Vergleich ist die oft unheilvolle Allianz von Technik und Macht, wie sie sich in der Rede des Ingenieurs im \emph{War Room} konkretisiert, vor dem Hintergrund des topisch gewordenen Pakts zwischen dem Gelehrten und dem Teufel gezeigt. Faust-Vergleiche ähnlicher Art sind in den Darstellungen der Biografien von NS-Ingenieuren alles andere als singulär. Zu denken ist an Rolf Hochhuths Drama \emph{Hitlers Dr. Faust} (1991/2000), das den Raketeningenieur Hermann Oberth als Faust und Adolf Hitler als dessen Mephisto zeigt, oder an den Nachruf des \emph{Washington Star} auf Wernher von Braun, um es zu wiederholen, Strangeloves Blaupause, in dem im Juni 1977 zu lesen war:\begin{quote}
\par A kind of Faustian shadow may be discerned in \textendash{} or imposed on \textendash{} the fascinating career of Wernher von Braun: A man so possessed of a vision, of an intellectual hunger, that any accommodation may be justified in its pursuit.\footnote{\cite[][]{leucht2018a}, hier S. 379\textendash{}380.} 
\end{quote}
\par Blickt man von hier auf das zu den Ingenieuren des Kalten Krieges bisher Gesagte zurück, wird man vorläufig festhalten können, dass in ihren verschiedenen Gesichtern, als Konstrukteur des sozialistischen Staates einerseits, als potenzieller Zerstörer menschlichen Lebens andererseits, der dialektische Status der Technik als Mittel des Fortschritts bzw. der (Selbst-)Vernichtung seine künstlerische Formung findet. Dabei bilden destruktive Ingenieurfiguren vom Zuschnitt eines Dr. Strangelove während des Kalten Krieges ein Paradigma. Oftmals schwierig von der Figur des \emph{mad scientist} zu unterscheiden, begegnen sie uns besonders in der Trivialkultur, nicht nur des Westens (dort in der schon erwähnten Science-Fiction-Renaissance seit den 1960er-Jahren), sondern bspw. auch in den Comicserien des DDR-Grafikers Hannes Hegen.\footnote{So bspw. in dessen Geschichte \emph{Die neue Sonne}, die erstmals 1958/59 erschienen ist. \cite[][]{hegen2011a}. Auf die unscharfen Grenzen zwischen den Tätigkeitsfeldern des Ingenieurs und denen anderer Figuren, bspw. des Erfinders, weist Popplow hin. Vgl.: \cite[][]{popplow2007a}, S. 952. Zur Figur des \emph{mad scientist}, vgl.: \cite[][]{sarasin2003a}.}  Eine besondere Auffälligkeit dieses Ingenieurtyps hat der Historiker Bernd Stöver angedeutet, der den Kalten Krieg in Kubricks Film als »männliche Sexualneurose« dargestellt sieht.\footnote{\cite[][]{stoever2008a}, S. 63.}  Und tatsächlich gibt es innerhalb dieser destruktiven Figurenvariante eine Verknüpfung politischer und sexueller Gewaltfantasien. Hierfür exemplarisch: Eine Figur aus Thomas Pynchons Roman~\emph{Gravity}\textquoteright{}\emph{s Rainbow} (1973) namens Blicero, ein Sadist und Ingenieur, in dessen Profil sich die zwei Bedeutungsschichten von Pynchons Romanmotiv, der Rakete, als technische Avantgarde und Phallus verbinden. Es handelt sich um eine Ingenieurfigur, die von einer gewaltsamen Parallele geprägt ist: So wie ihr sexuelle Erregung niemals ohne ein Opfer möglich ist, so ist auch technischer Fortschritt für sie ohne Gewalt nicht zu haben.
\subsection[Über Ingenieurinnen \textendash{} und Wider\-stände    im kollektiven Imaginären]{Über Ingenieurinnen \textendash{} und Wider\-stände \- \protect\\ im kollektiven Imaginären}
\par Die \textendash{} im Rückgriff auf Hausmann gesprochen \textendash{} in der Physiognomie dieser Ingenieurfigur sichtbar werdende Engführung von technischer Expertise und männlicher Sexualität erlaubt es, den Blick auf eine Frage zu richten, die den gesamten Essay durchzieht, jene nach der geschlechtlichen Gebundenheit des Ingenieurs als einer Figur des kollektiven Imaginären. Anders, besser: Wie steht es um die Ingenieurin in den Vorstellungswelten des 20. Jahrhunderts?\par In der wahrscheinlich einschlägigsten Untersuchung zu diesem Problem, \emph{Mann und Maschine: Eine genealogische Wissenssoziologie des Ingenieurs und der modernen Technikwissenschaften, 1850\textendash{}1930,} rekonstruiert die Soziologin Tanja Paulitz, dass die Maskulinisierung des Ingenieurberufs keinesfalls von Anfang an gegeben war, sondern erst in die Zeit des ausgehenden 19. Jahrhunderts fällt. Anhand der Schriften von Alois Riedler, Ende des 19. Jahrhunderts Rektor der TH Berlin Charlottenburg (die heutige TU Berlin) und ein maßgeblicher Akteur in den Debatten um die Professionalisierung des Ingenieurs, zeigt Paulitz exemplarisch, wie um 1900 das wirkmächtige Bild eines explizit männlich markierten Ingenieurs, eines \emph{Mannes} der Tat, das eines geschlechtlich weniger markierten Maschinenwissenschaftlers ablöst.\footnote{\cite[][]{paulitz2012a}, S. 341. Ein weiterer Schwerpunkt von Paulitz\textquoteright{} Rekonstruktion liegt auf den Abgrenzungskämpfen innerhalb der Ingenieurzunft, zwischen einem stärker szientifisch bzw. praktisch ausgerichteten Berufsbild, wobei Riedel Letzteres vertrete.}  Während sich letzteres Berufsbild, so Paulitz, noch auf die Kategorie Mensch gestützt habe und bildungsbürgerlich ausgerichtet gewesen wäre, sei jenes des Ingenieurs von Beginn weg geschlechtlich einseitig markiert und gegen die Werte des Bürgertums gerichtet gewesen. Im kollektiven Imaginären, den gesellschaftlich geteilten Vorstellungen vom Ingenieur, hat sich die geschlechtlich einseitig codierte Spielart durchgesetzt und verfestigt; will sagen, dass der Ingenieur, besonders in seiner Variante als Hoffnungsfigur, eine \emph{Imago} männlichen Zuschnitts bildet.\footnote{Noch 2020 wirbt eine der renommiertesten Schweizer Bildungsinstitutionen, die ETH Zürich, ausschließlich mit einem männlichen Werbeträger für ihre Ingenieurausbildung. Vgl.: \cite[][]{wuersten2020a}.} \par Zwei Beispiele aus der Zeit des Kalten Krieges mögen jedoch zeigen, wie die Literatur diese geschlechtlich einseitige Markierung durchaus registriert hat und darüber hinaus auch zu problematisieren versteht. In Müllers schon diskutiertem Stück \emph{Der Bau} begegnen wir neben dem passiv-zweiflerischen Hamlet-Ingenieur Hasselbein \textendash{} bereits das ist Arbeit an der geschlechtlichen Markierung des Ingenieurs: Man bedenke die lange Debatte über Hamlets Weiblichkeit (!) \textendash{} noch einer zweiten Ingenieurfigur namens Schlee. Es handelt sich um eine der in der Literatur des 20. Jahrhunderts vergleichsweise seltenen Ingenieurinnen, deren Arbeit auf der Baustelle sogleich mit Sexismen begegnet wird: »Das schickt mir mein Ministerium, wenn ich Ingenieure brauche. Eine Ballerina ...«,\footnote{\cite[][]{mueller1981a}, S. 48.}  so der Oberbauleiter, oder: »... Ingenieurin ... Sie hat Waffen, die uns nicht zur Verfügung stehn«,\footnote{\cite[][]{mueller1981a}, S. 77.}  so ihr männlicher Kollege Hasselbein. Aber die angesprochene Ingenieurin pariert diese Diskriminierungen: »Ich bin als Ingenieur hier, die Frau ist Nebensache.«\footnote{\cite[][]{mueller1981a}, S. 52.}  Müllers Entscheidung, eine Ingenieurin als Heldin in Erscheinung treten zu lassen, registriert den Umstand, dass in der DDR Karrieren von Frauen in technischen Berufen durchaus gefördert wurden.\footnote{\cite[][]{merkel1994a}, hier S. 368\textendash{}369; \cite[][]{gerhard1994a}.}  Beinahe 90 \% der Frauen waren dort berufstätig. Und auch in Illustrierten wie \emph{Die Frau von heute} oder \emph{Für Dich} wurden Kampagnen lanciert, um Frauen als Ingenieurinnen oder Erfinderinnen zu zeigen.\footnote{\cite[][]{merkel1994a}, S. 368\textendash{}369.} \par Ich möchte den Konflikt, den Müller um seine Ingenieurin Schlee spinnt, als eine kritische Reaktion auf ein markiert männliches Bild des Ingenieurs lesen. Denn Schlee wird von dem Vater ihres Kindes verlassen, mehr noch: verleugnet dieser Mann das gemeinsame Kind, während die Ingenieurin selbst es annimmt. Das Bild der Ingenieurin als Technikerin \emph{und} Mutter, die gleichermaßen in den Sphären des Mechanischen wie Organischen zu Hause ist, hebt eine Beschränktheit der viel stärker verbreiteten \emph{Imago} des männlichen Ingenieurs als \emph{lone-wolf inventor} hervor.\par Während Müllers Ingenieurin noch passiv bleibt, artikuliert eine andere Ingenieurin der DDR-Literatur, Tinka aus Volker Brauns gleichnamigem Drama, was ich eine dezidiert weiblich codierte Ingenieurfantasie nennen möchte. Auch Brauns In\-genieurin, die von einer Arbeiterin zu einer Ingenieurin für Automatisierung avanciert ist,\footnote{\cite[][]{braun1990a}, hier S. 136. Vgl.: \cite[][]{emmerich1996a}, S. 224.}  wird mit sexistischer Diskriminierung konfrontiert. In der Rede eines männlichen Akteurs ist die Auffassung der natürlichen Einheit Ingenieur/Mann bzw. Erotik/Frau artikuliert:\begin{quote}
\par Ein klarer Fall. Als sie {[}Tinka; RL{]} studierte, ging es noch; Ingenieur \textendash{} das ist zuviel. Das ist die Emanzipierung: sie macht sich von sich selbst frei. Die Frauen baun sich nach dem Bild des Manns um, statt was aus sich zu machen. Als wenn wir alles wären, was der Mensch sein kann. Wozu leben wir? Um die Welt zu tapeziern mit Konstruktionen. Für Wasserhähne und Vergaser. Wir sind mehr die Erben Newtons als die von Marx. Die Natur ist nicht so wichtig wie die Technik, unsre Natur schon gar nicht. Emanzipierung \textendash{} müßte auch Erotisierung sein, sonst kommt die Frau zu sich, aber Mann und Frau nicht.\footnote{\cite[][]{braun1990a}, S. 151.} 
\end{quote}
\par Entscheidend hier: die Formulierung, Frauen \emph{bauten} sich, würden sie Ingenieurinnen, nach dem Mann \emph{um}. Das ist Ausdruck einer als natürlich erachteten Männlichkeit des Ingenieurberufs, wie sie im kollektiven Imaginären tatsächlich fest verankert ist, von Paulitz aber als Ergebnis eines historischen Prozesses sichtbar gemacht wird; eines Prozesses, in dem Akteure eines noch neuen Berufsfeldes zur Durchsetzung ihrer Interessen Ausschlüsse produziert haben: antiszientifisch, antibürgerlich, antiweiblich. Spektakulär nun, wie Brauns Tinka gegen dieses \emph{Imaginäre} ankämpft:\begin{quote}
\par Könnt ich mir \emph{selber basteln}, den ich liebe. Nach meinen Wünschen seinen Körper \emph{baun} und sein Gehirn fülln. Da wir uns schon nicht mit unsern Kindern paaren können, die besser sind. Ihn zu dem machen, was er mir sein soll \textendash{} daß ich ihn brauch wie mich!\footnote{\cite[][]{braun1990a}, S. 157 (meine Hervorhebungen; RL).} 
\end{quote}
\par In diesen Worten Tinkas, die zuvor ihren Mann betrogen hat,~wird nichts Geringeres artikuliert als der Wunsch, sich den Partner nach eigenen Vorstellungen bauen zu können, worin eine Vorstellung von Ingenieurarbeit aufblitzt, auf die wir im letzten Abschnitt dieses Essays zurückkommen werden. Zweifellos handelt es sich bei Tinkas Figurenrede um gesellschaftliche Normen verletzende \texttt{Sätze.~Aus} der Perspektive des vorliegenden Essays passiert hier jedenfalls etwas in zweierlei Hinsicht Spektakuläres: Wir beobachten eine Verschiebung des uns nun von zahlreichen männlichen Ingenieurfiguren bekannten \texttt{Machbarkeitspostulats~1)} in \texttt{die~Rede} einer Frau und 2) von der Sphäre der Gesellschaft in die des Privaten.\par Man wird Braun vorwerfen können, dass ihm eine Umwertung der männlich markierten Ingenieurfigur nicht ganz gelingt, zumal die Fantasien seiner Ingenieurin auf den traditionell als weiblich konnotierten Bereich des Privaten beschränkt bleiben, jenen der Gesellschaft aber nicht direkt berühren, und Tinka zudem ihre Konstruktionsfantasien nicht umzusetzen vermag. Vielleicht ist hierin aber auch die Hartnäckigkeit des kollektiven Imaginären zu spüren, das heißt die Schwierigkeit, kollektiv geteilte Vorstellungen einfach umzuwerten. Den Versuch einer solchen Umwertung im Bereich der bildenden Kunst stellt eine Statue an der University of Illinois in Urbana-Champaign dar. Sie zeigt die Figur einer Ingenieurin sowie darunter in Stein gemeißelt die für den Ingenieur als Hoffnungsfigur typischen Charakteristika: »pioneering, innovative, passionate!« \textendash{} eine Intervention im öffentlichen Raum, die darauf abzielt, dem Ingenieur \textendash{} in Referenz auf Hausmann gesprochen \textendash{} ein feminines Gesicht zu geben.\footnote{Ein Bild dieser Bronze-Statue ist u.a. auf der Website der Künstlerin, \texttt{Julie~Rotblatt-Amrany}, zu sehen. Vgl.: \cite[][]{rotblatt-amrany-a}, \url{https://rotblattamrany.com}.} 
\subsection[Der Ingenieur in der Wüste:    Vom technischen Großprojekt    zur Krise]{Der Ingenieur in der Wüste: \- \protect\\ Vom technischen Großprojekt \- \protect\\ zur Krise}
\par In Müllers Hamlet-Ingenieur, in Kubricks Karikatur Wernher von Brauns sowie in der literarischen Kritik an einer geschlechtlich-einseitigen Codierung der Ingenieurfigur werden von verschiedenen Seiten aus Problematisierungen des Ingenieurs als Hoffnungsträger sichtbar, der \textendash{} das gilt es nicht zu vergessen \textendash{} in der \emph{pulp fiction} des Kalten Krieges, auf politischen Plakaten und in Fernsehsendungen als solcher intakt geblieben ist. Als prägnantester Ausdruck dafür, dass die Ingenieurfigur neben einem Hoffnungsträger immer stärker zu einer Krisenfigur wird, sei der Roman eines ausgebildeten Architekten \textendash{} seit jeher Konkurrenzdisziplin zum Ingenieurwesen \textendash{} ins Licht gerückt: Max Frischs Evergreen \emph{Homo Faber} (1957).\par Dieser mehrfach verfilmte Romanbestseller mit dem nüchternen Untertitel \emph{Ein Bericht} nutzt unter anderem eine Poetik des Raums, um an kollektiv gefestigten Bildern vom Ingenieur zu rütteln: »Ringsum nichts als Agaven, Sand, die rötlichen Gebirge in der Ferne, ferner als man vorher geschätzt hat, vor allem Sand und nochmals Sand, gelblich, das Flimmern der heißen Luft darüber, Luft wie flüssiges Glas.«\footnote{\cite[][]{frisch1977a}, S. 21.}  Diese Sätze stehen gleich zu Beginn von Frischs Erfolgsroman. Die Hauptfigur, der Ingenieur Walter Faber, befindet sich in der Wüste, nachdem er mit einem Passagierflugzeug, das von New York nach Mexico City hätte fliegen sollen, notlanden musste. Dabei wird die Technik von Frisch schon auf den Seiten davor in ein unheimliches Licht getaucht. Dem eigenen Absturz ist die Lektüre von einer Flugzeugkatastrophe vorangestellt:\begin{quote}
\par {[}...{]} und was mich {[}Walter Faber; RL{]} nervös machte, so daß ich nicht sogleich schlief, war nicht die Zeitung, die unsere Stewardeß verteilte, \emph{First Pictures Of World}\textquoteright{}\emph{s Greatest Air Crash In Nevada}, eine Neuigkeit, die ich schon am Mittag gelesen hatte, sondern einzig und allein diese Vibration in der stehenden Maschine mit laufenden Motoren \textendash{}\footnote{\cite[][]{frisch1977a}, S. 7.} 
\end{quote}
\par Fabers Notlandung in der Wüste ist unschwer als umfassende Bankrotterklärung einer technischen Weltanschauung zu erkennen: Hier hat nicht nur die Technik \emph{eines} Flugzeugs versagt, vielmehr steht der in die Wüste geworfene, nach Orientierung suchende Homo Faber für die Infragestellung eines ganzen Weltbilds.\footnote{Mit Blick auf die geschlechtliche Gebundenheit der Ingenieurfigur sei erwähnt, dass die Qualitäten Walter Fabers in der Romanarchitektur als männliche Qualitäten ausgewiesen sind. Exemplarisch hierfür eine Szene, in der Faber seiner Freundin der Technik gegenüber Verständnislosigkeit vorwirft: »›Walter‹, sagte sie, ›I\textquoteright{}m waiting.‹/ Als hätte unsereiner noch nie gewartet./ ›Technology!‹ sagte sie \textendash{} nicht nur verständnislos, wie ich\textquoteright{}s von Frauen gewohnt bin, sondern geradezu spöttisch, was mich nicht hinderte, das Apparätchen vollkommen zu zerlegen; ich wollte wissen, was los ist.« Vgl.: \cite[][]{frisch1977a}, S. 63.}  Denn überaus dezidiert ist Walter Faber \textendash{} »Ich mache mir nichts aus Romanen«\footnote{\cite[][]{frisch1977a}, S. 15. Vgl. auch S. 39, S. 76\textendash{}78 und S. 107.}  \textendash{} als Vertreter des einen, und zwar nur des einen Weltbilds gezeigt.\par Aber es ist, um es noch einmal deutlich zu sagen, der Ort der Wüste, anhand der sich eine Verschiebung in der Physiognomie der Ingenieurfigur zwischen Zwischenkriegszeit und Kaltem Krieg beobachten lässt: In der Imaginationsgeschichte des Ingenieurs spielt dieser Schauplatz nämlich eine nicht unwichtige Rolle. Zwischen den Weltkriegen florieren Erzählungen, in denen Ingenieure wilde Einöden in Orte des Fortschritts verwandeln. Für dieses Muster exemplarisch sei der Roman \emph{Elektropolis} (1928) von Otfrid von Hanstein genannt, in dem eine solche Tabula-rasa-Fantasie im Zentrum steht. Wildnis wird hier von einem schöpferischen Einzelnen in Lebensraum, die Stadt Elektropolis verwandelt.\footnote{\cite[][]{hanstein1928a}.} \par Walter Faber hingegen, in der Wüste nahe der Sierra Madre Oriental . . . Diese Lage ist nicht Ausgangspunkt für die Erschaffung einer neuen Welt, sondern Ausdruck eines Dilemmas. Einen Neubeginn gibt es bei Frisch nur noch im Innenleben der Figur: Ihr ›Schiffbruch‹, traditionell ein Motiv, in dem sich Katastrophe und Neubeginn verschränken, ist Anlass, das eigene, nunmehr 50-jährige Leben neu zu überdenken, die Geschichte und Entwicklung der eigenen emotionalen Bindungen zu verstehen, jedoch nicht um technische Großleistungen zu erbringen. Indem Frisch diesen Ingenieur im sprichwörtlichen Sinne ›in die Wüste schickt‹, setzt auch er die \emph{Imago} des gesellschaftlichen Hoffnungsträgers unter Druck. Er lässt seine Ingenieurfigur nicht mehr konstruieren, sondern nach Neuorientierung suchen.
\paragraph[\textemdash{}]{\textemdash{}}
\par Im März 2019 wurde mir, als ich im Café der Tel Aviv University saß, die folgende Geschichte erzählt: In der heute nordisraelischen, vormals jordanischen Stadt Naharija, von jüdischen Einwanderern aus Deutschland gegründet, wurde zwischen 1929 und 1932 unter der Führung von Pinhas Rutenberg ein Kraftwerk gebaut. Palästina stand damals noch unter britischer Herrschaft, und Rutenbergs Großprojekt war, wie ich später in der Universitätsbibliothek nachlesen konnte, Teil eines von Winston Churchill erteilten Mandats, Palästina zu elektrifizieren.\footnote{\cite[][]{shamir2017a}.}  Bei den Bauarbeiten an diesem Kraftwerk kamen vier Männer ums Leben, darunter Avraham Kegerlitzky, ein Ingenieur. Für den Verunglückten und einige seiner Kumpel wurde in Naharija ein Denkmal errichtet, für das Chaim Nachman Bialik, Israels Nationaldichter, die Verse schrieb. Mehr als sechzig Jahre später, 1994, nach der Unterzeichnung des Friedensabkommens zwischen Israel und Jordanien, wurde der Leichnam des verunglückten Ingenieurs auf israelischen Boden überführt. 2001 wurde er in einer öffentlichen Zeremonie auf dem Friedhof des Kibbutz Gesher beigesetzt. Man hatte den Ingenieur an ›seinen Ort‹ zurückgebracht.%
	%
%
	%
		%
			%
			%
			%
				\selectlanguage{ngerman}%
			%
			%
				%
					
						\chapter[head={Objekte des Ingenieurs}, tocentry={OBJEKTE DES INGENIEURS}]{Objekte des Ingenieurs}%
					
					\-
					\par			
					\newpage%
					\noindent%
				%
			%
			%
				%
					\regularfontdefault%
				%
			%
		%
	%
	%
	
\subsection[Raketen, Elektroautos,    Tunnel\-systeme: Echos~auf den Ingenieur    als großen~Mann]{Raketen, Elektroautos, \- \protect\\ Tunnel\-systeme: Echos~auf den Ingenieur\- \protect\\  als großen~Mann}
\par\noindent In einem Interview mit dem einschlägigen Titel \emph{The Future We}\textquoteright{}\emph{re Building \textendash{} And Boring} (2017) wird der in Südafrika geborene Elon Musk, ebenso oft als Unternehmer wie als Physiker, als Designer wie als Ingenieur bezeichnet, auf sein Unternehmen \emph{SpaceX} angesprochen. Musk skizziert das 2002 ins Leben gerufene Projekt als eine Möglichkeit, um die Menschheit in eine multiplanetarische Spezies (»a multi planetary species«) zu verwandeln.\footnote{\cite[][]{musk2017a}, 34:54.}  Das Vorhaben, das also darauf abzielt, den menschlichen Lebensraum ins Weltall zu erweitern, erinnert stark an literarische Kolonisationsfantasien der Zwischenkriegszeit, in denen meist Ingenieure in noch ungenutztem Lebensraum, Teilen Afrikas oder Australiens, Siedlungen errichten.\footnote{\cite[][]{hahnemann2010a}.}  Dass solche Ideen in öffentlichen Debatten um 2000 neu auftauchen, lässt den Eindruck entstehen, dass der Ingenieur in seiner Variante als megalomaner Weltverwandler, wie ihn Max Frisch 1957 verabschiedet hatte, wiederkehrt.\par An der Darstellung eines anderen von Musks Unternehmen, das seinerseits um das Problem der Verengung von Lebensraum kreist, ist noch viel deutlicher abzulesen, wie Vorstellungen vom Ingenieur als großem Mann, wie wir sie aus dem frühen 20. Jahrhundert kennen, heute fast ungebrochen wiederkommen. Bis in einzelne Details erinnern Beschreibungen von Musks \emph{Hyperloop}-Projekt, 2014 initiiert, an eine paradigmatische Ingenieurfiktion des frühen 20. Jahrhunderts, Bernhard Kellermanns Roman eines transatlantischen Tunnels. Verblüffend nahe an dem Vorhaben des fiktiven Ingenieurhelden Mac Allan heißt es über \emph{Hyperloop}: »Billed as a new mode of transportation, this machine was a large-scale pneumatic tube like the ones used to send mail around offices. Musk proposed linking cities like Los Angeles and San Francisco via an elevated version of this kind of tube that would transport people in cars and pods.«\footnote{\cite[][]{vance2015a}, S. 336.}  Während man für den Weg von Westwood, Kalifornien, zum Flughafen von LA heute ca. 16 Minuten brauche, würde Musks Tunnelsystem, das Autos in Hochgeschwindigkeit manövriert, es ermöglichen, ihn in nur 5\textendash{}6 Minuten zurückzulegen.\footnote{\cite[][]{musk2017a}, 01:58.} \par Aber nicht nur in Unternehmungen wie \emph{SpaceX} oder \emph{Hyperloop}, sondern auch in der Art und Weise, in der ihr Urheber gezeichnet ist, lassen sich Echos auf Ingenieurfiguren nach dem Muster von Kellermanns Mac Allan vernehmen. Ashlee Vance, ein Wirtschaftsjournalist und Biograf Elon Musks, beginnt seine Erzählung von dessen Leben so:\begin{quote}
\par Musk had struck me as a well-intentioned dreamer \textendash{} a card-carrying member of Silicon Valleys techno-utopian club. This group tends to be a mix of Ayn Rand devotees and engineer absolutists who see their hyperlogical worldviews as the an\-swer for everyone. If we would just get out of their way, they\textquoteright{}d fix all our problems.\footnote{\cite[][]{vance2015a}, S. 5.} 
\end{quote}
\par Seine Biografie, \emph{Elon Musk: Tesla, SpaceX}, \emph{and the Quest for a Fantastic Future}, liest sich dann wie ein langsames Bekenntnis: Der Biograf habe sich getäuscht, der von der Öffentlichkeit so gesehene Musk sei aufgrund seiner Exzellenz tatsächlich dazu in der Lage, unsere Welt zu verändern. Es sei mehr als ein Image. Denn im Gegensatz zu vielen Unternehmern des Silicon Valley, so Vance, denen es nur um den ökonomischen Erfolg ginge, wolle Musk unsere Gesellschaft besser machen: »What Musk has developed that so many of the entrepreneurs in Silicon Valley lack is a meaningful world view. He\textquoteright{}s the possessed genius on the grandest quest anyone has ever concocted.«\footnote{\cite[][]{vance2015a}, S. 17.}  Entscheidend die Wendungen: \emph{possessed~genius}~und \emph{meaningful world view}, also, frei übersetzt, ein von seiner Sache besessenes Genie, jemand, dessen Perspektive auf die Welt bedeutsam sei \textendash{} zwei Formulierungen, die ohne Zweifel als Versatzstücke des Ingenieurs als großer Mann gelten können.\footnote{Einschlägig auch der Hinweis, dass Musk, den Vance wiederholt mit Henry Ford (S. 15, 152) und Thomas Alva Edison (S. 21) vergleicht, von seinen Mitarbeitern das Unmögliche verlange, nur um sie dann eines Besseren zu belehren, dass nämlich das unmöglich Scheinende tatsächlich möglich sei (S. 152). Denken wir zurück an die bereits zitierten Worte Mac Allans in \emph{Transatlantic Tunnel}: »You may find it a bit fantastic, but I know it can be done.« Noch ein anderes Detail unterstreicht das Bild des grenzüberschreitenden Genies, als das Musk hier gezeichnet ist. Man habe ihn zwar zum Ingenieurstudium an der prestigereichen Stanford University zugelassen, doch habe er sein Studium dort nicht abgeschlossen (S. 369). Ein Detail, zugegeben, jedoch ein überaus sprechendes. Denn Musk wird als Ingenieur gezeigt, der einer Ausbildungsinstitution gar nicht bedarf. Auch die Ingenieure vor dem Ersten Weltkrieg haben teilweise in Opposition zu den Werten des Bildungsbürgertums (vgl. den Exkurs über die Ingenieurin in Abschnitt III dieses Essays) ihre Distanz zur akademischen Institution betont. Die Seitenangaben in dieser Fußnote beziehen sich auf: \cite[][]{vance2015a}.} \par Sehen wir also genauer hin, erkennen wir, dass in unserer Gegenwart eine medial erzeugte, öffentliche Figur aufgetaucht ist, in deren Gesichtszügen \textendash{} denken wir an Hausmanns Gemälde \textendash{} solche des Ingenieurs als großer Mann, wie er sich in den Vorstellungswelten der Zwischenkriegszeit tummelte, wiederkehren, und in der sich, sei es in Fremd- oder Selbstbeschreibungen, kollektiv geteilte Hoffnungen und Wünsche unserer Zeit verdichten.\par Noch ein anderes Phänomen wird an dieser öffentlichen Figur greifbar: Achten wir darauf, mit welchen Objekten Elon Musk auf den zahlreichen Bildern seiner Person gezeigt wird, werden wir sehen, dass es sich dabei in allererster Linie um physische Artefakte handelt, elektrische Autos, Tunnelsysteme, Raketen; fertige, oftmals zum Verkauf stehende Waren. Im Rahmen einer Imaginationsgeschichte des Ingenieurs betrachtet, steht diesen Fabrikaten eine andere Gruppe von Objekten gegenüber: die Werkzeuge, die von Ingenieuren dafür benutzt werden, um Betrachtungen anzustellen, Reflexionen zu entfalten. Zu denken ist bspw. an die Ideenskizze, das Modell oder Reißbrett, Objekte, die keine Artikel für den lebensweltlichen Gebrauch abgeben, sondern mit denen \texttt{gedacht}, ja Erkenntnisse gewonnen werden, die möglicherweise einem später fertigen Artefakt zugutekommen.\footnote{Über die Möglichkeiten, mit Modellen Einsichten am Kleinen für das Große zu gewinnen, schreibt \cite[][]{sattelmacher2021a}, S. 9.} \par Idealtypisch gesprochen, lässt sich sagen, dass die erste Gruppe von Objekten, die Artefakte, den Ingenieur mehr als eine Figur der Produktion zeigt, im Sinne eines Herstellens oder Zustandebringens physischer Artefakte; während ihn die zweite, jene der Werkzeuge, stärker als eine Figur der Reflexion erkennen lässt, im Sinne des Hervorbringens von Einsichten, Entwürfen, Erkenntnissen. Mit den jeweiligen Objekten des Ingenieurs, die ich streng analytisch in Artefakte und Werkzeuge unterscheiden möchte, fällt also ein je unterschiedliches Licht auf die Figur des Ingenieurs. Ähnlich wie seine Proportionen (groß/klein) oder Orte (Konstruktion/Destruktion) lassen auch die Objekte des Ingenieurs verschiedene Gesichter an ihm hervortreten: den Ingenieur als eine Figur der Produktion physischer Artefakte auf der einen Seite; als eine der Reflexion, die in einem Prozess der Betrachtung Einsichten gewinnt, auf der anderen.\par Mit dieser Unterscheidung, die eine weitere Spannung innerhalb der Ingenieurfigur sichtbar macht, jene zwischen Produktion und Reflexion, tritt neben die Leitfrage dieses Abschnitts nach der Wiederkehr des Ingenieurs als Hoffnungsträger unserer Zeit, eine weitere, nämlich die, inwiefern verschiedene Objekte den Ingenieur als eine Figur der Produktion bzw. Reflexion exponieren?\par Auf die uns hier interessierende Zeit von 1900 bis in unsere Gegenwart blickend, möchte ich behaupten, dass der Ingenieur im kollektiven Imaginären als eine Figur der Produktion insgesamt weitaus präsenter ist als eine der Reflexion. Es dominiert, anders gesprochen, das Gesicht des Ingenieurs als eines Erzeugers physischer gegenüber geistiger Hervorbringungen. Diese Asymmetrie, so eine These dieses Abschnitts, hat sich nach 2000 \textendash{} und hierfür steht das Phänomen Elon Musk, ein Produktingenieur par excellence und smarter Verkäufer von Ingenieurprodukten \textendash{} weiter verstärkt. Da diese Verschiebung in der \texttt{Physiognomie~unserer} \texttt{Figur~nirgendwo} so gut greifbar wird wie an ihren Objekten, sol\-len sie den Dreh- und Angelpunkt dieses IV. und letzten Abschnitts \texttt{bilden}.
\subsection[I-Products: Der Ingenieur als    Künstler und gesellschaftlicher    Hoffnungsträger]{I-Products: Der Ingenieur als \- \protect\\ Künstler und gesellschaftlicher\- \protect\\  Hoffnungsträger}
\par Deutlicher noch als im Falle Elon Musks ist um 2000 die weitere Verlagerung zum Produkt an einem anderen Ort des Silicon Valley zu sehen. Die Rede ist von jenem Mann, der dem Ingenieur noch 1996, in einer in der Einleitung dieses Essays zitierten Passage die Fähigkeit zuspricht, die Probleme der Menschheit zu lösen, Steve Jobs, globale Ikone der Computerindustrie, kein Ingenieur im engeren Sinne, aber eine medial erzeugte Figur, in deren Physiognomie wir Züge des Ingenieurs als gesellschaftlichen Hoffnungsträger wiedererkennen können. Zunächst zwei Aussagen, eine Selbst- und eine Fremdbeschreibung, dieses Mannes:\label{JOBS}\begin{quote}
\par I {[}Steve Jobs; RL{]} think great artists and great engineers are similar, in that they both have a desire to express themselves. In fact some of the best people working on the original Mac were poets and musicians on the side. {[}...{]} Great artists like Leonardo da Vinci and Michelangelo were also great at science. Michelangelo knew a lot about how to quarry stone, not just how to be a sculptor.\footnote{Jobs, zitiert nach \cite[][]{isaacson2011a}, S. 521.} 
\end{quote}
\par Und einer seiner Biografen, Walter Isaacson, sagt über ihn:\begin{quote}
\par History will place him in the pantheon right next to Edison and Ford. More than anyone else of his time, he made products that were completely innovative, combining the power of poetry and processors. {[}...{]} And he was able to infuse into its {[}the company\textquoteright{}s; RL{]} DNA the design sensibilities, perfectionism, and imagation that make it likely to be, even decades from now, the company that thrives best at the intersection of artistry and technology.\footnote{\cite[][]{isaacson2011a}, S. 520.} 
\end{quote}
\par Die Objekte Steve Jobs\textquoteright{}, gemessen an denen von Musk kleinformatig, oft unscheinbar, werden in diesen Aussagen wiederholt in einen Bereich zwischen Technik und Kunst gerückt. Die Entgegensetzungen Künstler/Ingenieure, Poesie/Prozessoren, Kunstfertigkeit/Technologie erinnern etymologisch betrachtet daran, dass in dem griechischen Wortstamm von Technik, der ganz allgemein ein sachgemäßes Können bezeichnet, die Bereiche von Kunst- und Handwerk noch nicht voneinander getrennt sind.\footnote{\cite[][]{technik1998a}, hier S. 940.}  Und tatsächlich: Egal ob Macintosh, iPod oder iPhone, jedes dieser Produkte konnte nicht nur aufgrund seiner Funktionalität, sondern auch wegen seiner ästhetischen Gestalt Karriere machen. Die bewusst gearbeitete Form all dieser Objekte erklärt sich daraus, dass Jobs schon Mitte der 1980er-Jahre einen Grafikdesigner, Paul Rand, für das Logo der Neugründung NeXT engagierte und später den unter anderem von der Bauhaus-Ästhetik inspirierten Jony Ive damit beauftragte, zuerst den iPod, später das iPhone zu gestalten.\footnote{Nach Isaacson erfolgte die Herstellung all der genannten Produkte in enger Zusammenarbeit von Ingenieuren und Designern. Exemplarisch hierfür ist das Kapitel 26 seiner Biografie. Vgl.: \cite[][]{isaacson2011a}, S. 312\textendash{}319.}  Über den iPod heißt es in Jobs\textquoteright{} Biografie, die schon zitierte Synthese von Kunst und Technik wiederaufnehmend: »{[}...{]} the I-Pod became the essence of everything, Apple was destined to be: poetry connected to engineering, arts and creativity intersecting with technology.«\footnote{\cite[][]{isaacson2011a}, S. 361.}  Die Besonderheit dieser Produkte, nicht nur mit Blick auf ihre technische Leistung, sondern bewusst als Objekte ästhetischen Erlebens konzipiert zu sein, lässt das Gesicht eines Künstleringenieurs hervortreten, das zunächst anders scheint als jenes des gesellschaftlichen Hoffnungsträgers, mit diesem aber, wie wir gleich sehen werden, durchaus kompatibel ist.\par Die hier zur Debatte stehenden Artefakte und damit der für ihre Herstellung verantwortlich zeichnende Steve Jobs erheben neben einem ästhetischen nämlich noch einen sozialen Anspruch: den einer Verbesserung der Gesellschaft qua Technik, wenn auch in einer spezifischen Form. Richten wir, um diese Dimension genauer zu erfassen, die Aufmerksamkeit auf die Produktvorführung des Macintosh im Jahr 1984, an dem dieser Anspruch besonders gut sichtbar wird:\begin{quote}
\par On January 24 1984\- \protect\\  Apple Computer will introduce\- \protect\\  Macintosh.\- \protect\\  And you\textquoteright{}ll see why 1984\- \protect\\  won\textquoteright{}t be like 1984.\footnote{\cite[][]{isaacson2011a}, S. 184. Für die Produktvorführung des Macintosh vgl.: \cite[][]{apple1984a}, online auf \url{https://www.youtube.com/watch?v=2zfqw8nhUwA}.} 
\end{quote}
\par Dieses Zitat, das während der Superbowl am 22. Januar 1984 über Millionen US-amerikanischer Bildschirme flimmerte, nimmt auf einen literarischen Text Bezug, der auch nicht akademischen Leser*innen zumindest in seinen Umrissen geläufig ist, George Orwells paradigmatisch dystopischen Roman \emph{Nineteen Eighty-Four} (1949).\par Es handelt sich um eine Negativutopie, in deren Gesellschaft des Jahres 1984 technische Fabrikate als Mittel der Unterdrückung dienen. Demgegenüber setzt diese Werbung eine Vorstellung von technischen Artefakten als Instrumenten der Autonomie. Dieser Gegensatz wird in dem dreiminütigen Video, das dem Zitat vorangeht, sowohl szenisch als auch farblich vorbereitet: Eine in bunten Farben gekleidete junge Frau, die mit einem Hammerwurf einen Monitor zertrümmert, steht einer Masse von einförmig grau gekleideten Menschen gegenüber. Dem Literaturhistoriker sei die Bemerkung erlaubt, dass die Wahl einer jungen Frau durchaus in der Logik von Orwells Text, mehr noch in jener der dystopischen Tradition überhaupt steht. Bei Orwell ist es nämlich eine weibliche Figur, Julia, die Winston, den Protagonisten aus \emph{Nineteen Eighty-Four,} vom Terror des herrschenden Systems und der besseren Sache überzeugen möchte; und in dem schon behandelten Roman \emph{Wir} von Jewgenij Samjatin, nachweislich Orwells Vorläufer, ist es ebenfalls eine Frau, I-330, welche die Hauptfigur, D-503, Handlanger des Regimes, auf die bessere Seite ziehen möchte. Es ist diese für Dystopien typische Konstellation einer männlich konnotierten Technikdiktatur, der \emph{eine} Einzelne Nadelstiche versetzt (sie kehrt auch in Ray Bradburys \emph{Fahrenheit 451} von 1953 in der Liebe zwischen Guy Montag und Clarisse wieder), die dieser Werbespot aufnimmt. Erich Fromm hat in seinem Nachwort zu Orwells Klassiker von einer Welt geschrieben: »in which man builds machines which act like men and develops men who act like machines {[}...{]}.«\footnote{\cite[][]{fromm2003a}, hier S. 337. Vgl. (ebenfalls auf S. 337): »{[}S{]}o 1984 teaches us, the danger with which all men are confronted today, the danger of a society of automatons who will have lost every trace of individuality, of love, of critical thought, and yet who will not be aware of it.«}  Gegen diese Schreckensvision setzen die Werber von Apple ein Produkt, das gesellschaftliche Befreiung und ästhetisches Erleben verspricht und eine dezidiert weibliche Kontur zeigt. Die Feminisierung dieses Versprechens ist bemerkenswert vor der Folie der zuvor genannten Schwierigkeiten, die Figur der Ingenieurin im kollektiven Imaginären zu etablieren. Man wird die Beobachtung formulieren können, dass weibliche Figuren den geschlechtlich gebundenen Ingenieurmythos in den Massenmedien zwar nur teilweise umwerten können, zumindest aber punktuell wie hier, als Personifikationen alternativer technischer Artefakte funktionieren.\par Für die hier aufgeworfene Frage nach \texttt{der~Wiederkehr}~\texttt{ge\-wisser} Bausteine des Ingenieurs als \texttt{gesellschaftlicher} Hoff\-nungs\-träger ist entscheidend, dass das in Selbst- und Fremdbeschreibungen dominierende Gesicht von Steve Jobs als eine Art Künstleringenieur von einer anderen, uns gut bekannten Ingenieurphysiognomie überblendet wird, dem des \texttt{gesellschaftlichen} Hoffnungsträgers. Entscheidend: Es sind die Artefakte dieses Mannes, die gesellschaftliche Emanzipation verheißen. Wenn Jobs\textquoteright{} Biograf ihn nicht überraschend einen »Prometheus Unbound« nennt,\footnote{Vgl. \cite[][]{isaacson2011a}, Kapitel 18.}  bezieht er sich darauf, dass Prometheus, wiederkehrender Vergleichspunkt für den Gesellschaftsingenieur der Zwischenkriegszeit, der Menschheit das Feuer brachte, damit diese sich aus ihrer Abhängigkeit von den Göttern löse. Zu Beginn des 21. Jahrhunderts, so die streitbare Suggestion dieses Vergleichs, seien es die Fabrikate Steve Jobs\textquoteright{}, denen das Potenzial zukommt, uns zur Freiheit zu führen. Diese Freiheit aber ist nicht wie beispielsweise die von Neuraths Gesellschaftstechniker verheißene eine des Kollektivs, sondern eine des Einzelnen, libertär und antietatistisch.
\subsection[Entwürfe, Tabellen, Notizen:    Der Ingenieur als~eine Figur    der Reflexion]{Entwürfe, Tabellen, Notizen: \- \protect\\ Der Ingenieur als~eine Figur \- \protect\\ der Reflexion}
\par Halten wir vorläufig zwei Beobachtungen fest: In biografischen sowie massenmedialen Inszenierungen von Elon Musk und \texttt{Steve} Jobs kehrt um die Jahrtausendwende das Versprechen des Ingenieurs als gesellschaftlicher Hoffnungsträger wieder. Hier wie dort werden Bausteine dieser \emph{Imago} aufgenommen und im Kontext moderner Massenmedien (Websites, Talkshows, Videoclips) sowie der populären Biografik neu aufgeladen. Noch ein zweiter Punkt ist aber entscheidend: Beide Figuren stehen für eine sich zuspitzende Asymmetrie in der Imaginationsgeschichte des Ingenieurs, eine Dominanz des Ingenieurs als eine Figur der Produktion gegenüber jener der Reflexion, wie sie sich anhand des Übergewichts von physischen Artefakten und des in den Hintergrund Tretens von Werkzeugen manifestiert. Um diesen zweiten Punkt nun weiter zu differenzieren, ist es wichtig, sich noch einmal klarzumachen, dass der Ingenieur nicht erst seit der Jahrtausendwende, sondern seit seiner ständischen Profilierung im frühen \texttt{20.~Jahrhundert} enger mit der Vorstellung eines praktischen Tuns als mit der von Reflexion verbunden war. Eine Passage aus Steven Shapins Monografie \emph{Scientific Life. A Moral History of Late Modern Vocation} (2008) legt diese Einsicht nahe:\begin{quote}
\par The laity might imagine that the scientists had simply inherited the cultural authority of the priests, but many scientists circulated views of scientific knowledge very different from priestly notions of Truth. The scientist was properly to be understood not on the model of the philosopher but on the model of the engineer and technician. The active was to replace the contemplative; technique (an attempt to control) was to replace speculation (an attempt to understand and to tell the Truth about the world).\footnote{\cite[][]{shapin2008a}, S. 30\textendash{}31.} 
\end{quote}
\par Es geht in diesem Zitat \textendash{} wohlgemerkt \textendash{} um Veränderungen in der Vorstellung vom Wissenschaftler, nicht vom Ingenieur. Shapin behauptet, dieser wäre zu Beginn des 20. Jahrhunderts stärker am Techniker als am Philosophen oder Priester ausgerichtet gewesen, mehr an der Aktion als der Kontemplation. Entscheidend für unseren Zusammenhang: Der Ingenieur wird quasi diskussionslos auf die Seite der Aktion gerückt und jener der Kontemplation gegenübergestellt. Doch der Umstand, dass der Ingenieur in der Geschichte des kollektiven Imaginären als Figur der Reflexion weniger deutlich ausgeprägt ist als sein Counterpart, kann nicht darüber hinwegtäuschen, dass es ihn als eine solche doch gibt und immer schon gab. Denken wir an uns bekannte bildnerische und literarische Quellen: den an einem Reißbrett gezeigten Ingenieur aus Ernst Fischers Erzählung \emph{Das Irrenhaus}, der als eine karikaturistische Variante des Ingenieurs als eine Figur der Reflexion gelesen werden kann. Seine in sich kreisenden Reflexionen bringen eine Technik als Technik hervor und gerade nicht ein Artefakt für das Gemeinwohl; oder denken wie an die mit Lineal, Reißbrett und Skizzenplan gezeigten Ingenieure auf Hausmanns Gemälde, die sich umschauen, die betrachten und vermessen, jedoch keine Produkte herstellen.\footnote{Zu denken ist auch an den Perspektivplan in Volker Brauns \emph{Tinka}. Vgl.: \cite[][]{braun1990a}, S. 174.}  Der reflektierende Ingenieur \textendash{} denken wir auch an Heiner Müllers Hamlet-Ingenieur, der nachdenkt \emph{anstatt} zu produzieren \textendash{} es gibt ihn, doch ist er in der Imaginationsgeschichte dieser Figur weit weniger häufig. Die Frage, die sich uns an diesem Punkt also stellt: Wo lässt sich heute, in der Ära der Produktingenieure noch das Gesicht des Ingenieurs als einer Figur der Reflexion erkennen? In welchen Diskursen ist von ihm und seinen Werkzeugen eigentlich noch die Rede?\par Es ist die Literatur, in der wir im Zeitalter von Elon Musk und Steve Jobs die \emph{Imago} eines solchen Ingenieurs aufbewahrt finden: Walter Kappacher, einer der unauffälligsten, medienscheusten und nicht nur für den Nobelpreisträger von 2019, Peter Handke, einzigartigsten Prosaschriftsteller unserer Zeit,\footnote{\cite[][]{handke2007a}.}  hat im Jahr 2000 mit dem Roman \emph{Silberpfeile} die faszinierende Geschichte eines Ingenieurs vorgelegt. Es sind, genauer gesagt, zwei Geschichten, die uns hier vermittelt werden: Ein Icherzähler, Rennsportjournalist, dessen Lebensgefährtin auf unbestimmte Zeit verreist ist, möchte die Geschichte von Paul Windisch, einem in die Jahre gekommenen Ingenieur rekonstruieren. Dieser hat in der Zwischenkriegszeit zunächst an der Konstruktion von Rennautos mitgewirkt und zu Kriegsende im österreichischen Zwangsarbeiterlager Schlier für das NS-Regime am Bau der V2-Rakete gearbeitet. In dieser durch die Brille eines Journalisten vermittelten Geschichte eines Ingenieurlebens im 20. Jahrhundert wird das Bild eines vergangenen, heute weitgehend verblassten Gesichts des Ingenieurs evoziert:\begin{quote}
\par Gewaltige Brände hatten anscheinend alles in Schutt und Asche gelegt; von meinen Papieren, Konstruktionszeichnungen, Tabellen, Notizen würde vermutlich nichts mehr übrig sein.\footnote{\cite[][]{kappacher2000a}, S. 184.} 
\end{quote}
\par Ein kurzes, aber für diese Figur prägnantes Zitat, in dem Kappachers Ingenieur dem Journalisten von den Verheerungen des Zweiten Weltkrieges erzählt. »{[}A{]}lles«, das ist für ihn nicht ein physisches Fabrikat wie ein Auto oder eine Rakete. »{[}A{]}lles«, das sind für diesen Ingenieur seine Werkzeuge, Papiere, Konstruktionszeichnungen, Tabellen und Notizen. Überhaupt lesen sich die \emph{Silberpfeile} stellenweise wie eine Enzyklopädie von Ingenieurwerkzeugen \textendash{} ohne Anspruch auf Vollständigkeit ist dort die Rede von Ideenskizzen und Konstruktionsplänen, Entwürfen und Zeichenbrettern, Reißbrettern und Modellen \textendash{}, währenddessen die Produkte, zu deren Herstellung mit ihnen Einsichten gewonnen werden, im Hintergrund bleiben.\footnote{Vgl.: \cite[][]{kappacher2000a}, S. 21; 69, 149\textendash{}150; 200; 56, 71, 208, 62; 58, 78. Für diese Seite der Figur einschlägig sind auch die Modellversuche und die Probefahrten. Vgl.: \cite[][]{kappacher2000a}, S. 65 und S. 91.}  Indem uns Kappacher seinen Ingenieur mit diesen Objekten hantierend und von ihnen erzählend zeigt, entsteht vor unseren Augen das Bild eines Ingenieurs, der entwirft, ausprobiert und überdenkt.\footnote{Sein Gegenpol, der Ingenieur als Praktiker und Mann der Tat, taucht nur am Rande auf. An einer Stelle der \emph{Silberpfeile} wird beinahe beiläufig Kellermanns \emph{Der Tunnel} zitiert, dessen Ingenieur, Mac Allan, als eine Gegenfigur zu Windisch gesehen werden kann: »Er {[}Bernd Rosemeyer; RL{]} hat mir ein Buch geschenkt, dasselbe, das er auf dem Schiff nach New York immer bei sich, aber nicht gelesen hatte, er sagte, er sei nicht weit gekommen damit, er lese lieber Biografien. Es handelte sich um den Roman \emph{Der Tunnel} von Bernhard Kellermann, ein Bestseller.« Vgl.: \cite[][]{kappacher2000a}, S. 78.}  Die narrative Konstruktion des Romans, die das Leben dieses Ingenieurs der Vergangenheit zuordnet, bekräftigt die hier vertretene Beobachtung vom weitgehenden Verschwinden dieses Gesichts aus unserer Zeit.\par Blicken wir von hier aus auf das bisher Gesagte zurück, dann lässt sich die Behauptung aufstellen, dass Vorstellungen vom Ingenieur als Hersteller weltverändernder Fabrikate und großem Mann, wie sie die Trivialkultur der Zwischenkriegszeit produziert hat, bspw. in der Fiktion von Kellermanns \emph{Tunnel}, heute in die kommerzielle Biografik globaler Ikonen, Elon Musk oder Steve Jobs, eingegangen sind, während sich ein diesen Vorstellungen entgegengesetztes Bild vom Ingenieur, reflektierend, nachdenklich, in einem stillen Schreibprojekt wie jenem Walter Kappachers bewahrt hat. Das Spannungsverhältnis dieser beiden diskursiven Pole, populäre Biografie auf der einen Seite, Literatur im engeren Sinne auf der anderen, manifestiert sich auch darin, dass die Literatur um die Problemgeschichte des Ingenieurberufs, das heißt, seine Berührungspunkte mit den mörderischen Ideologien des 20. Jahrhunderts weiß, während die genannten Heldenepen unserer Tage diese überspielen. Nirgendwo wird das so gut greifbar wie an einem Objekt, Elon Musks Rakete, deren Vorgeschichte nicht, wie von ihm einmal behauptet, in der Antike liegt, sondern in der Kriegstechnik des NS-Regimes, von der uns Kappachers Ingenieur so eloquent und ausführlich zu berichten weiß.\footnote{Vgl.: \cite[][]{kappacher2000a}, S. 159; S. 160; S. 167; S. 182. Jobs und Musk holen, um die Vorgeschichte ihrer Produkte aufzuweisen, zu jeweils weiten historischen Rückgriffen aus; Musk auf die Antike (vgl. \cite[][]{musk2017a}, 37:23), Jobs auf die Renaissance (vgl. das Zitat hier auf S. \pageref{JOBS}, in dem Jobs auf Leonardo da Vinci zu sprechen kommt). In beiden Fällen wird die unmittelbare Vorgeschichte dieser Produkte übersprungen.} 
\subsection[Hefte, \emph{Cahiers} \textendash{} und eine Variante    des Ingenieurs als eine Figur    der Reflexion]{Hefte, \emph{Cahiers} \textendash{} und eine Variante \- \protect\\ des Ingenieurs als eine Figur \- \protect\\ der Reflexion}
\par Bleiben wir noch für einen Moment bei Kappachers unzeitgemäßem Ingenieurhelden. Es fügt sich bruchlos in sein Profil, dass \textendash{} und besonders wie \textendash{} er beim Erzählen seiner Lebensgeschichte auf Leonardo da Vinci zu sprechen kommt. Er sagt:\begin{quote}
\par Am meisten hatte mich ein ziemlich zerfledderter Band mit Aufzeichnungen von Leonardo da Vinci beeindruckt. Bis dahin war mir nicht bewußt gewesen, daß Leonardo auch ein großartiger Wissenschafter und Erfinder war, ein Ingenieur gewissermaßen ein Kollege; seine Mona Lisa hatte mir nie sonderlich imponiert, aber die Entwurfszeichnungen zu Geräten und Maschinen haben mich fasziniert.\footnote{\cite[][]{kappacher2000a}, S. 168\textendash{}169. Vgl. auch die folgende Passage: »Möglicherweise kam mir die Idee beim Betrachten der Entwurfszeichnungen Leonardos.« (S. 200).} 
\end{quote}
\par Es sind, kurz gesagt, Leonardos Entwürfe, die diesen fiktiven Ingenieur so begeistern. Diese Faszination verbindet ihn mit Paul Valéry, dessen Gedanken zu Leonardo da Vinci in der Zeit um \texttt{1900~von} demselben Gestus bestimmt sind: einem Interesse nicht an Leonardos abgeschlossenen Werken, sondern an seinen Entwürfen. Mit dem Ziel, eine weitere Variante des Ingenieurs als eine Figur der Reflexion zu profilieren, möchte ich nun einen Seiten- und historischen Rückblick auf Valérys Essays zu Leonardo werfen.\par Es war zwischen 1894 und 1929, als der französische Lyriker seine Aufsätze über Leonardo verfasste. Dreh- und Angelpunkt sind auch ihm da Vincis Hefte, Notiz- und Skizzenbücher, in den begrifflichen Registern dieses Abschnitts gesprochen, Werkzeuge, mit denen da Vinci technische Möglichkeiten durchgedacht, weiterentwickelt und wieder verworfen hat.\footnote{\cite[][]{popplow2007a}, S. 962.}  Über diese heißt es bei Valéry:\begin{quote}
\par Il {[}da Vinci; RL{]} laisse après soi, et comme dans l\textquoteright{}ombre de son œuvre peinte, un lot d\textquoteright{}étranges manuscrits qui se dispersent. {[}...{]} Ce sont des cahiers couverts d\textquoteright{}écriture et de croquis. Cette écriture est inversée ; il faut la lire par réflexion dans un miroir. Quant aux croquis, ils manifestent, avant toute lecture, une multitude de soucis et de recherches différentes mêlées. Figures géométriques, ou mécaniques, magnifiques dessins d\textquoteright{}anatomie de l\textquoteright{}homme ou du cheval, projets d\textquoteright{}architecture, ustensiles ou armes, personnages en action, ébauches de compositions, études des mouvements des fluides \textendash{} que sais-je !\footnote{\cite[][]{val1942a}, hier S. 10.} 
\end{quote}
\par Nach Valéry zeigen diese Hefte da Vinci in erster Linie als einen Denker; und zwar insofern, als sie Konstruktionen in der Vorstellung entwerfen, die erst Jahrhunderte später zu Artefakten werden:\begin{quote}
\par Enfin le moment vient que le problème de la machine à voler se propose, et puis s\textquoteright{}impose, à l\textquoteright{}ambition de l\textquoteright{}ingénieur moderne. {[}...{]} On trouve alors que les recherches merveilleusement prématurées de Léonard sont dans la voie de la solution scientifique du problème. La première machine volante qui ait volé, {[}...{]} porte la voilure même du type \emph{chauve-souris} que Léonard avait minutieusement étudiée et dont il a dessiné et calculé tous les éléments de construction.\footnote{\cite[][]{val1942a}, S. 11.} 
\end{quote}
\par Was Valéry an Leonardo fesselt, sind, materiell gesprochen, seine Hefte, geistesgeschichtlich gesagt, die an ihnen abzulesenden Entwicklungen seiner Gedanken, seine Methode. Entscheidend ist nun, dass Valérys Interesse daran in dem Umstand begründet ist, ausgehend von Leonardos Arbeitsweise ein Modell für seine eigene Arbeit zu finden. Die »Beschäftigungen des Künstlers, Wissenschaftlers, Ingenieurs, Schriftstellers {[}gemeint ist Leonardo da Vinci; RL{]}«, so Sabine Mainberger, »liefern Valéry das Muster für seine eigenen Betätigungen: {[}...{]} Statt Belagerungsmaschinen und Kanalsystemen sollen Gedichte und das Denken selbst mit vergleichbarer Methodik, Präzision und Strenge gebaut werden.«\footnote{\cite[][]{mainberger2009a}, hier S. 132\textendash{}133.} \par Diese Orientierung des Franzosen an Leonardo, also des Dichters am Ingenieur, lässt eine Figur entstehen, die von Bohnenkamp als \emph{Dichteringenieur} bezeichnet wird.\footnote{Vgl. Fußnote 2 dieses Essays.}  Damit ist eine Art von Dichter beschrieben, dessen Arbeitsweise an den Methoden (entwerfen, reflektieren, verwerfen), aber auch Werkzeugen (Hefte, Skizzen, Stifte) des Ingenieurs Maß nimmt. Für diese Verschmelzung bezeichnend ist besonders Valérys Rede vom »poète de l\textquoteright{}hypothèse«.\footnote{\cite[][]{val1931a}, hier S. 86; \cite[][]{val1942a}, S. 11.}  In ihm verschwimmen die Unterschiede zwischen den Werkzeugen des Dichters und jenen des Ingenieurs als einer Figur der Reflexion, Leonardos Heften und den \emph{Cahiers} Valérys.\footnote{In der Montagetechnik Sergej Eisensteins kommen sich die Methoden des Ingenieurs und jene des Künstlers in ähnlicher Weise nahe. Vgl.: \cite[][]{steiner1996a}.}  Wie gelingt es Leonardo, so könnte man diese Verbindung von Dichter und Ingenieur fassen, die Gesetze der Physik gedanklich aufzuheben, um zu fliegen? Und wie gelingt es Valéry, von Leonardo inspiriert, ästhetische Konventionen aufzulösen, um zu dichten?
\subsection[Computer, Spritze, Reagenzglas:    \emph{Human Engineering}    zwischen Theorie und Praxis]{Computer, Spritze, Reagenzglas: \- \protect\\ \emph{Human Engineering} \- \protect\\ zwischen Theorie und Praxis}
\par Ein Wort Valérys soll uns nun zu einer Leitfrage dieses IV. Abschnitts, jener nach der Wiederkehr des Ingenieurs als Hoffnungsträger unserer Zeit, zurückführen. In seinem zuletzt zitierten Essay schreibt der Franzose über den Prozess des Konstruierens: Dieser »existe entre un projet ou une vision déterminée, et les matériaux que l\textquoteright{}on a choisis. On substitue un ordre à un autre qui est initial, quels que soient les objets qu\textquoteright{}on ordonne. Ce sont des pierres, des couleurs, des mots, des concepts, des hommes, etc., {[}\dots{}{]}«.\footnote{\cite[][]{val1931a}, S. 89.}  Es ist das Syntagma Steine/ Farben/ Worte/ Begriffe/ Menschen, das hier von Interesse ist. Denn es ruft die Vorstellung auf, dass sich Neues nicht nur aus Steinen oder Worten, sondern auch aus Menschen gewinnen ließe. Alles andere als explizit ist hier eine Form des \emph{engineering} angesprochen, die in Literatur und Philosophie schon lange reflektiert wird und im kollektiven Imaginären unserer Tage, vermehrt im populären Film auftaucht: das \emph{human engineering}.\footnote{Ich entnehme diese Formulierung dem Hauptwerk Günther Anders\textquoteright{}, aus dem ich im Folgenden zitiere, vgl.: \cite[][]{anders1980a}, S. 37. Zum Auftauchen dieser Variante der Ingenieurfigur im aktuellen Film vgl. die materialreiche Studie: \cite[][]{meyer2013a}. Für die Literatur exemplarisch ist Mary Shellys \emph{Frankenstein} (1818).} \par Bereits 1980 skizziert der Philosoph Günther Anders im zweiten Teil seines Hauptwerks \emph{Die Antiquiertheit des Menschen} die Möglichkeit, \emph{»aus Lebewesen andere Lebewesen zu erschaffen«}\footnote{\cite[][]{anders1981a}, S. 23.}  und verwendet hierfür den Begriff einer »Ingenieurarbeit am Menschen«, oder den des »\emph{Human Engineering}«.\footnote{\cite[][]{anders1980a}, S. 37.}  Für unsere Perspektive entscheidend ist, dass Anders in diesem Zusammenhang vom Auftauchen einer neuen Figur spricht, die er in einer Verallgemeinerung den \emph{homo creator} nennt: »Mit dem Titel ›homo creator‹ meine ich die Tatsache, daß wir imstande sind, richtiger: uns instandgesetzt haben, aus Natur \emph{Produkte} zu erzeugen, die nicht (wie das aus Holz gebaute Haus) in die Klasse der ›Kulturprodukte‹ gehören, sondern in die der \emph{Natur}.«\footnote{\cite[][]{anders1981a}, S. 21.}  Während, nach Anders, der \emph{homo faber} noch Tunnel, Brücken, Häuser, ich ergänze, Autos, Raketen, iPhones hervorgebracht habe, produziere der \emph{homo creator} Natur. Bemerkenswert: dass Anders, um die Figur des \emph{homo creator}, deren Aufkommen von der Entstehung der Atombombe überschattet sei, zu charakterisieren, auf Prometheus rekurriert.\footnote{Anders spricht in diesem Zusammenhang von einer »dritten industriellen Revolution«. Diese habe mit der Atombombe, dem Umstand, dass wir pausenlos unseren Untergang produzieren, eingesetzt. Siehe \cite[][]{anders1981a}, S. 20.}  Einmal mehr ist es diese mythologische Gestalt, mit der ein neu entstehendes Gesicht des Ingenieurs vermessen wird:\begin{quote}
\par Mit dem Ausdruck »Hybridität« haben wir gewöhnlich die Figur des Prometheus verbunden, unter dessen Bild sich unsere Väter und auch wir selbst noch in den letzten 175 Jahren (von Goethe über Shelley und Ibsen bis zu Sartres »Mouches«) allegorisch gesehen hatten. Wenn wir uns fragen, ob diese Figur noch gilt; ob sich ihre allegorische Repräsentanz auch für unsere »Human Engineering« treibenden Zeitgenossen bewahrt hat, kommen wir zu folgenden zweideutigen Antworten. {[}...{]}\par Maßlos anmaßende Ansprüche erheben auch sie zwar \textendash{} nur ebenso anmaßende, daß sie sich selbst als unangemessen verwerfen.\par Zerfleischung erleiden auch sie zwar \textendash{} nur eben nicht deshalb, weil ein Zeus ihre zu hochfliegenden Ambitionen bestrafte, sondern weil sie sich selbst züchtigen für die Tatsache, ihrer »Zurückgebliebenheit«, für die »Schande ihrer Geburt«.\footnote{\cite[][]{anders1981a}, S. 49.} 
\end{quote}
\par Was Anders hervorkehrt ist, dass jener Ingenieur, der nicht mehr Kultur, sondern eben Natur (»Seiendes«)\footnote{\cite[][]{anders1981a}, S. 21.}  konstruiert, in dem seltsamen Paradox gefangen ist, sich einerseits zu einem Herren aufgeschwungen, zugleich aber seine Freiheit als Mensch preisgegeben zu haben. In dem Moment, in dem der Mensch selbst zum Gegenstand des \emph{engineering} werde, sei seine Existenz in neuer Weise aufs Spiel gesetzt.\par Und genau an diesem Punkt setzt der Hollywood-Film \emph{I am Legend} (2007) an, auf den für das Auftauchen des \emph{human engineer} im kollektiven Imaginären unserer Zeit exemplarisch ein Schlaglicht geworfen werden soll. Sein Ausgangspunkt sind die katastrophalen Folgen einer »Ingenieurarbeit am Menschen«: Ein Masernvirus, das von Menschen umgebaut wurde, um Krebs zu heilen, ist in eine fatale Richtung mutiert. Es bewirkt, dass sich Menschen in sogenannte \emph{darkseekers} verwandeln, das heißt, tollwutähnliche Symptome entwickeln, das Tageslicht meiden und in Aggressivität gegen nicht Infizierte verfallen. Die Geschichte dieser Manipulation erläutert eine Virologin zu Filmbeginn so: »Take something designed by nature and reprogram it to make it work for the body rather than against it. {[}...{]} The measles virus which has been engineered at a genetic level to be helpful rather than harmful.«\footnote{\cite[][]{lawrence2007a}, 01:14.}  Ebendieses Vorhaben, \emph{to be helpful}, hat sich in sein Gegenteil verkehrt. Entstanden ist nicht ein fitterer, sondern ein sich selbst gefährdender Mensch, was den Auftritt von Dr. Robert Neville, dargestellt von Will Smith, vorbereitet, der die katastrophalen Folgen dieses \emph{human engineering} rückgängig machen will. Neville, wissenschaftliche Koryphäe, Familienvater und US-Soldat, versucht zunächst im Labor an Ratten ein Heilmittel gegen das von Menschenhand geschaffene Problem zu entwickeln. Es gelingt ihm schließlich, sich auf einem seiner Streifzüge durch das nahezu ausgestorbene New York einer Infizierten zu bemächtigen. Aus dem Rohstoff ihres Körpers schafft er es, ein Heilmittel zu gewinnen.\par Bemerkenswert ist dieser eher seltene Fall eines afroamerikanischen Ingenieurhelden auch deshalb, weil hier die idealtypisch gezogene Achse von Produktion und Reflexion gewissermaßen porös wird. Beobachten wir nämlich, mit welchen Objekten Neville hantiert, sehen wir, dass diese ihn, denken wir an seinen Computer oder seine Spritzen, die er zum Durchspielen von Lösungen verwendet, sowohl als eine Figur der Reflexion als auch als eine der Produktion zeigen: Hier ist besonders an das Filmende zu denken, an dem Neville an eine der letzten Überlebenden ein Reagenzglas übergibt, welches das von ihm hergestellte Gegenpräparat enthält, ein physisches Produkt, um das Überleben der Menschheit zu sichern. Neben der Vorstellung, dass das Know-how dieses, mit Anders gesprochen, \emph{homo creator} die Gesellschaft retten könne, kehrt mit ihr auch ein anderer Baustein aus der Imaginationsgeschichte des Ingenieurs wieder: \emph{Human engineering} umfasst die Errettung der Menschheit ebenso wie die potenzielle Katastrophe. Man traut dieser Figur, tätig in ihrem mit Computern und Reagenzgläsern ausgestatteten Labor, zu, die Welt von einem tödlichen Erreger zu befreien wie auch sie in eine Pandemie zu stürzen, Viren ebenso artifiziell hergestellt wie Impfstofftechnologien (Stichwort: \emph{immune engineering}) entwickelt zu haben. Im Gesicht des \emph{human engineer}, einem der gegenwärtigsten Figuren in der Imaginationsgeschichte des Ingenieurs, überblenden sich Vorstellungen von Rettung und Untergang, Konstruktion und Destruktion, aber eben auch Reflexion und Produktion, teuflischem Kalkül und lebensrettender \texttt{Substanz}.%
	%
%
	%
		%
			%
			%
			%
				\selectlanguage{ngerman}%
			%
			%
				%
					
						\chapter[head={Grammatik eines Hoffnungs\-trägers}, tocentry={GRAMMATIK EINES HOFFNUNGS\-TRÄGERS}]{Grammatik eines Hoffnungs\-trägers}%
					
					\-
					\par			
					\newpage%
					\noindent%
				%
			%
			%
				%
					\regularfontdefault%
				%
			%
		%
	%
	%
	\par\noindent Über die weitere Karriere des Ingenieurs als gesellschaftlicher Hoffnungsträger wird man sich aller Voraussicht nach keine Sorgen machen müssen. Denn in Momenten gesellschaftlicher Krisen, so eine These dieses Essays, taucht die \emph{Imago} des Ingenieurs als handlungsorientierende Größe \textendash{} wenn auch vielfach variiert \textendash{} immer wieder neu auf. Neben dem zuletzt genannten \emph{human engineer} ist es gegenwärtig besonders die Vorstellung vom \emph{climate} oder \emph{geo engineer(ing)}, die Hoffnung also, globale ökologische Abläufe durch technisches Know-how korrigieren zu können, die als ein Ausweg aus einem der größten Dilemmata unserer Zeit, der Klimakrise, beschworen wird. Es ist dabei noch keinesfalls klar, wie das Gesicht einer solchen Hoffnungsfigur gezeichnet sein wird, doch geben uns die Erzählungen eines Jules Verne oder Alfred Döblin, die schon früh mit literarischen Mitteln weit ausholende Interventionen in das Ökosystem evozieren, erste Hinweise darauf.\footnote{\cite[][]{buettner2021a}.}  Aussichtsreicher als Vorhersagen über den weiteren Verlauf dieser Imaginationsgeschichte zu treffen ist es, auf das hier Gesagte zurückzublicken. Unter drei Gesichtspunkten möchte ich abschließend darüber nachdenken, was aus der Fülle der hier diskutierten Quellen für die Frage zu gewinnen ist, wie die \emph{Imago} des Ingenieurs funktioniert: Wie steht es um die verschiedenen Grenzen, um die historischen Konjunkturen und schließlich um die titelgebende Grammatik des Ingenieurs als gesellschaftlicher Hoffnungsträger?
\subsection[Grenzen]{Grenzen}
\par Das Konzept der Figur hat mir in diesem Aufsatz ein Instrumentarium an die Hand gegeben, um die Breite der Vorstellungen zu vermessen, die sich seit dem Ersten Weltkrieg um den Ingenieur herum gebildet haben. Eine Schwierigkeit des Figurenkonzeptes besteht zweifellos darin zu suggerieren, dass der Ingenieur (wie jede andere Figur, auf die dieses Konzept angewendet wird) \texttt{eine~nach} außen hin, das heißt gegenüber anderen Figuren klar abgrenzbare Entität sei. Das genaue Gegenteil ist aber der Fall. Einige der Einzelanalysen dieses Essays haben implizit erkennen lassen, dass der Ingenieur an eine Reihe anderer Figuren grenzt. So lässt sich bspw. im Rückblick auf die drei Hauptabschnitte sagen, dass der Ingenieur in seiner Variante als kleiner Mann (Abschnitt II) der Figur des Bürokraten ähnelt, der zerstörerische Ingenieur (Abschnitt III) jener des \emph{mad scientist} und schließlich der Ingenieur in seiner seltenen Spielart als eine Figur der Reflexion oder Theorie (griech. \emph{theoréein}; betrachten) (Abschnitt IV) der eines Mathematikers.\par Manchmal, aber nicht immer treten Ingenieure also servil wie Bürokraten, destruktiv wie \emph{mad scientist} und reflektierend wie Mathematiker auf. Solche Überlappungen zeigen die \emph{Imago} des Ingenieurs nicht als eine glasklar abgetrennte Größe, sondern als mit anderen Figuren verschlungen. Diese Randzonen zu sehen ist deshalb so wichtig, weil sich erst von ihnen aus der eigentliche Kern der Ingenieurfigur zeigt: das ihr eigene Versprechen, eine bessere Welt aufgrund von technischer Expertise herzustellen zu können. Keine Beamtenfigur, kein verrückter Professor und kein grübelnder Mathematiker schließt diese der Ingenieurfigur eigene Vorstellung ein.\par Die Aufzählung dieser dem Ingenieur benachbarten Figuren, die sich übrigens problemlos ergänzen ließe, bspw. um den Erfinder, Gelehrten oder auch Playboy,\footnote{Skulí Sigurdsson verdanke ich den Hinweis auf den Ingenieur als Partygänger, von ihn bewundernden Frauen umgeben und mit einem Cocktailglas in der Hand. In dieser Spielart ähnelt der Ingenieur dem Playboy und steht in scharfem Kontrast zum traditionell als misanthrop und frauenfeindlich dargestellten Gelehrten. Kellermanns Mac Allan trägt Züge dieser Variante.}  erlaubt es, eine Grenze noch ganz anderer Art sichtbar zu machen. Anders gesagt, gehört es zu den Funktionsweisen der \emph{Imago} vom Ingenieur als einem gesellschaftlichen Hoffnungsträger, dass sie ihrerseits Grenzen produziert. In die mediale Sprache unserer Tage gefasst, handelt es sich bei dieser Vorstellung nämlich um eine \emph{Imago} vom ›weißen Mann‹; in dem Sinne, dass sie, wie wir gesehen haben, in Hinsicht auf \emph{\texttt{gender}}, aber auch \emph{race} stark gebunden ist und dementsprechende Ausschlüsse produziert. Wenn man sich klarmacht, dass \texttt{unsere} \texttt{Vorstellungen} die soziale Welt, das heißt Handlungen und Entscheidungen genauso mitbestimmen wie andersherum Tatsachen der sozialen Welt unsere Vorstellungen inspirieren, wird man sagen können, dass diese wirkmächtige \emph{Imago} des 20. und 21. Jahrhunderts eine geschlechtliche und ethnische Diversifizierung des Ingenieurs in der Gesellschaft erschwert hat. Die Erfolgsgeschichte dieser tendenziell weißen, männlichen \emph{Imago} hat in diesem Sinne einen langen Schatten auf Teile unserer Gesellschaft geworfen, aus dem sich herauszulösen ein noch nicht abgeschlossenes Projekt darstellt.
\subsection[Konjunkturen]{Konjunkturen}
\par Doch bleiben wir noch einen Moment bei der Erfolgsgeschichte und ihren Zäsuren. In groben Zügen ergibt sich auf der Grundlage der hier behandelten Quellen das Bild einer Blütezeit des Ingenieurs als Hoffnungsträger nach 1918 und um 2000 sowie das einer Krise dieser Vorstellung in der Zeit des Kalten Krieges. Etwas kleinteiliger gesagt, entsteht nach dem Ersten Weltkrieg als Antwort auf dessen tiefgreifende soziale Erschütterungen besonders an den Extremen des politischen Spektrums und über nationale Grenzen hinweg die Vorstellung vom Ingenieur als einer Figur, die auf der Grundlage technischer Rationalität gesellschaftliche Spannungen lösen könne. Diese \emph{Imago} wird in der Frühphase des Kalten Kriegs \textendash{} auch hier: über nationale und ideologische Konfliktlinien hinweg \textendash{} wiederbelebt, wobei das Klima des \texttt{(Wieder-)Aufbaus} und der atomaren Aufrüstung die hierfür entscheidenden Voraussetzungen bilden. Allerdings ist der Ingenieurmythos in dieser Zeit brüchiger geworden: Die Verheißung von gesellschaftlichem Fortschritt auf der Grundlage von Technik steht nach dem \emph{Katastrophenzeitalter} (Hobsbawm) unter Druck. Das zeigt exemplarisch die Karriere eines Wernher von Braun, ein großer Mann, dem schließlich der Prozess gemacht wird, sowie aus dem Bereich der Literatur die Figur des Walter Faber aus Max Frischs Erfolgsroman. Die Vorstellung, dass technische Expertise die Welt verbessere, gewinnt um die Jahrtausendwende (Stichwort: Silicon Valley) wieder an Rückenwind. In den \textendash{} bildlich gesprochen \textendash{} Physiognomien jener Männer, die diese Verheißung nun verkörpern, werden Gesichtszüge des Ingenieurs als gesellschaftlicher Hoffnungsträger sichtbar, wie wir ihn aus der Zwischenkriegszeit kennen. Bausteine des nunmehr einhundert Jahre alten Ingenieurmythos werden hier aufgegriffen und durch neue Medien wiederbelebt. Daneben ist zu beobachten, wie sich die \emph{Imago} des Ingenieurs heute in Krisendebatten unserer Zeit verästelt: In den Silhouetten des \emph{human} oder \emph{climate engineer} ist die Hoffnung, dass schwer kontrollierbare Bedrohungen auf der Grundlage technischer Rationalität beherrschbar würden, wiederzuerkennen. Das Wort \emph{engineering} samt seiner Assoziationen eines kontrollierten Machens scheint, wo von den drängendsten Fragen unserer Zeit die Rede ist, wieder an Attraktivität zu gewinnen.\par Diese historischen Tendenzen verstehen sich als Hypothesen, die durch weitere materialorientierte Untersuchungen zu überprüfen sein werden. Grundsätzlich ist für jede der drei heißen Phasen des Ingenieurmythos eine Ko-Präsenz von Ingenieuridolatrie und -kritik feststellbar, wenn auch die Gewichte zwischen diesen beiden Polen sowie die diskursiven und medialen Orte, an denen diese jeweils artikuliert werden, variieren.
\subsection[Grammatik]{Grammatik}
\par Das letzte Wort dieses Essays soll nun der titelgebenden Grammatik des Ingenieurs als gesellschaftlicher Hoffnungsträger gelten. Es geht dabei um die Frage nach den Gesetzmäßigkeiten, die sich über seine sich wandelnde Physiognomie hinweg beobachten lassen. Vergegenwärtigen wir uns, dass der Begriff der Grammatik, vom griechischen Wort für Buchstabe kommend, zweierlei meint: zum einen die wissenschaftliche \emph{Betrachtung} von Sprache, zum anderen die \emph{Regelsysteme} in der Sprache.\footnote{\cite[][]{behse1974}, hier S. 846 und S. 852.}  Auch der vorliegende Essay ist im Sinne dieser doppelten Wortbedeutung aufgebaut: Es ging in den Abschnitten II bis IV darum, den Ingenieur aus unterschiedlichen Perspektiven, nach seiner Größe, seinen Orten und Objekten hin, zu \emph{betrachten}. Und es soll nun darum gehen, im Sinne der zweiten Bedeutung von Grammatik, die allgemeinen \emph{Regularitäten} dieser \emph{Imago} genauer zu benennen. Welche Strukturen sind über die Spannungsfelder groß/klein, konstruktiv/destruktiv, produzierend/reflektierend hinweg zu beobachten? Zu dieser Grammatik des Ingenieurs als gesellschaftlicher Hoffnungsträger abschließend drei Thesen:\begin{quote}
\par \emph{kontexttranszendent}
\end{quote}
\par Wann immer Ingenieure als gesellschaftliche Hoffnungsträger evoziert werden, ist eine Übertragung vom Kontext der Technik in einen anderen zu beobachten. Die Verheißung des Ingenieurs ist in der Vorstellung begründet, dass sich berufstypische Expertisen, wie das rationale Planen und Konstruieren, auf andere Felder, bspw. Politik und Gesellschaft, Klima und Körper übertragen ließen. Dieses Prinzip, das im Kern der Ingenieurimago arbeitet, erklärt die Kombinierbarkeit dieser Figur mit einer derartigen Fülle anderer gesellschaftlicher Handlungsfelder.\begin{quote}
\par \emph{archaisch und neu}
\end{quote}
\par Wiederkehrend in der Gestaltung des Ingenieurs als gesellschaftlicher Hoffnungsträger ist auch die spannungsreiche Überlagerung von, salopp gesagt, Alt und Neu. Hinter dem Ingenieur, der um 1900 als eine neuartige Figur auf die Bühne der politischen Imagination tritt, stehen geläufige Mythen wie Prometheus oder Faust. Damit diese sozial wirkmächtig werden, müssen sie nicht nur weitergeformt, sondern in den jeweils neuen Medien einer Epoche, Film, Fernsehen, Internet, auch mit Gegenwärtigkeit aufgeladen werden. Der Ingenieur als gesellschaftlicher Hoffnungsträger steht als Fortsetzung archaischer Figuren also mit einem Bein in der Mythologie, mit dem anderen am jeweiligen Puls seiner Zeit.\begin{quote}
\par \emph{unbestimmt}
\end{quote}
\par Eine dritte Gesetzmäßigkeit, vielmehr Voraussetzung für das Funktionieren dieser Hoffnungsfigur liegt in ihrer Unbestimmtheit. Sie erschließt sich jedoch nur, wenn wir den Ingenieur nun nicht in seinen einzelnen Ausprägungen, sondern in seiner nunmehr über 100 Jahre umfassenden Gesamtheit betrachten: Um von den unterschiedlichsten politischen Lagern und für die verschiedenen gesellschaftlichen Handlungsfelder zu einem Hoffnungsträger stilisiert werden zu können, ist Unschärfe nötig. Erinnern wir uns noch einmal an die physiognomische Schemenhaftigkeit von Hausmanns Figuren. Es ist die dort sichtbar werdende Unbestimmtheit, die jene Fülle an Gestaltungen erst möglich macht, die dieser Essay rekonstruiert hat. Eine andere Möglichkeit, um diese die Karriere des Ingenieurs als gesellschaftlicher Hoffnungsträger konstituierende Dimension zu fassen, ist Albrecht Koschorkes Versuch zu entnehmen, Kultur- und Erzähltheorie zu verbinden. Der Literaturwissenschaftler hat dort davon gesprochen, dass der Erzählvorgang »ein zu hoher Komplexität befähigtes, sich in sich abschließendes und doch poröses Gebilde mit eigenen Gesetzen zusammen{[}füge{]}«.\footnote{\cite[][]{koschorke2012a}, S. 21.}  Vielleicht könnte man sich die Ingenieurfigur nicht in ihren einzelnen Manifestationen, sondern als Gesamtphänomen als ein solches poröses Gebilde vorstellen, eine in sich abschließende Erzählung, die durchlässig genug geblieben ist, um seit nunmehr über 100 Jahren immer aufs Neue erzählt und wohl auch künftig weitererzählt zu werden.%
	%
%
	%
		%
			%
			%
			%
				\selectlanguage{ngerman}%
			%
			%
				%
					%
						\chapter[head={}, tocentry={ANHANG}]{Anhang}%
					
					\-
					\par			
					\newpage%
					\noindent%
				%
			%
			%
				%
					\regularfontdefault%
				%
			%
		%
	%
	%
	%
	%
%



% ------------------------------------------------------------------------
% Annex Title Page
% ------------------------------------------------------------------------


\selectlanguage{ngerman}
\cleardoublepage{}


% ------------------------------------------------------------------------
% Index
% ------------------------------------------------------------------------

\printindex{}

% ------------------------------------------------------------------------
% Bibliographie (always texput)
% ------------------------------------------------------------------------



\IfFileExists{texput.bib}{
	\smallfontdefault
	\raggedcolumns
	\begin{multicols}{2}
		\RaggedRight
		\subsection[Literatur]{Literatur}%
		\printbibliography[heading=none]{}
	\end{multicols}
	\newpage	
}{}

% ------------------------------------------------------------------------
% LoF
% ------------------------------------------------------------------------

\ifbool{printlof}{%
	\smallfontdefault
	\raggedcolumns
	\begin{multicols}{2}
		\RaggedRight
		\subsection[Bildnachweise]{Bildnachweise}%
		\begingroup%
		\renewcommand*{\addvspace}[1]{}%
		\listoffigures
		\endgroup%
	\end{multicols}%
	\newpage%	
}{}

% ------------------------------------------------------------------------
% Thanks (links) Postface Page / Über den Autor (rechts)
% ------------------------------------------------------------------------
%
\smallfontdefault%
\raggedcolumns%
\begin{multicols*}{2}%
	\RaggedRight%
	\setlength{\parindent}{\genericindent}%
	%
		\subsection{Dank}%
		\par Mein Dank gilt Nils Güttler (Zürich), Niki Rhyner (Zürich) und Max Stadler (Berlin), die diesen Essay von Anfang an voller Ideen und mit viel Engagement begleitet haben. Hinweise, die den Aufbau dieses Essays entscheidend mitbestimmt haben, verdanke ich Lin Chalozin-Dovrat (Tel Aviv), Menachem Fisch (Tel Aviv), Anke te Heesen (Berlin), Skúli Sigurdsson (Berlin), Jens Thiel (Berlin) und Karl Wagner (Wien). Ich danke den Teilnehmer*innen des Workshops zur Imaginationsgeschichte des Ingenieurs (ETH Zürich, Universität Zürich 2016) sowie dem Redaktionsteam des intercom-Verlags für die Gelegenheit, an dem Projekt, wissenschaftliches Publizieren neu und innovativ zu denken, mitwirken zu können.%
		\vfill%
		\null%
		\columnbreak
	%
	%
		%
			\subsection*{Über den Autor}%
		%
		\par Robert Leucht, 1975 in Wien geboren, ist Literaturwissenschaftler. Zu seinen Forschungsinteressen zählen die Geschichte der literarischen Utopie, die Exilliteratur und der Zusammenhang von Literatur und Technik. Seit 2019 ist er Professor für Neuere Deutsche Literatur an der Universität Lausanne.%
	%
\end{multicols*}%

% ------------------------------------------------------------------------
% Imprint Page
% ------------------------------------------------------------------------
\KOMAoptions{parskip=true}%
%
\smallfontdefault%
\raggedcolumns%
\begin{multicols*}{2}%
	\RaggedRight%
	%
		%
			\subsection*{Mono}%
			\vspace{-1.25em}%
		%
		\par Mono ist die neue Taschenbuchreihe im intercom Verlag. Mono bietet eine Plattform für Autor*innen, die sich auf pointierte und überraschende Art mit der Geschichte und Gegenwart von Wissenschaft und Technik, Umwelt und Arbeit, Alltag und politischer Ökonomie auseinandersetzen. Im Frühjahr 2022 folgt Beate Fricke mit Mono 02: \emph{Silk, Stone, Slaves}. \par Entstanden ist Mono zwischen 2019 und 2021 als Kooperation mit dem gta Verlag. Zusammen haben wir die Infrastruktur für ein hybrides Publikationsformat entwickelt, das Website (open access) und gedrucktes Buch konsequent zusammen denkt \textendash{} technisch und gestalterisch.\par Konzept\\ Moritz Gleich, Niki Rhyner, Max Stadler\par Redaktion\\ Ines Barner, Nils Güttler, Niki Rhyner, Max Stadler\par Gestaltung\\ Reinhard Schmidt, Nadine Wüthrich\par Entwicklung\\ Urs Hofer%
		\vfill%
		%
			\null%
		%
		\columnbreak%
	%
	\subsection*{Impressum}%
	\vspace{-1.25em}%
	\par Mono 01\\ Robert Leucht\\ Der Ingenieur: Grammatik eines\\ Hoffnungsträgers\par Abbildung Umschlagrückseite\\ Raoul Hausmann, \emph{Die Ingenieure} (1920), © 2022, ProLitteris, Zurich\par Korrektorat\\ Franziska Schwarzenbach\par Bandbetreuung\\ Ines Barner, Max Stadler\par Druck\\ LaBuonaStampa, Schweiz\par Bindung\\ Legatoria Mosca, Schweiz\par Druckkostenbeitrag\\ Publié avec un subside de la Commission des publications de la Faculté des lettres de l\textquoteright{}Université de Lausanne.\\ Mit Unterstützung der Professur für\\ Wissenschaftsforschung, ETH Zürich.\par open access\\ intercom-mono.com/01\\ DOI: 10.5281/zenodo.6225260\\ CC BY-NC-ND\par Kontakt\\ \href{mailto:\&\#x69;\&\#x6e;\&\#102;\&\#x6f;\&\#64;\&\#x69;\&\#x6e;\&\#x74;\&\#x65;\&\#x72;\&\#99;\&\#x6f;\&\#x6d;\&\#x76;\&\#x65;\&\#x72;\&\#x6c;\&\#x61;\&\#103;\&\#x2e;\&\#x63;\&\#x68;}{info@intercomverlag.ch}\\ \href{http://www.intercomverlag.ch}{www.intercomverlag.ch}\par ⁣1. Auflage\\ Printed in Switzerland\\ ISBN 978-3-9524954-8-3\\ ISSN 2813-0111\\ © Text: Robert Leucht\\ © 2022 intercom Verlag, Zürich%
\end{multicols*}%
\newpage%

\cleardoublepage
\end{document}